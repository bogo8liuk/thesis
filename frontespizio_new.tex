\documentclass[10pt,a4paper]{article}
\usepackage[italian]{babel}
\usepackage{newlfont}
\usepackage{listings}
\usepackage{tikz}
\usepackage{enumitem}
\usepackage{hyperref}

\lstset{language=Haskell
    , basicstyle={\small\ttfamily}
}

\textwidth=450pt\oddsidemargin=0pt

\begin{document}
\begin{titlepage}

\begin{center}

{{\Large {\textsc {Alma Mater Studiorum $\cdot$ Universit\`a di Bologna}}}} \rule[0.1cm]{15.8cm}{0.1mm}

\rule[0.5cm]{15.8cm}{0.6mm}
{\small {\bf SCUOLA DI SCIENZE\\
Corso di Laurea in Nome corso di Laurea }}

\end{center}

\vspace{15mm}
\begin{center}

{\LARGE
    {\bf COMPILATORE PER LINGUAGGIO DI}
}\\
\vspace{3mm}
{\LARGE
    {\bf PROGRAMMAZIONE FUNZIONALE}
}\\
\vspace{3mm}
{\LARGE
    {\bf SPERIMENTALE}
}\\

\end{center}

\vspace{40mm}
\par
\noindent
\begin{minipage}[t]{0.47\textwidth}

{\large
    {\bf Relatore:\\
        Chiar.mo Prof.\\
        CLAUDIO SACERDOTI COEN
    }
}

\end{minipage}

\hfill
\begin{minipage}[t] {0.47\textwidth}\raggedleft
{\large
    {\bf Presentata da:\\
        LUCA BORGHI
    }
}

\end{minipage}

\vspace{20mm}
\begin{center}

{\large
    {\bf Sessione\\
        III sessione
        2021-2022
    }
}%2018-2019

\end{center}

\end{titlepage}

\textwidth=450pt\oddsidemargin=0pt

\section{Introduzione}

%TODO: dove viene descritto il problema, la subsection può essere rimossa
\subsection{Il linguaggio}

\subsection{Haskell come modello}
Il compilatore è stato sviluppato mediante il linguaggio \textit{Haskell}, anch'esso linguaggio di programmazione funzionale.
Inoltre, il linguaggio target di langgg è \textit{Core}, un formalismo fortemente basato su \textit{System F}, una
variante tipata del lambda-calcolo che introduce un meccanismo di quantificazione universale sui tipi. Core, inoltre,
viene utilizzato da GHC (compilatore Haskell) come rappresentazione intermedia di Haskell. All'interno di questo
contesto, GHC espone delle API che permettono al cliente di utilizzare le funzionalità del compilatore (cfr. [2]); è quindi
possibile creare e manipolare programmi Core mediante le API, le quali sono state scritte in Haskell. Per quest'ultimo
motivo, Haskell è stato scelto come linguaggio di implementazione del compilatore di langgg, tuttavia, nel contesto
del progetto, Haskell ha anche un altro importante ruolo: alcune delle sue funzionalità sono state, direttamente
o indirettamente, fonte di ispirazione per il design di langgg. Nel prossimo paragrafo vengono presentate le principali
caratteristiche di langgg ed è possibile ritrovare la maggior parte di esse anche in Haskell. Un altro linguaggio
che è stato fonte di ispirazione, ma con minore impatto, è \textit{OCaml}, linguaggio multi-paradigma (funzionale non
puro), soprattutto per la sintassi di langgg e per il costrutto delle polymorphic variants (cfr. paragrafo sugli
sviluppi futuri).

\subsection{Caratteristiche del linguaggio}
Tra le principali caratteristiche del linguaggio vi sono:
\begin{description}
\item[Linguaggio funzionale puro] Come Haskell, langgg è un linguaggio funzionale puro. La nozione di linguaggio
di programmazione \textit{funzionale} è piuttosto lasca e non vi è una vera e propria definizione formale,
tant'è che il termine viene spesso utilizzato (e, talvolta, abusato) per indicare un linguaggio avente alcune
particolari specifiche attribuibili al paradigma di programmazione funzionale. langgg può essere quindi considerato
funzionale poiché, semplicemente, ha numerose caratteristiche proprie del paradigma funzionale, quali: tipi di dati
algebrici, pattern matching, funzioni di ordine superiore, immutabilità, polimorfismo parametrico etc.. Per quanto
riguarda la nozione di \textit{purità}, nel paradigma funzionale viene fatta spesso la distinzione
tra linguaggi puri e impuri; anche qui, non vi sono vere e proprie definizioni e la questione è spesso oggetto di
controversie. Una proposta di definizione è stata fornita da Amr Sabry in [1]: la purità ha a che fare con il passaggio
dei parametri, in particolare, un linguaggio L può essere
considerato puro se, dato un qualsiasi programma p scritto in L, le funzioni di p sono dei "\textit{mapping}" puri
dai valori in input ai valori in output, indipendentemente dal tipo di passaggio dei parametri.
\item[Type-system statico con type-inference] langgg è un linguaggio con type-system statico, inoltre, la fase di
type-checking garantisce la proprietà di type-safety. Il linguaggio permette anche di omettere le indicazioni di
tipo (type-hinting) nella definizione di simboli; in caso non vi sia type-hinting per la definizione di un simbolo,
il compilatore inferirà il tipo "più generale possibile" per il simbolo.
\item["Everything is an expression"] tutti i costrutti all'interno di langgg possono essere considerati espressioni
prive di side-effects; non vi sono costrutti di controllo o costrutti che modificano variabili di stato esterne
al contesto locale. I costrutti principali sono:
    \begin{enumerate}
    \item definizioni di: tipi, variabili, proprietà (che corrispondono alle type-classes di Haskell), istanze di
    proprietà, alias di tipi, firme di simboli; le definizioni di simboli hanno le seguenti grammatiche:
    \begin{lstlisting}
      S :=
          let var args = E

     MS :=
          let var MPM
    \end{lstlisting}
    \item espressioni date dalla seguente grammatica:
    \begin{lstlisting}
      E :=
          var
          datacon
          literal
          E E
          lam args -> E
          lam MPM
          S in E
          MS in E
          match E with PM

     PM :=
          ME -> E | ... | ME -> E

    MPM :=
          | ME ... ME = E | ... | ME ... ME = E

     ME :=
          var
          _
          literal
          datacon ME ... ME
          
    \end{lstlisting}
    \item un costrutto, valutato a compile-time (cfr. parsing degli operatori), per definire una "categoria" di
    operatori. Le categorie di operatori hanno degli identificatori per poterle nominare, inoltre definiscono le
    seguenti proprietà:
        \begin{enumerate}
        \item una lista di operatori appartenenti alla proprietà. Sono compresi gli identificatori di variabili (gli
        operatori stessi sono identificatori di variabili) che possono essere utilizzati con la sintassi degli operatori;
        \item una lista di categorie di operatori i quali avranno meno precedenza degli operatori della categoria
        corrente;
        \item una lista di categorie di operatori i quali avranno più precedenza degli operatori della categoria
        corrente;
        \item la fissità degli operatori.
        \end{enumerate}
        %TODO: aggiungi pezzo di codice per mostrare il costrutto
    \end{enumerate}
\item[Tipi di dati algebrici] langgg supporta anche i tipi di dati algebrici (che sono l'unico modo per definire nuovi
tipi). Con essi, vi è anche la nozione di \textit{data constructor}, ovvero un costrutto dotato di tipo che può essere
accompagnato da dati. Per eseguire operazioni con i tipi di dati algebrici vi è il costrutto del pattern matching
(cfr. grammatica delle espressioni) che permette di destrutturare un data constructor dai suoi dati associati.
\item[Polimorfismo parametrico] il sistema di tipi del linguaggio supporta le variabili di tipo. Questo tipo di
polimorfismo permette di avere singoli algoritmi per una moltitudine di tipi.
\item[Kind] Il linguaggio ha la nozione di \textit{kind} e permette la manipolazione di tipi di kind differenti.
Quindi, data la seguente definizione di tipo:
\begin{lstlisting}
type M a b = DataCon1 a | DataCon2 b | DataCon3 a Char
\end{lstlisting}
dove \textit{a} e \textit{b} sono variabili di tipo, è possibile, ad esempio, avere tipi della forma:
\begin{lstlisting}
M Int Char
M a b
M String
M x
M
\end{lstlisting}
\`E possibile altresì avere un tipo della forma:
\begin{lstlisting}
m Int
\end{lstlisting}
dove \textit{m} è una variabile di tipo a cui viene applicata il tipo \textit{Int}.
\item[Polimorfismo ad hoc] attraverso il meccanismo delle type-classes (all'interno del linguaggio vengono chiamate
proprietà), è possibile creare funzioni polimorfe che possono essere applicate ad argomenti di tipi differenti e, a
seconda dei tipi degli argomenti, viene "selezionata" la giusta implementazione. Ogni proprietà può avere uno, ma anche
più tipi associati; inoltre, i tipi associati alle proprietà possono essere anche variabili di tipo.

\end{description}

\section{Il compilatore}
\tikzstyle{process} = [rectangle, minimum width=3cm, minimum height=1cm, text centered, draw=black, fill=white!30]
\tikzstyle{object} = [rectangle, minimum width=3cm, minimum height=1cm, text centered, draw=black, fill=white!30]
\tikzstyle{arrow} = [thick,->,>=stealth]

\subsection{La struttura}
\begin{tikzpicture}[node distance=2cm]
\node (src) [object] {Codice sorgente};
\node (parser) [process, below of=src] {Parser};
\draw [arrow] (src) -- (parser);
\node (checker) [process, below of=parser] {Checker della correttezza};
\draw [arrow] (parser) -- (checker);
\node (typedBuilder) [process, below of=checker] {Costruzione dei token tipati};
\draw [arrow] (checker) -- (typedBuilder);
\node (prepare) [process, below of=typedBuilder] {Preparazione alla type inference};
\draw [arrow] (typedBuilder) -- (prepare);
\node (typeInf) [process, below of=prepare] {Type inference - type checking};
\draw [arrow] (prepare) -- (typeInf);
\node (CoreGen) [process, below of=typeInf] {Generazione codice Core};
\draw [arrow] (typeInf) -- (CoreGen);
\node (backend) [process, below of=CoreGen] {Backend};
\draw [arrow] (CoreGen) -- (backend);
\node (exe) [process, below of=backend] {Eseguibile};
\draw [arrow] (backend) -- (exe);
\node (desugar) [process, right of=prepare, xshift=7cm] {Desugaring};
\draw [arrow] (desugar) -- (checker);
\draw [arrow] (desugar) -- (typedBuilder);
\draw [arrow] (desugar) -- (prepare);
\draw [arrow] (desugar) -- (typeInf);
\draw [arrow] (desugar) -- (CoreGen);
\end{tikzpicture}

La struttura del compilatore è sequenziale, con eccezione fatta per la fase di \textit{desugaring}. Quest'ultima si
occupa di rimuovere il cosiddetto "zucchero sintattico" dalle strutture dati - che mantengono le informazioni del
programma - che, durante le varie fasi di compilazione, possono subire semplificazioni. Talvolta, alcune semplificazioni
vengono posticipate il più possibile per permettere al compilatore di restituire in output messaggi d'errore comprensibili
per l'utente. Un esempio ne è la trasformazione che subiscono le strutture dati che rappresentano il costrutto del
pattern matching. Il modulo che si occupa del desugaring si presenta come una libreria che espone delle API che agiscono
sulle strutture dati del compilatore. \`E compito del compilatore fare le chiamate al modulo di desugaring. Di seguito,
vi è una più ampia e precisa descrizione delle singole fasi di compilazione, con enfasi particolare sugli algoritmi
più interessanti utilizzati per risolvere i singoli sottoproblemi e su come le varie fasi interagiscono tra loro.

\subsection{L'albero di sintassi astratta}
Nella prima parte del compilatore viene eseguito il parsing del sorgente.
Questa operazione produce quindi l'albero di sintassi astratta (AST). Il codice che gestisce i "token" dell'AST è
all'interno del modulo \texttt{Compiler.Ast.Tree}; l'entry point dell'albero è dato dal token \texttt{Program}, il
quale, come nodi figli, ha delle \texttt{Declaration} che rappresentano i vari costrutti del linguaggio.

\begin{lstlisting}
newtype Program a = Program [Declaration a]

data Declaration a =
      ADT (AlgebraicDataType a)
    | AliasADT (AliasAlgebraicDataType a)
    | Intf (Interface a)
    | Ins (Instance a)
    | Sig (Signature a)
    | Let (SymbolDeclaration a)
    | LetMulti (MultiSymbolDeclaration a)
\end{lstlisting}

Di seguito vi è lo schema della prima parte del compilatore:

\begin{tikzpicture}[node distance=2cm]
\node (src) [object] {Codice sorgente};
\node (rawast) [object, below of=src] {Albero di sintassi astratta};
\draw [arrow] (src) -- node[anchor=west] {Parser} (rawast);
\node (rawastBuiltin) [object, below of=rawast] {Albero di sintassi astratta};
\draw [arrow] (rawast) -- node[anchor=west] {Aggiunta token built-in} (rawastBuiltin);
\node (rawastNames) [object, below of=rawastBuiltin] {Albero di sintassi astratta};
\draw [arrow] (rawastBuiltin) -- node[anchor=west] {Check esistenza dei nomi} (rawastNames);
\node (rawastArgs) [object, below of=rawastNames] {Albero di sintassi astratta};
\draw [arrow] (rawastNames) -- node[anchor=west] {Check numero degli argomenti} (rawastArgs);
\node (rawastAlias) [object, below of=rawastArgs] {Albero di sintassi astratta};
\draw [arrow] (rawastArgs) -- node[anchor=west] {Sostituzione degli alias di tipo} (rawastAlias);
\node (desugar) [object, right of=rawastAlias, xshift=7cm] {Desugaring};
\draw [arrow] (desugar) -- (rawastAlias);
\end{tikzpicture}

\subsubsection{Il parser}
Il parser è la prima componente del compilatore (dopo la lettura del sorgente). \`E stato scritto mediante la libreria
open-source \texttt{Parsec} [4], la quale si basa sul concetto di parser combinator monadico. Il codice risiede nel
modulo \texttt{Compiler.Syntax} ed ha una struttura gerarchica: parte definendo i combinators di "pezzi" primitivi dei
token dell'AST (\texttt{Compiler.Syntax.Lib.SimpleParser}), dopodiché, nel modulo \texttt{Compiler.Syntax.Grammar} vi
è il parsing vero e proprio dei token dell'AST e, infine, vi è l'entry point del parser, ovvero
\texttt{Compiler.Syntax.Parser}. \`E bene notare che all'interno del modulo \texttt{Compiler.Syntax.Lib.SimpleParser}
non sono visibili le API dell'AST, in quanto esso si occupa soltanto del parsing dei costrutti del linguaggio e non
della generazione dei token.

\subsubsection{Parsing degli operatori}
langgg permette di definire operatori che vengono trattati come simboli di variabili (funzioni). In questo contesto,
il linguaggio espone all'utente un costrutto particolare che permettere di definire le proprietà degli eventuali
operatori. Questo costrutto viene valutato durante il parsing e definisce una categoria di operatori, ecco un esempio:

\begin{lstlisting}
OPERATORS_CATEGORY {#
    name : Application ;
    operators : |>, <|, `applyTo ;
    lesser than : Comparison, Numeric ;
    greater than : Functor ;
    fixity : InfixRight ;
#}
\end{lstlisting}

Il primo "campo" è il nome della categoria ed è utile per poter identificare la categoria, infatti, il terzo e il
quarto campo definiscono rispetto a quali categorie gli attuali operatori hanno precedenza. Il secondo campo definisce
l'insieme di operatori che fanno parte della categoria. Infine, l'ultimo campo è una costante e definisce la fissità
degli operatori: infissa senza associazione, infissa con associazione a sinistra, infissa con associazione a destra,
postfissa o prefissa. La versione infissa è solo per gli operatori binari, mentre postfissa e prefissa solo per gli
operatori unari. Si può notare che nella lista di operatori vi è anche un simbolo di variabile preceduto dal simbolo
di backtick; questo è possibile, in quanto langgg permette di utilizzare i simboli di variabili come operatori
facendoli precedere dal carattere backtick.

\section*{Bibliografia}
\begin{enumerate}[label={[\arabic*]}]
    %[1]
    \item Amr Sabry - What is a purely functional language? - in: Journal of Functional Programming, Volume 8, Issue 1,
    January 1998, pp. 1 - 22
    %[2]
    \item Martin Sulzmann, Martin Odersky, Martin Wehr - Type Inference with Constrained Types - in: Theory and Practice
    of Object Systems · January 1999
    %[3]
    \item Stephen Diehl - Dive into GHC: Targeting Core - url: \url{https://www.stephendiehl.com/posts/ghc_03.html}
    %[4]
    \item libreria Parsec - url: \url{https://hackage.haskell.org/package/parsec}
\end{enumerate}

\end{document}
