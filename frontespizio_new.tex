\documentclass[12pt,a4paper]{article}
\usepackage[italian]{babel}
\usepackage{newlfont}

\textwidth=450pt\oddsidemargin=0pt

\begin{document}
\begin{titlepage}

\begin{center}

{{\Large {\textsc {Alma Mater Studiorum $\cdot$ Universit\`a di Bologna}}}} \rule[0.1cm]{15.8cm}{0.1mm}

\rule[0.5cm]{15.8cm}{0.6mm}
{\small {\bf SCUOLA DI SCIENZE\\
Corso di Laurea in Nome corso di Laurea }}

\end{center}

\vspace{15mm}
\begin{center}

{\LARGE
    {\bf COMPILATORE PER LINGUAGGIO DI}
}\\
\vspace{3mm}
{\LARGE
    {\bf PROGRAMMAZIONE FUNZIONALE}
}\\
\vspace{3mm}
{\LARGE
    {\bf SPERIMENTALE}
}\\

\end{center}

\vspace{40mm}
\par
\noindent
\begin{minipage}[t]{0.47\textwidth}

{\large
    {\bf Relatore:\\
        Chiar.mo Prof.\\
        CLAUDIO SACERDOTI COEN
    }
}

\end{minipage}

\hfill
\begin{minipage}[t] {0.47\textwidth}\raggedleft
{\large
    {\bf Presentata da:\\
        LUCA BORGHI
    }
}

\end{minipage}

\vspace{20mm}
\begin{center}

{\large
    {\bf Sessione\\
        III sessione
        2021-2022
    }
}%2018-2019

\end{center}

\end{titlepage}

\section{Introduzione}

%TODO: dove viene descritto il problema, la subsection può essere rimossa
\subsection{Il linguaggio}

\subsection{Haskell come modello}

\subsection{Caratteristiche del linguaggio}
Tra le principali caratteristiche del linguaggio vi sono:
\begin{description}
\item[Linguaggio funzionale puro] Come Haskell, langgg è un linguaggio funzionale puro. La nozione di linguaggio
di programmazione \textit{funzionale} è piuttosto lasca e non vi è una vera e propria definizione formale,
tant'è che il termine viene spesso utilizzato (e, talvolta, abusato) per indicare un linguaggio avente alcune
particolari specifiche attribuibili al paradigma di programmazione funzionale. langgg può essere quindi considerato
funzionale poiché, semplicemente, ha numerose caratteristiche proprie del paradigma funzionale, quali: tipi di dati
algebrici, pattern matching, funzioni di ordine superiore, immutabilità, etc.. Per quanto riguarda la nozione di
\textit{purità}, nel paradigma funzionale viene fatta spesso la distinzione
tra linguaggi puri e impuri; anche qui, non vi sono vere e proprie definizioni e la questione è spesso oggetto di
controversie. Una proposta di definizione è stata da Amr Sabry in "\textit{What is a purely functional language?}"... %cfh
\end{description}

\section{Il compilatore}

\end{document}
