\documentclass[10pt,a4paper]{article}
\usepackage[italian]{babel}
\usepackage{newlfont}
\usepackage{listings}
\usepackage{tikz}
\usepackage{enumitem}
\usepackage{blindtext}
\usepackage{hyperref}
\usepackage{proof}
\usepackage{amssymb}
\usepackage{amsmath}

\lstset{
    language=Haskell,
    basicstyle={\small\ttfamily}
}

\tikzstyle{process} =
    [rectangle, minimum width=3cm, minimum height=1cm, text centered, draw=black, fill=white!30]
\tikzstyle{object} =
    [rectangle, minimum width=3cm, minimum height=1cm, text centered, draw=black, fill=white!30]
\tikzstyle{arrow} = [thick,->,>=stealth]

\textwidth=450pt\oddsidemargin=0pt

\begin{document}
\begin{titlepage}

\begin{center}

{{\Large {\textsc {Alma Mater Studiorum $\cdot$ Universit\`a di Bologna}}}} \rule[0.1cm]{15.8cm}{0.1mm}

\rule[0.5cm]{15.8cm}{0.6mm}
{\small {\bf SCUOLA DI SCIENZE\\
Corso di Laurea in Nome corso di Laurea }}

\end{center}

\vspace{15mm}
\begin{center}

{\LARGE
    {\bf COMPILATORE PER LINGUAGGIO DI}
}\\
\vspace{3mm}
{\LARGE
    {\bf PROGRAMMAZIONE FUNZIONALE}
}\\
\vspace{3mm}
{\LARGE
    {\bf SPERIMENTALE}
}\\

\end{center}

\vspace{40mm}
\par
\noindent
\begin{minipage}[t]{0.47\textwidth}

{\large
    {\bf Relatore:\\
        Chiar.mo Prof.\\
        CLAUDIO SACERDOTI COEN
    }
}

\end{minipage}

\hfill
\begin{minipage}[t] {0.47\textwidth}\raggedleft
{\large
    {\bf Presentata da:\\
        LUCA BORGHI
    }
}

\end{minipage}

\vspace{20mm}
\begin{center}

{\large
    {\bf Sessione\\
        III sessione
        2021-2022
    }
}%2018-2019

\end{center}

\end{titlepage}

\textwidth=450pt\oddsidemargin=0pt

\tableofcontents
\newpage

\section{Introduzione}
Il paradigma funzionale nella programmazione non è certamente nuovo in ambito accademico: le sue origini risiedono a
parecchie decine di anni fa, grazie agli studi Alonzo Church sui formalismi del \textit{lambda-calcolo} negli anni '30.
Sebbene vi siano numerosi studi a livello accademico, il successo non è stato tale nel mondo industriale. Uno dei primi
linguaggi funzionali fu Lisp, sviluppato a fine anni '50, il quale è stato, in un certo senso,
pioniere di alcuni concetti e costrutti come la gestione automatica della memoria, funzioni di ordine superiore, la
ricorsione, etc.. Successivamente, negli anni '70, citando i più famosi, nacquero Scheme (dialetto dello stesso Lisp) ed ML.
Negli '80, è la volta di Erlang e Standard ML, nei '90 di OCaml, Haskell e Racket. Nonostante molti di questi linguaggi
vengano considerati come general-purpose, non avranno inizialmente molto successo come lo hanno avuto, invece, le
controparti dei linguaggi imperativi e dei linguaggi orientati agli oggetti. Un caso emblematico è quello di
Erlang, il quale nacque nell'ambito della telefonia come linguaggio proprietario dell'azienda Ericsson. Successivamente,
è stato adottato per molteplici tipi di applicativi, diventando un linguaggio di più ampio successo, soprattutto in ambiti
in cui la gestione della concorrenza è molto importante. Negli ultimi anni, infatti, c'è stata una ``riscoperta" del
paradigma funzionale in quanto quest'ultimo permette di scrivere codice in stile dichiarativo e componibile, fornendo
una maggiore manutenibilità e garantendo proprietà importanti dei programmi. Oltre al già citato Erlang, utilizzato ad
esempio nello sviluppo dell'applicativo Whatsapp \hyperlink{bibl30}{[30]}, altri linguaggi funzionali utilizzati
nell'industria sono Haskell, con cui è stata implementata la famosa blockchain Cardano
\hyperlink{bibl31}{[31]}, OCaml, con cui è stato scritto il primo compilatore Rust ed è largamente utilizzato da Facebook
in molti dei suoi software come Hack
\hyperlink{bibl32}{[32]} e Flow \hyperlink{bibl33}{[33]}, etc.. Inoltre, numerosi linguaggi di programmazione non
propriamente funzionali ispirano alcuni dei loro costrutti al paradigma funzionale. Esempi ne sono il costrutto di
pattern matching per Python e Rust oppure il costrutto di lambda-astrazione per C\# e JavaScript oppure le
funzioni come ``cittadini di prima classe" per Lua e Go, etc..

I linguaggi di programmazione funzionali hanno spesso caratteristiche molto diverse tra loro: linguaggi come Erlang e
Clojure hanno una tipizzazione dinamica, a differenza di altri linguaggi come Haskell e OCaml che effettuano un'analisi
statica; Haskell e Elm vengono considerati come linguaggi puri, mentre OCaml, F\# e molti altri linguaggi sono
considerati come impuri; Lisp, Scheme, Closure e Racket sono linguaggi omoiconici che manipolano \textit{S-expressions},
mentre altri linguaggi permettono la meta-programmazione utilizzando meccanismi differenti dalle \textit{S-expressions}, un
esempio è Template Haskell \hyperlink{bibl34}{[34]}.

Il progetto di tesi, consiste nello sviluppo di un compilatore per un nuovo linguaggio di programmazione
funzionale chiamato Fex. L'obiettivo iniziale è la costruzione di un linguaggio che racchiuda le caratteristiche
più interessanti degli odierni linguaggi di programmazione funzionali. In particolare Fex deve avere
un type-system statico che possa
garantire proprietà importanti a tempo di compilazione. Inoltre, caratteristiche che romperebbero la ``purità" del linguaggio
devono essere gestite sempre tramite il sistema di tipi: un esempio ne sono gli effetti (cfr. \hyperlink{Effetti}{Effetti}).
Tuttavia, gli obiettivi iniziali non sono stati completamente raggiunti, in particolare Fex risulta momentaneamente
un linguaggio minimale le cui \textit{features} sono riconducibili, per la maggior parte, a quelle di Haskell.
Il codice sorgente è disponibile al link \url{https://github.com/bogo8liuk/Fex-lang} (\hyperlink{bibl36}{[36]}).

\paragraph{Lavoro iniziale}
Prima dello sviluppo del compilatore Fex, c'è stata una preparazione in vista del prodotto finale: vi sono stati
alcuni brevi studi di fattibilità che riguardavano il nuovo linguaggio, in particolare sono stati scelti il target
di compilazione e il linguaggio con il quale scrivere il software per il compilatore. Inizialmente, come target di
compilazione è stata scelta la rappresentazione intermedia di Llvm \hyperlink{bibl35}{[35]} (Llvm IR) e OCaml come
linguaggio per il compilatore. Tuttavia, benché il codice Llvm IR offra portabilità su molte architetture e, al contempo,
buone prestazioni, questa prima scelta
sul target di compilazione è stata scartata. Il motivo risiede nel fatto che Fex rappresenta un linguaggio ad altissimo
livello e l'utilizzo di Llvm IR non renderebbe disponibile un garbage collector e, più in generale, un run-time system.
Ciò implicherebbe un oneroso lavoro per costruire l'astrazione esatta.
Successivamente, è stata valutata un'altra opzione: utilizzare una rappresentazione intermedia di un altro linguaggio ad alto
livello come target di compilazione. La scelta è ricaduta su Core, una rappresentazione intermedia di Haskell.
Questa scelta ha due grandi vantaggi: avere a disposizione un run-time system (quello di Haskell); Core e Fex hanno costrutti
molto simili.
Come conseguenza è cambiato anche il linguaggio con cui sviluppare il compilatore: la scelta è ricaduta su Haskell,
in quanto GHC (the Glasgow Haskell Compiler) espone delle API scritte in Haskell per costruire e manipolare
programmi Core.

\subsection{Note}
Prima di presentare l'articolo, è necessario fare alcune precisazioni utili alla lettura:
\begin{itemize}
    \item la sintassi degli identificatori di Fex è Haskell-like, quindi, tutto ciò che viene considerato come
    \textit{variabile} nel linguaggio ha lettera iniziale minuscola, mentre tutto ciò che viene considerato come
    \textit{valore} ha lettera iniziale maiuscola;
    \item i termini \textit{constraint} e \textit{predicato} sono considerati sinonimi (a meno di contesti particolari).
    Essi vengono intercambiati abbastanza liberamente;
    \item i termini \textit{schema di tipo} e \textit{poli-tipo} sono considerati sinonimi. Nell'articolo viene preferita
    per la maggior parte delle volte la prima forma;
    \item talvolta, è possibile che venga presentato codice sorgente (Haskell) del compilatore Fex, tuttavia, verranno
    isolati e mostrati solamente i punti più interessanti, rimuovendo i commenti e il codice non direttamente inerente
    al contesto. Altre volte, gli algoritmi e i concetti utilizzati nel compilatore vengono mostrati con uno
    pseudo-codice oppure con uno pseudo-linguaggio della logica. Questo per evitare di presentare dettagli implementativi
    poco utili ai fini delle spiegazioni;
    \item per quanto riguarda la struttura della tesi, nella prima parte, vengono presentate le principali caratteristiche di
    Fex senza entrare in eventuali dettagli implementativi, inoltre, verranno menzionati i formalismi principali a cui Fex
    si ispira. Dopodiché, verrà presentato il compilatore, il quale si divide in tre fasi principali:
    \begin{enumerate}
        \item nella prima fase, vi è la descrizione della costruzione dell'albero di sintassi astratta e di tutti
        i controlli e le manipolazioni sull'albero stesso necessari per la seconda fase;
        \item nella seconda fase, appaiono le varie nozioni di tipo di Fex, in particolare vi è la generazione delle
        symbol-tables che conterranno i token ``tipati";
        \item nella terza e ultima fase, vi è la generazione del codice Core che successivamente viene tradotto dal backend di
        Haskell in vari target di compilazione.
    \end{enumerate}
    Nell'ultima parte, vengono mostrati alcuni possibili sviluppi futuri di Fex, presentando le eventuali nuove caratteristiche
    e le possibili implementazioni nel compilatore Fex.
\end{itemize}

\subsection{Haskell e System-F come modelli}
Il compilatore è stato sviluppato utilizzando il linguaggio \textit{Haskell}, anch'esso linguaggio di programmazione
funzionale.
Inoltre, il linguaggio target di Fex è \textit{Core}, un linguaggio che estende \textit{System-F}
(cfr. \hyperlink{System-F}{System-F}). Core, inoltre,
viene utilizzato da GHC (compilatore Haskell) come rappresentazione intermedia di Haskell. All'interno di questo
contesto, GHC espone delle API che permettono al cliente di utilizzare le funzionalità del compilatore
\hyperlink{bibl3}{[3]}; è quindi
possibile creare e manipolare programmi Core mediante le API, le quali sono scritte in Haskell. Per quest'ultimo
motivo, Haskell è stato scelto come linguaggio di implementazione del compilatore di Fex, tuttavia, nel contesto
del progetto, Haskell ha anche un altro importante ruolo: alcune delle sue funzionalità sono state, direttamente
o indirettamente, fonte di ispirazione per il design di Fex. Nel prossimo paragrafo vengono presentate le principali
caratteristiche di Fex ed è possibile ritrovare la maggior parte di esse anche in Haskell. Un altro linguaggio
che è stato fonte di ispirazione, ma con minore impatto, è \textit{OCaml}, linguaggio multi-paradigma (funzionale non
puro), soprattutto per la sintassi di Fex e per il costrutto delle polymorphic variants (cfr.
\hyperlink{Varianti polimorfe}{Varianti polimorfe}).

\hypertarget{System-F}{\paragraph{System-F}}
I costrutti principali di Fex si basano su System-F, una versione ``tipata" del lambda-calcolo che introduce un meccanismo
di quantificazione universale sui tipi. Esso è definito con i seguenti quattro costrutti:
\[ e := x \; | \; e_1 \; e_2 \; | \; \lambda x. \; e \; | \; let \; x = e_1 \; in \; e_2 \]
Come vedremo in seguito (cfr. \hyperlink{Ricorsione}{Ricorsione}), questo formalismo risulta non essere Turing-completo.
Fex (come lo stesso
Haskell) basa i propri costrutti e il proprio type-system, nonché il suo algoritmo di inferenza dei tipi, su questo
formalismo, tuttavia, quest'ultimo viene esteso con altri costrutti che offrono una maggiore espressività.

\hypertarget{Caratteristiche del linguaggio}{\subsection{Caratteristiche del linguaggio}}
Tra le principali caratteristiche del linguaggio vi sono:

\paragraph{Linguaggio funzionale puro}
Come Haskell, Fex è un linguaggio funzionale puro. La nozione di linguaggio
di programmazione \textit{funzionale} è piuttosto lasca e non vi è una vera e propria definizione formale,
tant'è che il termine viene spesso utilizzato (e, talvolta, abusato) per indicare un linguaggio avente alcune
particolari specifiche attribuibili al paradigma di programmazione funzionale. Fex può essere quindi considerato
funzionale poiché, semplicemente, ha numerose caratteristiche proprie del paradigma funzionale, quali: tipi di dati
algebrici, pattern matching, funzioni di ordine superiore, immutabilità, polimorfismo parametrico etc.. Per quanto
riguarda la nozione di \textit{purità}, nel paradigma funzionale viene fatta spesso la distinzione
tra linguaggi puri e impuri; anche qui, non vi sono vere e proprie definizioni e la questione è spesso oggetto di
controversie. Una proposta di definizione è stata fornita da Amr Sabry \hyperlink{bibl1}{[1]}: la purità ha a che
fare con il passaggio
dei parametri, in particolare, un linguaggio $ L $ può essere
considerato puro se, dato un qualsiasi programma $ p $ scritto in $ L $, al variare delle strategie di valutazione
delle espressione di $ p $ (call-by-value, call-by-name, call-by-reference, call-by-need, etc.), non vi sono differenze
osservabili, escludendo la divergenza.

\paragraph{Type-system statico con type-inference}
Fex è un linguaggio con type-system statico. Questo significa che le proprietà sul sistema di tipi vengono verificate
a tempo di compilazione, infatti, la fase di
type-checking garantisce la proprietà di type-safety. Il linguaggio permette anche di omettere le indicazioni di
tipo (type-hinting) nella definizione di simboli; in caso non vi sia type-hinting per la definizione di un simbolo,
il compilatore inferirà il tipo ``più generale possibile" per il simbolo.

\paragraph{``Everything is an expression"}
Tutti i costrutti all'interno di Fex possono essere considerati espressioni
prive di side-effects; non vi sono costrutti di controllo o costrutti che modificano variabili di stato esterne
al contesto locale di un'espressione. I costrutti principali sono:
    \begin{enumerate}
    \item definizioni di: tipi, variabili, proprietà (che corrispondono alle type-classes di Haskell), istanze di
    proprietà, alias di tipi, firme di simboli; le definizioni di simboli hanno le seguenti grammatiche:
    \begin{lstlisting}
S :=
    let Var Args = E         --definizione

MS :=
    let Var MPM              --definizione con patterns

Args :=
    Var ... Var
    \end{lstlisting}
    \item espressioni date dalla seguente grammatica:
    \begin{lstlisting}
E :=
    Var                --variabile
    Datacon            --data-constructor
    Literal            --letterale: stringa, numero o carattere
    E E                --applicazione di espressione
    lam args -> E      --lambda-astrazione
    lam MPM            --lambda-astrazione con patterns
    S in E             --costrutto let..in
    MS in E            --costrutto let..in con patterns
    match E with PM    --pattern matching

PM :=
    ME -> E | ... | ME -> E

MPM :=
    | ME ... ME = E | ... | ME ... ME = E

ME :=                  --pattern
    Var                --variabile
    _                  --caso default
    Literal            --letterale: stringa, numero o costante
    Datacon ME ... ME  --patterns applicati a un data-constructor
    \end{lstlisting}
    \item un costrutto, valutato a compile-time (cfr. \hyperlink{Parsing degli operatori}{Parsing degli operatori}),
    per definire una ``categoria" di
    operatori. Le categorie di operatori hanno degli identificatori per poterle nominare, inoltre definiscono le
    seguenti proprietà:
        \begin{enumerate}
        \item una lista di operatori appartenenti alla proprietà. Sono compresi gli identificatori di variabili (gli
        operatori stessi sono identificatori di variabili) che possono essere utilizzati con la sintassi infissa degli
        operatori;
        \item una lista di categorie di operatori i quali avranno meno precedenza degli operatori della categoria
        corrente;
        \item una lista di categorie di operatori i quali avranno più precedenza degli operatori della categoria
        corrente;
        \item la fissità degli operatori.
        \end{enumerate}
    \end{enumerate}

\paragraph{Tipi di dati algebrici}
Fex supporta anche i tipi di dati algebrici (che sono l'unico modo per definire nuovi
tipi). Con essi, vi è anche la nozione di \textit{data constructor}, ovvero un costrutto dotato di tipo e al quale
possono essere associati dei dati. Per eseguire operazioni con i tipi di dati algebrici vi è il costrutto del pattern
matching che permette di destrutturare un data constructor dai suoi dati associati:
\begin{lstlisting}
type Arith =
      Plus Expr Expr
    | Minus Expr Expr
    | Times Expr Expr
    | Number Int

let eval expr =
    match expr with
          Plus e1 e2 -> eval e1 + eval e2
        | Minus e1 e2 -> eval e1 - eval e2
        | Times e1 e2 -> eval e1 * eval e2
        | Number n -> n
\end{lstlisting}

\paragraph{Polimorfismo parametrico}
Il sistema di tipi del linguaggio si basa fortemente su quello di System-F, il quale, come
è stato già menzionato in precedenza, introduce i quantificatori nei tipi. Quindi Fex fornisce il supporto per le variabili
di tipo, inoltre, permette di quantificarle (pur con alcune restrizioni). I quantificatori nei tipi permettono ai token
``tipabili" del linguaggio di possedere più tipi nello stesso momento, infatti, questo tipo di polimorfismo permette di avere
singoli algoritmi per una moltitudine di tipi, ad esempio:
\begin{lstlisting}
type List' a = Empty | Cons a (List' a)

let map
    | _ Empty -> Empty
    | f (Cons x t) -> Cons (f x) (map f t)
\end{lstlisting}

\paragraph{Tipi higher-kinded}
Il linguaggio permette la manipolazione di \textit{funzioni di tipi}, introducendo la nozione di
\textit{kind} (cfr. \hyperlink{Il sistema di tipi}{Il sistema di tipi}). Prima di mostrare degli esempi, è necessario fare
una piccola nota sulla notazione che verrà utilizzata nell'articolo: in letteratura, con la terminologia
\textit{variabile di tipo} si intende un ``placeholder" per un tipo che non necessita argomenti; in questo articolo, con tale
notazione si indicherà, invece, un ``placeholder" valido anche per tipi che necessitano argomenti.
Data la seguente definizione di tipo:
\begin{lstlisting}
type M a b = DataCon1 a | DataCon2 b | DataCon3 a Char
\end{lstlisting}
dove \texttt{a} e \texttt{b} sono variabili di tipo, è possibile, ad esempio, avere tipi della forma:
\begin{lstlisting}
M Int Char
M a b
M String
M x
M
\end{lstlisting}
\`E possibile altresì avere un tipo della forma:
\begin{lstlisting}
m Int
\end{lstlisting}
dove \texttt{m} è una variabile di tipo a cui viene applicata il tipo \textit{Int}.

\paragraph{Polimorfismo ad-hoc}
Attraverso il meccanismo delle type-classes (all'interno del linguaggio vengono chiamate
proprietà), è possibile creare funzioni polimorfe che possono essere applicate ad argomenti di tipi differenti e, a
seconda dei tipi degli argomenti, viene ``selezionata" un'implementazione. Ogni proprietà può avere uno, ma anche
più tipi associati. Si osservi il seguente esempio:
\begin{lstlisting}
type SuccessHttpCodes =
      Ok200
    | Moved301

type ErrorHttpCodes =
      Err401
    | Err404
    | Err500

property Show a =
    val show : a -> String
;;

instance Show SuccessHttpCodes =
    let show
        | Ok200 -> "Request success"
        | Moved301 -> "Resource moved permanently"
;;

instance Show ErrorHttpCodes =
    let show
        | Err401 -> "Unauthorized"
        | Err404 -> "Resource not found"
        | Err500 -> "Internal server error"
;;
\end{lstlisting}

\subsection{Il compilatore}
\begin{tikzpicture}[node distance=2cm]
\node (src) [object] {Codice sorgente};
\node (parser) [process, below of=src] {Parser};
\draw [arrow] (src) -- (parser);
\node (checker) [process, below of=parser] {Checker della correttezza};
\draw [arrow] (parser) -- (checker);
\node (typedBuilder) [process, below of=checker] {Costruzione dei token tipati};
\draw [arrow] (checker) -- (typedBuilder);
\node (prepare) [process, below of=typedBuilder] {Preparazione alla type inference};
\draw [arrow] (typedBuilder) -- (prepare);
\node (typeInf) [process, below of=prepare] {Type inference - type checking};
\draw [arrow] (prepare) -- (typeInf);
\node (CoreGen) [process, below of=typeInf] {Generazione codice Core};
\draw [arrow] (typeInf) -- (CoreGen);
\node (backend) [process, below of=CoreGen] {Back-end};
\draw [arrow] (CoreGen) -- (backend);
\node (exe) [process, below of=backend] {Eseguibile};
\draw [arrow] (backend) -- (exe);
\node (desugar) [process, right of=prepare, xshift=7cm] {Desugaring};
\draw [arrow] (desugar) -- (checker);
\draw [arrow] (desugar) -- (typedBuilder);
\draw [arrow] (desugar) -- (prepare);
\draw [arrow] (desugar) -- (typeInf);
\draw [arrow] (desugar) -- (CoreGen);
\end{tikzpicture}
\newline

La struttura del compilatore è sequenziale, con eccezione fatta per la fase di \textit{desugaring}. Quest'ultima si
occupa di rimuovere il cosiddetto ``zucchero sintattico" dalle strutture dati - che mantengono le informazioni del
programma - che, durante le varie fasi di compilazione, possono subire semplificazioni. Talvolta, alcune semplificazioni
vengono posticipate il più possibile per permettere al compilatore di restituire in output messaggi d'errore comprensibili
per l'utente (cfr. \hyperlink{Fasi di desugaring}{Fasi di desugaring}). Il modulo che si occupa del desugaring si presenta
come una libreria che espone delle API che agiscono
sulle strutture dati del compilatore. \`E compito del compilatore fare le chiamate ai vari sotto-moduli di desugaring.
Nel corso dell'articolo,
vi è una più ampia e precisa descrizione delle singole fasi di compilazione, con enfasi particolare sugli algoritmi
più interessanti utilizzati per risolvere i singoli sottoproblemi e su come le varie fasi interagiscono tra loro.

\section{L'albero di sintassi astratta}
Nella prima parte del compilatore viene eseguito il parsing del sorgente e, se non vi sono errori di sintassi, viene prodotto
l'albero di sintassi astratta (AST). Il codice che gestisce i ``token" dell'AST è
all'interno del modulo \texttt{Compiler.Ast.Tree}; l'entry-point dell'albero è dato dal token \texttt{Program}, il
quale, come nodi figli, ha delle \texttt{Declaration} che rappresentano i principali costrutti globali del linguaggio.

\begin{lstlisting}
newtype Program a = Program [Declaration a]

data Declaration a =
      ADT (AlgebraicDataType a)
    | AliasADT (AliasAlgebraicDataType a)
    | Intf (Interface a)
    | Ins (Instance a)
    | Sig (Signature a)
    | Let (SymbolDeclaration a)
    | LetMulti (MultiSymbolDeclaration a)
\end{lstlisting}

Di seguito vi è lo schema della prima parte del compilatore: \newline

\begin{tikzpicture}[node distance=2cm]
\node (src) [object] {Codice sorgente};
\node (rawast) [object, below of=src] {Albero di sintassi astratta};
\draw [arrow] (src) -- node[anchor=west] {Parser} (rawast);
\node (rawastBuiltin) [object, below of=rawast] {Albero di sintassi astratta};
\draw [arrow] (rawast) -- node[anchor=west] {Aggiunta token built-in} (rawastBuiltin);
\node (rawastNames) [object, below of=rawastBuiltin] {Albero di sintassi astratta};
\draw [arrow] (rawastBuiltin) -- node[anchor=west] {Check esistenza dei nomi} (rawastNames);
\node (rawastArgs) [object, below of=rawastNames] {Albero di sintassi astratta};
\draw [arrow] (rawastNames) -- node[anchor=west] {Check numero degli argomenti} (rawastArgs);
\node (rawastAlias) [object, below of=rawastArgs] {Albero di sintassi astratta};
\draw [arrow] (rawastArgs) -- node[anchor=west] {Sostituzione degli alias di tipo} (rawastAlias);
\node (desugar) [object, right of=rawastAlias, xshift=7cm] {Desugaring};
\draw [arrow] (desugar) -- (rawastAlias);
\end{tikzpicture}

\hypertarget{Il parser}{\subsection{Il parser}}
Il parser è la prima componente del compilatore (dopo la lettura del sorgente). \`E stato scritto mediante la libreria
open-source \texttt{Parsec} \hyperlink{bibl4}{[4]}, la quale si basa sul concetto di parser combinator monadico.
Il codice risiede nel
modulo \texttt{Compiler.Syntax} ed ha una struttura gerarchica: il codice di più ``basso livello" definisce i combinators
di ``pezzi" primitivi dei
token dell'AST (\texttt{Compiler.Syntax.Lib.SimpleParser}), dopodiché, nel modulo \texttt{Compiler.Syntax.Grammar} vi
è la generazione vera e propria dei token dell'AST e, infine, vi è l'entry-point del parser, ovvero
\texttt{Compiler.Syntax.Parser}. \`E bene notare che all'interno del modulo \texttt{Compiler.Syntax.Lib.SimpleParser}
non sono visibili le API dell'AST, in quanto esso si occupa soltanto del parsing dei costrutti del linguaggio e non
della generazione dei token.

\hypertarget{Parsing degli operatori}{\subsubsection{Parsing degli operatori}}
Fex permette di definire operatori che vengono trattati come simboli di variabili. In questo contesto,
il linguaggio espone all'utente un costrutto particolare che permettere di definire le proprietà degli eventuali
operatori. Questo costrutto viene valutato durante il parsing e definisce una categoria di operatori, ecco un esempio:

\begin{lstlisting}
OPERATORS_CATEGORY {#
    name : Application ;
    operators : |>, <|, `applyTo ;
    lesser than : Comparison, Numeric ;
    greater than : Functor ;
    fixity : InfixLeft ;
#}
\end{lstlisting}

Il primo ``campo" è il nome della categoria ed è utile per poter identificare la categoria, infatti, il terzo e il
quarto campo definiscono rispetto a quali categorie gli attuali operatori hanno, rispettivamente, meno o più precedenza.
Il secondo campo definisce
l'insieme di operatori che fanno parte della categoria. Infine, l'ultimo campo è una costante e definisce la fissità
degli operatori: infissa senza associazione, infissa con associazione a sinistra, infissa con associazione a destra,
postfissa o prefissa. La versione infissa è solo per gli operatori binari, mentre postfissa e prefissa solo per gli
operatori unari. Si può notare che nella lista di operatori vi è anche un simbolo di variabile preceduto dal carattere
di backtick; questo è possibile, in quanto Fex permette di utilizzare i simboli di variabili come operatori
facendoli precedere dal carattere backtick. La valutazione di questo costrutto è interamente integrata nel parser ed
è obbligatorio per l'utente definire le categorie degli operatori all'inizio del sorgente. La libreria \texttt{Parsec}
offre delle API anche per il parsing degli operatori; l'algoritmo di gestione delle categorie si occupa di costruire
una tabella di operatori, implementata semplicemente come una lista di liste, ordinando i gruppi (di categorie) in
base alla loro precedenza. Le informazioni sulle categorie vengono estrapolate dal parsing del costrutto e vengono
passate successivamente all'algoritmo di ordinamento delle categorie. Quest'ultimo si può ridurre all'inserimento
di un elemento in una lista di liste:

\begin{lstlisting}
insert(x, ll):
    match ll with
        [] -> [[x]]
        (l :: lt) ->
            if any x' in l. x' < x        // Condizione d'inserimento (1)
            then
                if areAmbigous(x, ll)
                then fail
                else [x] :: l :: lt
            else if all x' in l. x' == x  // Condizione d'inserimento (2)
            then
                if areAmbigous(x,ll)
                then fail
                else subInsert(x, l) :: lt
            else l :: insert(x, lt)
\end{lstlisting}

L'algoritmo scorre la lista finché non:
\begin{enumerate}
    \item trova una sottolista $ l $ in cui esiste almeno un elemento minore di $ x $; in questo caso, controlla
    eventuali ambiguità delle categorie, poiché l'utente potrebbe aver definito categorie tali che la nozione di
    ordinamento tra loro non è transitiva. La funzione \texttt{areAmbigous} nel pezzo di pseudo-codice si occupa
    di eseguire questo controllo. Se non esistono ambiguità tra le categorie, allora viene creata una nuova lista
    singoletto contenente $ x $ che viene inserita davanti alla sottolista $ l $
    \item trova una sottolista $ l $ in cui tutti gli elementi sono uguali a $ x $; in questo caso, $ x $
    viene inserito in $ l $, eseguendo sempre prima il controllo sulle ambiguità.
\end{enumerate}
In tutti gli altri casi, la corrente sottolista $ l $ contiene almeno un elemento $ x' $ tale che
$ x' > x $, quindi l'inserimento non può ancora avvenire. Quando la tabella è completa, essa viene passata alla
funzione \texttt{buildExpressionParser} della libreria \texttt{Parsec} che si occupa di costruire il parser per le
espressioni.

\hypertarget{Check dei nomi}{\subsection{Check dei nomi}}
Dopo la fase di parsing e dell'aggiunta di token built-in del linguaggio,
viene effettuato il controllo di esistenza di ogni tipo di simbolo: nomi di tipo, nomi di variabili, nomi di proprietà,
etc.. Perciò deve valere la seguente condizione:
    \[ \forall name \in AST. \; \exists def(name) \in AST \]
dove \textit{def} è la funzione che ritorna la definizione di un token dell'AST.
Questo tipo di check è molto importante in quanto fasi successive del compilatore basano le loro computazioni
sull'ipotesi che tutti i simboli siano stati definiti. Il codice risiede nel modulo \texttt{Compiler.Args}.

\hypertarget{Check degli argomenti}{\subsection{Check degli argomenti}}
Il controllo degli argomenti viene effettuato, a differenza del check dei nomi che viene eseguito su ogni tipo di nome,
solo sui nomi di tipo, di alias e di proprietà. In particolare, devono valere le seguenti condizioni:
\[ \forall name \in AST. \]
\begin{enumerate}
    \item:
        \[ isTypeName(name) \Longrightarrow \#args(name) \leq \#args(def(name)) \]
    \item:
        \[ isAliasName(name) \Longrightarrow \#args(name) = \#args(def(name)) \]
    \item:
        \[ isPropertyName(name) \Longrightarrow \#args(name) = \#args(def(name)) \]
\end{enumerate}
dove \textit{isTypeName}, \textit{isAliasName}, \textit{isPropertyName} sono le funzioni che ritornano \textit{true}
se il nome in input è, rispettivamente, un nome di tipo, un nome di alias, un nome di proprietà, \textit{false} altrimenti,
\textit{args} è la funzione che calcola gli argomenti di un token e il simbolo \textit{\#} è la funzione che calcola
la cardinalità di un insieme. In tutti e tre i casi, gli argomenti di un simbolo sono una sequenza di variabili di tipo,
ad esempio:
\begin{lstlisting}
type Map k v = Entry (Tuple2 k v) | Node (Tuple2 k v) (Map k v) (Map k v)

alias IntMap v = Map Int v

property Existence e a =
    val exists : e -> a -> Bool
;;
\end{lstlisting}
Gli argomenti di \texttt{Map} sono \texttt{k} e \texttt{v}, quelli di \texttt{IntMap} sono \texttt{v}, mentre quelli di
\texttt{Existence} sono \texttt{e} e \texttt{a}.
La condizione (1) è meno stringente di (2) e di (3),
in quanto l'utente può ``manipolare" non solo tipi, ma anche funzioni di tipi,
anche conosciute come \textit{type constructor}. La condizione (2) è fondamentale per la prossima fase del compilatore.
Il codice risiede nel modulo \texttt{Compiler.Args}.

\hypertarget{Prime fasi di desugaring}{\subsection{Prime fasi di desugaring}}
Prima della generazione dei token ``tipati", il compilatore si occupa di effettuare alcune fasi di desugaring. Esse si
occupano dell'eliminazione o dell'aggiornamento di alcuni costrutti presenti in un programma per semplificare le fasi
successive del compilatore.

\hypertarget{Eliminazione degli alias di tipo}{\subsubsection{Eliminazione degli alias di tipo}}
Fex offre un costrutto che permette all'utente di definire alias di tipo. Ecco un esempio:
\begin{lstlisting}
alias CharAnd x = Tuple2 Char x
\end{lstlisting}
Questo tipo di costrutto viene completamente valutato in fase di compilazione: ogni occorrenza di nome di alias viene
trattata come una vera e propria \textit{macro}, quindi viene sostituita con il tipo associato all'alias. Il modulo di
Desugaring si occupa di questo task (\texttt{Compiler.Desugar.Alias}).
Come si nota nell'esempio, gli alias ammettono argomenti (variabili di tipo) che vengono passati alla funzione di tipo.
Il linguaggio ammette anche alias di alias, tuttavia, ciò può portare a \textit{cicli}, come ad esempio:

\begin{lstlisting}
alias A = B
alias B = C
alias C = A
\end{lstlisting}

Per questo motivo, l'algoritmo di sostituzione degli alias implementa anche la ``cycle detection": se esiste un ciclo di
alias, il programma viene rifiutata. Inoltre, nella
sostituzione è fondamentale che valga la condizione sugli alias nel check degli argomenti
(cfr. \hyperlink{Check degli argomenti}{Check degli argomenti}):
    \[ \forall name \in AST. \; isAliasName(name) \Longrightarrow \#args(name) = \#args(def(name)) \]
Se la sopracitata condizione non fosse vera, non sarebbe possibile effettuare l'unificazione nella kind-inference (cfr.
kind-inference).

\hypertarget{Eliminazione delle firme di funzioni}{\subsubsection{Eliminazione delle firme di funzioni}}
Fex espone un costrutto, detto \textit{signature} (in italiano ``firma"), che permette di indicare il tipo di un
binding. Ad esempio:

\begin{lstlisting}
val id : a -> a
\end{lstlisting}

L'esempio appena mostrato indica che la variabile \textit{id} ha tipo $ \forall \alpha. \alpha \mapsto \alpha $. Questo
tipo di costrutto non è nient'altro che ``zucchero sintattico" per il type-hinting dei binding. I token dei binding
- \texttt{SymbolDeclaration} e \texttt{MultiSymbolDeclaration} - possono possedere delle informazioni sul proprio tipo,
ad esempio, guardando la definizione di \texttt{MultiSymbolDeclaration}:

\begin{lstlisting}
data MultiSymbolDeclaration a =
    MultiSymTok (SymbolName a) (Hint a) (MultiPatternMatch a) a
\end{lstlisting}

si può notare come il costruttore \texttt{MultiSymTok} prenda in input un token \texttt{Hint}. Il task del modulo
\texttt{Compiler.Desugar.Sigs} è di eliminare dall'AST i costrutti \texttt{Signature} che rappresentano, appunto, le
firme delle variabili e aggiungere l'informazione sul tipo contenuta in esse come type-hinting delle definizioni dei
bindings.

\hypertarget{Renaming delle variabili di tipo}{\subsubsection{Renaming delle variabili di tipo}}
Prima di iniziare le fasi di costruzione dei token ``tipati", dato un programma $ P $, esso viene visitato e le occorrenze
delle variabili di tipo vengono tutte rinominate in modo che:
\begin{itemize}
    \item per ogni token $ t, t' \in P $, tali che $ t $ non sia lo stesso token di $ t' $, vale:
    \[ \forall v \in binders(t). \; \neg \exists v' \in binders(t'). \; v = v' \]
\end{itemize}
dove $ binders $ è la funzione che ritorna le variabili di tipo che fungono da binders nel token in questione, mentre il
test di uguaglianza
tra variabili di tipo è definito come il test di uguaglianza dei letterali che rappresentano gli identificatori delle
variabili.
Chiaramente, le variabili legate vengono cambiate in base a come vengono aggiornati i binders, ad esempio, si consideri
la seguente definizione:
\begin{lstlisting}
type T a b = C1 Int a | C2 a (T a b)
\end{lstlisting}
Se viene effettuata la seguente sostituzione:
\[ \{ a \mapsto a_1, b \mapsto a_2 \} \]
Allora la nuova definizione di T avrà la seguente forma:
\begin{lstlisting}
type T a1 a2 = C1 Int a1 | C2 a1 (T a1 a2)
\end{lstlisting}
Il motivo di questo renaming verrà descritto in seguito (cfr. \hyperlink{Kind-inference}{Kind-inference}). Il codice
risiede nel modulo \texttt{Compiler.Desugar.TyVars}.

\section{Generazione dei token ``tipati"}
Dopo la generazione dell'albero di sintassi astratta e alcune fasi di desugaring, questa seconda componente del
compilatore si occupa della generazione dei token ``tipati". Questi ultimi prendono questa nomea in quanto, a questo
livello, compare la nozione di tipo del linguaggio.
\newline

\begin{tikzpicture}[node distance=2cm]
\node (desast) [object] {AST ``dezuccherato"};
\node (desastContsCheck) [object, below of=desast] {AST ``dezuccherato"};
\draw [arrow] (desast) -- node[anchor=east] {Check dei constraints} (desastContsCheck);
\node (typesTable) [object, below of=desastContsCheck] {Tabella dei tipi};
\draw [arrow] (desastContsCheck) -- node[anchor=east] {Kind-inference} (typesTable);
\node (consTable) [object, below of=typesTable] {Tabella dei costrutturi};
\draw [arrow] (typesTable) -- node[anchor=east] {Costruzione dei data constructor} (consTable);
\node (contsTable) [object, below of=consTable] {Tabella dei constraints o predicati};
\draw [arrow] (consTable) -- node[anchor=east] {Costruzione dei constraints} (contsTable);
\node (instances) [object, below of=contsTable] {Tabelle di: metodi di proprietà; metodi di istanza; istanze};
\draw [arrow] (contsTable) -- node[anchor=east] {Valutazione delle istanze di proprietà} (instances);
\node (bindings) [object, below of=instances] {Bindings dell'AST pronti per la type-inference};
\draw [arrow] (instances) -- node[anchor=east] {``Preparazione" alla type-inference} (bindings);
\node (tyBindings) [object, below of=bindings] {Bindings tipati};
\draw [arrow] (bindings) -- node[anchor=east] {type-inference} (tyBindings);
\node (desugar) [object, right of=bindings, xshift=6cm] {Desugaring};
\draw [arrow] (desugar) -- (instances);
\draw [arrow] (desugar) -- (bindings);
\draw [arrow] (desugar) -- (tyBindings);
\end{tikzpicture}

\hypertarget{Approccio a tabelle}{\subsection{Approccio a tabelle}}
A differenza della prima componente del compilatore, dove l'unica struttura dati di primo livello era l'AST, in
questo caso vi sono molteplici strutture dati. Innanzitutto, nel modulo \texttt{Compiler.Ast.Typed},
vi sono le definizioni di tutti i token tipati e le operazioni su di essi; proprio in questo modulo compaiono:
\begin{enumerate}
    \item le nozioni che riguardano i tipi del linguaggio:
    \begin{lstlisting}
data LangKind               --kind
data LangVarType a          --variabile di tipo
data LangHigherType a       --mono-tipo
data LangSpecConstraint a   --``constraint" o predicato
data LangQualType a         --mono-tipo qualificato
data LangTypeScheme a       --poli-tipo o schema di tipo
    \end{lstlisting}
    La nozione di \textit{kind} serve per aggiungere un'informazione ai tipi; in Fex, non vi è alcun costrutto per
    esprimere delle espressioni che hanno un kind come informazione aggiuntiva. Un \textit{mono-tipo} è, come vedremo, un
    tipo in cui possono comparire variabili di tipo, le quali sono tutte quantificati esistenzialmente, ad esempio:
\[ \exists \alpha, \beta. \; \alpha \rightarrow \beta \]
    La nozione di \textit{predicato} rappresenta, invece, un'affermazione su un tipo e ciò è utile per restringere
    l'insieme di tipi adatto ad istanziare un determinato mono-tipo; come vedremo in seguito, i predicati vengono espressi
    attraverso il meccanismo delle \textit{proprietà}, ad esempio, nel seguente pezzo di codice, il token \texttt{Ord} serve
    per esprimere predicati:
    \begin{lstlisting}
property Ord a =
    <metodi di proprieta'>
;;
    \end{lstlisting}
    Un \textit{mono-tipo qualificato} è un tipo sul quale vi sono uno o più predicati. Uno \textit{schema di tipo} è un
    tipo che ammette variabili di tipo quantificate universalmente:
\[ \forall \alpha, \beta. \; \alpha \rightarrow \beta \]
    Si faccia riferimento alla sezione sul sistema di tipi per le definizioni formali (cfr.
    \hyperlink{Il sistema di tipi}{Il sistema di tipi}).
    \item i token tipati che costituiscono un programma:
    \begin{lstlisting}
data NotedVar a           --variabili
data NotedVal a           --valori: letterali e data-constructor
data NotedMatchExpr a     --espressioni per il pattern match
data NotedExpr a          --expressioni
    \end{lstlisting}
    \`E bene fare una precisazione su \texttt{NotedMatchExpr}. Considerando il sotto-costrutto di un \textit{caso} del
    costrutto di pattern matching, sia esso della forma:
\[ pattern \rightarrow expr \]
    il token \texttt{NotedMatchExpr} rappresenta un \textit{pattern}.
    \item operazioni che riguardano la manipolazione dei tipi quali unificazione, test di specificità, specializzazione,
    instanziazione, generalizzazione. Inoltre, vi sono le funzioni e le strutture dati per gestire il dispatch statico.
\end{enumerate}
Nonostante la presenza di token tipati, non esiste una corrispondente versione tipata dell'AST con un unico entry-point,
bensì le informazioni che riguardano un programma vengono memorizzate in tabelle (cfr. modulo
\texttt{Compiler.Types.Tables}):
\begin{lstlisting}
newtype TypesTable a          --tabella dei type-constructor
newtype DataConsTable a       --tabella dei data-constructor
newtype ConstraintsTable a    --tabella dei constraint-constructor
newtype InstsTable a          --tabella dei bindings delle istanze
newtype PropMethodsTable a    --tabella dei metodi di proprieta'
newtype ImplTable a           --tabella delle istanze
data    TypedProgram a        --tabella dei bindings tipati
\end{lstlisting}
Vedremo nel dettaglio ogni tabella nelle descrizioni delle varie fasi del compilatore. Tuttavia, è necessaria una nota
su \texttt{TypesTable} e \texttt{ConstraintsTable}. Come si legge dai commenti, esse sono tabelle per memorizzare dei
costruttori. Tali costruttori sono necessari alla costruzione dei tipi e dei predicati all'interno di un programma. Ad esempio,
date le seguenti definizioni in Fex:
\begin{lstlisting}
type Box a = Boxing a
property Stateful m =
    val getState : m a -> a
;;
\end{lstlisting}
Verranno creati dei token (presenti in \texttt{Compiler.Ast.Typed}):
\begin{lstlisting}
data LangNewType a            --type-constructor
data LangNewConstraint a      --constraint-constructor
\end{lstlisting}
che rappresenteranno rispettivamente il modello per tipi \texttt{Box} e il modello per constraints \texttt{Stateful} e che
verranno memorizzati nelle suddette tabelle. Mentre, quanto riguarda \texttt{Boxing}, esso rappresenta un data-construcotr.

Sempre nel modulo \texttt{Compiler.Types.Tables}, viene definito anche il cosiddetto ``binding tipato":
\begin{lstlisting}
type BindingSingleton a = (NotedVar a, [NotedVar a], NotedExpr a)
data TypedBinding a =
      TyNonRec (BindingSingleton a)
    | TyRec [BindingSingleton a]
\end{lstlisting}
Osservando l'implementazione di \texttt{TypedBinding}, si nota come esistano due tipi di binding. Il primo è per i binding
non ricorsivi, mentre il secondo è per i binding che sono mutualmente ricorsivi fra loro
(cfr. \hyperlink{Type-inference}{Type-inference}). In seguito, vedremo perché si rende necessaria questa distinzione.

\hypertarget{Confronto con GHC}{\subsubsection{Confronto con GHC}}
Come è stato detto precedentemente, l'approccio del compilatore è quello di costruire tabelle man mano che le informazioni
vengono inferite dall'AST. GHC (the Glasgow Haskell Compiler) utilizza un approccio differente, in quanto non utilizza
alcuna ``symbol table", bensì ogni token tipato (di GHC) nella compilazione di un programma Haskell può
puntare ad altri token tipati \hyperlink{bibl5}{[5]}. Si crea così un grafo di strutture dati tipate.
Ad esempio, GHC, per gestire
le entità di type constructor e data constructor, utilizza rispettivamente i token \texttt{TyCon} e \texttt{DataCon} (si
ricordi che GHC è scritto in Haskell):
\begin{lstlisting}
data TyCon
data DataCon
\end{lstlisting}
Ogni token di tipo \texttt{TyCon} punterà a una lista di \texttt{DataCon} che, a loro volta, conterranno la referenza
al loro costruttore di tipo. Come puntualizza Edward Y. Zang nell'introduzione dell'articolo \hyperlink{bibl6}{[6]},
uno svantaggio di questo approccio è che il grafo è immutabile e quindi, per poter
aggiornare i nodi del grafo è necessario ricostruire il grafo da zero. Tuttavia, questo problema è mitigato, in quanto
gli aggiornamenti del grafo sono parecchio rari, inoltre, man mano che GHC ottiene informazioni dal programma Haskell,
accrescerà il grafo senza aggiornare i nodi preesistenti; in questo modo, non vi è alcuna necessità di costruire il
grafo da zero. La scelta, nel compilatore Fex, è stata quella di un approccio a tabelle proprio per evitare casi di mutua
ricorsione tra tipi e, quindi, di codice potenzialmente più complesso per l'aggiornamento dei dati che riguardano i token
del programma. Infatti, sebbene alcune tabelle, una volta costruite, vengano utilizzate in modalità di sola lettura, altre
tabelle, come \texttt{TypedProgram}, vengono aggiornate più volte durante le varie fasi del compilatore.

\hypertarget{Il sistema di tipi}{\subsection{Il sistema di tipi}}
In questa sezione, verrà presentato il sistema di tipi di Fex. Come è stato già menzionato in precedenza, il
linguaggio supporta il polimorfismo ad hoc e il polimorfismo parametrico; quest'ultimo viene implementato attraverso
il concetto di variabile di tipo, la quale può essere considerata come un ``placeholder" per i tipi. Il sistema di tipi
che verrà presentato è fortemente basato su quello di System-F, tuttavia, vi sono alcune estensioni che permettono una
maggiore espressività.

\hypertarget{Mono-tipo}{\subsubsection{Mono-tipo}}
Di seguito, viene definito il concetto di \textit{mono-tipo}:
\[ MT \; := \; \alpha \; | \; T \; | \; MT \; MT \]
dove $ \alpha $ è una variabile di tipo e $ T $ è un costruttore di tipo. \`E fondamentale che, dato un programma $ P $,
l'insieme dei mono-tipi in $ P $ abbia almeno il tipo funzione $ (\rightarrow) $.
Come si può dedurre dalla definizione stessa, non si deve confondere
il concetto di mono-tipo con quello di tipo \textit{monomorfo}, infatti i mono-tipi, a differenza dei tipi monomorfi,
ammettono le variabili di tipo. L'ultimo caso è il tipo \textit{funzione}; nella sezione sull'inferenza di tipo verrà
mostrato il motivo per il quale è fondamentale garantire la presenza del tipo funzione all'interno del type-system.
Nel compilatore, la definizione di mono-tipo si trova nel modulo \texttt{Compiler.Ast.Typed}:
\begin{lstlisting}
data LangHigherType a =
      LTy (LangType a)
    | HApp [LangHigherType a] a
\end{lstlisting}
dove \texttt{LangType} rappresenta tutti i tipi che possono essere definiti in Fex. Si noti come vi è un secondo
caso (\texttt{HApp}), il quale rappresenta il tipo funzione. Il costruttore \texttt{HApp} prende in input una lista
di mono-tipi, tuttavia il tipo funzione può avere al massimo due argomenti, perciò il modulo \texttt{Compiler.Ast.Typed}
si occupa internamente di rifiutare qualsiasi valore con costruttore \texttt{HApp} che ha più di due argomenti.
La causa per la quale non vi è un numero fissato di argomenti è la presenza delle funzioni di tipo, infatti,
in questo modo, è possibile esprimere senza difficoltà, all'interno del compilatore, tipi come:
\begin{itemize}
\item \texttt{(->) a}
\item \texttt{(->)}
\end{itemize}

\hypertarget{Tipi higher-kinded}{\subsubsection{Tipi higher-kinded}}
Fex permette all'utente di utilizzare \textit{funzioni di tipo} (o \textit{costruttori di tipo}). A questo scopo,
viene introdotta la nozione di \textit{kind}, il quale rappresenta un'informazione aggiuntiva per i tipi. Informalmente,
dato un type-system \textit{TS}, un kind-system \textit{KS} si può vedere come un type-system di livello ``superiore"
a \textit{TS}. Ora diamo una definizione più precisa di kind:
\[ K \; := \; \kappa \; | \; * \; | \; K \rightarrow K \]
dove $ \kappa $ è una variabile di kind e $ * $ è la costante primitiva di kind (detta anche ``tipo"), il quale
rappresenta tutti quei tipi che non hanno bisogno di parametri. Ecco alcuni esempi:
\begin{itemize}
    \item $ * \rightarrow * $ è il kind delle funzioni di tipo unarie;
    \item $ (* \rightarrow *) \rightarrow * \rightarrow * $ è il kind delle funzioni di tipo binarie che prendono in input
    una funzione di tipo unaria e un tipo e ritornano un altro tipo.
    \item $ \kappa_1 \rightarrow \kappa_2 $ è il kind delle funzioni unarie che prendono in input una funzione di tipo
    di kind $ \kappa_1 $ e ritorna un'altra funzione di tipo di kind $ \kappa_2 $.
\end{itemize}
Data la presenza di tipi \textit{higher-kinded} all'interno di Fex, è doveroso fare un'ulteriore precisazione che
riguarda i mono-tipi: alla definizione di mono-tipo deve essere aggiunta la condizione sulla correttezza dei kind.
Ad esempio, dato il tipo:
\[ t_1 \rightarrow t_2 \]
sapendo che il kind del costruttore di tipo funzione è $ * \rightarrow * \rightarrow * $, è necessario, affinché il suddetto
kind venga rispettato, che il kind di $ t_1 $ e $ t_2 $ sia $ * $.
Nel codice sorgente, la definizione di kind nel compilatore Fex si trova in \texttt{Compiler.Ast.Typed}:
\begin{lstlisting}
data LangKind =
      LKVar String
    | LKConst
    | SubLK [LangKind]
\end{lstlisting}
Il costruttore \texttt{LKVar} rappresenta le variabili di kind, il costruttore \texttt{LKConst} rappresenta il kind
\textit{tipo} $ * $, mentre il costruttore \texttt{SubLK} rappresenta l'applicazione di kinds.

\hypertarget{Subtyping: predicati e tipi qualificati}{\subsubsection{Subtyping: predicati e tipi qualificati}}
La nozione di mono-tipo ammette le variabili di tipo, le quali fungono da placeholder per qualsiasi tipo dello stesso
kind della variabile. Tuttavia, solamente con la nozione di mono-tipo non è possibile fare asserzioni sulle variabili
di tipi, se non, appunto, che sono placeholder adatti a qualsiasi tipo. Fex supporta anche una nozione di sottotipo.
Prendiamo, ad esempio, la seguente variabile di tipo:
\[ \alpha : * \]
la notazione $ : * $ serve, in questo caso, per rendere esplicito il kind della variabile. $ \alpha $ è un placeholder
adatto a tipi quali, ad esempio, \texttt{Int}, \texttt{List b} oppure \texttt{List Char} poiché hanno tutti kind $ * $.
$ \alpha $ è un placeholder molto lasco, in quanto è adatto per tutti quei tipi di kind $ * $, quindi si potrebbero fare
asserzioni su $ \alpha $, ad esempio, si può asserire che $ \alpha $ è un placeholder
valido per tutti quei tipi $ S $ che sono sottotipi di un certo tipo $ T $. Una definizione informale di sottotipo è la
seguente: se $ S $ è sottotipo di $ T $ (scriviamo $ S \leq T $), allora ogni termine di $ S $ può essere utilizzato
in maniera \textit{safe} in ogni \textit{contesto} in cui un termine di $ T $ è richiesto, dove le definizioni di
\textit{safe} e \textit{contesto} sono dipendenti dal formalismo.
Aggiungiamo una notazione per descrivere questa nozione:
\[ \alpha \leq T . \; T' \]
Possiamo quindi restringere i possibili tipi adatti a ``riempire" il placeholder rappresentato da $ \alpha $. Più in
generale, possiamo considerare $ \alpha \leq T $ come un predicato $ P $ su $ \alpha $. Utilizzeremo, quindi, la seguente
notazione:
\[ P(\alpha) \Rightarrow T' \]
Ora abbiamo una sintassi per descrivere \textit{predicati} o \textit{constraint} sulle variabili di tipo. Tale
comportamento si può estendere a qualsiasi forma di tipo, non solo alle variabili di tipo, ma per farlo è necessario
prima definire la sintassi dei predicati. In Fex, un predicato o constraint ha la seguente forma:
\[ C := P \; MT_1 \; ... \; MT_n \]
dove $ P $ è il nome di una proprietà (cfr. \hyperlink{Polimorfismo ad-hoc}{Polimorfismo ad-hoc}). Ora possiamo quindi
definire un'estensione dei
mono-tipi, in modo che su questi ultimi si possano aggiungere delle ipotesi. Per una definizione più generale possibile,
ammettiamo che su un mono-tipo si possa applicare un numero arbitrario di predicati:
\[ QT := MT \; | \; C \Rightarrow QT \]
Questa appena data è la nozione di \textit{tipo qualificato}. La definizione si trova nel modulo
\texttt{Compiler.Ast.Typed}:
\begin{lstlisting}
data LangQualType a = Qual [LangSpecConstraint a] (LangHigherType a)
\end{lstlisting}

\hypertarget{Schemi di tipo e polimorfismo parametrico}{\subsubsection{Schemi di tipo e polimorfismo parametrico}}
Finora abbiamo trattato mono-tipi e tipi qualificati, tuttavia, la sola presenza delle variabili di tipo non è sufficiente
a costruire \textit{schemi di tipi}. Si consideri il seguente esempio:
\[ map : (\alpha \mapsto \beta) \mapsto List \; \alpha \mapsto List \; \beta \]
In questo caso, il tipo della funzione $ map $ non è uno \textit{schema di tipo}, in quanto le variabili di tipo
$ \alpha $ e $ \beta $
rappresentano un tipo fissato non ancora conosciuto. Per ottenere il polimorfismo parametrico, è necessario introdurre
dei quantificatori, ad esempio, possiamo rifinire il tipo di $ map $ nel modo seguente:
\[ map : \forall \alpha, \beta . \; (\alpha \mapsto \beta) \mapsto List \; \alpha \mapsto List \; \beta \]
A differenza di prima, in cui $ map $ aveva un tipo prefissato, qui può avere molteplici tipi. Ora diamo la definizione
di \textit{schema di tipo} o \textit{poli-tipo} in Fex:
\[ PT := QT \; | \; \forall \alpha. \; PT \]
Questa definizione lascia spazio alla presenza di \textit{variabili libere}, in quanto non necessariamente tutte le
variabili di tipo che compaiono in un tipo qualificato sono legate a un quantificatore. Le variabili libere vengono
trattate come costanti. Si noti come i quantificatori possono apparire solamente alla testa di un tipo, ad esempio, il
seguente tipo non è accettato in Fex:
\[ \forall \alpha . \; (\forall \alpha . \; \alpha \mapsto \alpha) \mapsto \alpha \]
La definizione di schema di tipo in Fex si trova nel modulo \texttt{Compiler.Ast.Typed}:
\begin{lstlisting}
data LangTypeScheme a = Forall [LangVarType a] (LangQualType a)
\end{lstlisting}
La politica sulla sintassi degli schemi di tipi di Fex è omettere il quantificatore $ \forall $ nelle indicazione sul tipo.
Per questo, ogni qual
volta vi è un costrutto di type-hinting (firme comprese), il tipo viene considerato come uno schema di tipo in cui tutte
le variabili di tipo che compaiono vengono legate a un quantificatore, ad esempio:
\begin{lstlisting}
val map : (a -> b) -> List a -> List b
\end{lstlisting}
in questo caso la funzione \texttt{map} avrà il seguente tipo:
\[ map : \forall \alpha, \beta . \; (\alpha \mapsto \beta) \mapsto List \; \alpha \mapsto List \; \beta \]

\hypertarget{Test di uguaglianza tra tipi}{\subsubsection{Test di uguaglianza tra tipi}}
Il test di uguaglianza tra tipi si differenzia parecchio tra le diverse nozioni di ``tipo", inoltre, non sempre risulterà
utile avere una definizione di uguaglianza tra alcune nozioni di tipo. Partendo dalla nozione di mono-tipo, due
mono-tipi $ \tau $ e $ \tau' $ sono uguali se hanno i termini identici. Dato che Fex supporta la nozione di mono-tipo
higher-kinded, è banale asserire che:
\[ kindOf(\tau) \neq kindOf(\tau') \Longrightarrow \tau \neq \tau' \]
Per quanto riguarda i tipi qualificati, è necessario prendere in considerazione anche i predicati che precedono un
mono-tipo. Definiamo, quindi, il test di uguaglianza tra predicati. Siano $ (C \; \tau_1 \; ... \; \tau_n) $ e
$ (C' \; \tau_1' \; ... \; \tau_m') $ due constraints, essi sono uguali se:
\[ C = C' \wedge n = m \wedge \forall i \in \{1, ..., n\}. \; \tau_i = \tau_i' \]
Ora possiamo definire il test uguaglianza per i tipi qualificati. Siano $ (P_1, ..., P_n \Rightarrow \tau) $ e
$ (Q_1, ..., Q_m \Rightarrow \tau') $ due tipi qualificati, essi sono uguali se:
\begin{enumerate}
    \item $ \tau = \tau' $
    \item $ \forall i \in \{1, ..., n\}. \; \exists j \in \{1, ..., m\}. \; P_i = Q_j $
    \item $ \forall j \in \{1, ..., m\}. \; \exists i \in \{1, ..., n\}. \; Q_j = P_i $
\end{enumerate}
Ora rimangono gli schemi di tipo. A differenza dei mono-tipi, qui le variabili possono essere quantificate, quindi non
basterebbe più definire il test di uguaglianza analizzando solamente i termini del mono-tipo. Nel codice sorgente in
\texttt{Compiler.Ast.Typed}, non vi è alcuna definizione di uguaglianza tra schemi di tipo poiché questa non è necessaria
a nessun'altra operazione.

\hypertarget{Polimorfismo ad-hoc}{\subsection{Polimorfismo ad-hoc}}
Fex supporta il polimorfismo ad-hoc esponendo costrutti chiamati \textit{proprietà}, le quali sono concettualmente
molto simili alle \textit{type classes} di Haskell. Ecco un esempio di definizione di proprietà in Fex:
\begin{lstlisting}
property Eq a =
    val (==) : a -> a -> Bool
    val (/=) : a -> a -> Bool
;;
\end{lstlisting}
Questo pezzo di codice produce idealmente due token corrispondenti ai metodi \texttt{(==)} e \texttt{(/=)},
i quali possono essere istanziati attraverso il meccanismo delle istanze che verrà presentato prossimamente. I token
appena costruiti avranno le seguenti firme:
\begin{lstlisting}
val (==) : Eq a => a -> a -> Bool
val (/=) : Eq a => a -> a -> Bool
\end{lstlisting}
Si noti come le firme effettive non siano le stesse di quelle date dall'utente, le quali risultano incomplete.
All'inizio della dicitura dei tipi effettivi, si può notare un \textit{constraint} o \textit{predicato} seguito da una
freccia, la quale è intesa come implicazione. I tipi dei metodi di proprietà avranno quindi la forma:
    \[ Pred(\overline{\alpha}) \Rightarrow ty(\overline{\alpha}) \]
dove \textit{Pred} è un predicato e \textit{ty} è un tipo qualunque sulle variabili $ \overline{\alpha} $.
Il polimorfismo ad-hoc viene, dunque, supportato attraverso i predicati, i quali rappresentano delle ipotesi aggiuntive
sui tipi. L'istanziazione di una proprietà consiste nel dichiarare dei tipi che fungeranno da ``caso specifico" per i
metodi di proprietà (i ``modelli"), ad esempio, riprendendo la proprietà \texttt{Eq} del precedente pezzo di codice,
in Fex possiamo definire un'istanza del tipo:
\begin{lstlisting}
instance Eq Char =
    <implementazioni>
;;
\end{lstlisting}
I dettagli implementativi sono omessi. In ogni caso, i metodi di istanza avranno le seguenti firme:
\begin{lstlisting}
val (==) : Char -> Char -> Bool
val (/=) : Char -> Char -> Bool
\end{lstlisting}
Si può notare come il predicato su \texttt{Eq} non esista più e ciò è dovuto all'istanziazione, ovvero è stato fissato
un tipo per il quale vale il predicato \texttt{Eq}.

\hypertarget{Estensioni del polimorfismo ad-hoc}{\subsection{Estensioni del polimorfismo ad-hoc}}
Il meccanismo di polimorfismo ad-hoc precedentemente presentato incontra numerose estensioni in Fex, alcune delle
quali impongono condizioni aggiuntive sul programma.

\hypertarget{Numero arbitrario di argomenti delle proprieta}{\subsubsection{Numero arbitrario di argomenti delle proprietà}}
Nell'esempio con \texttt{Eq}, la proprietà possedeva soltanto un argomento, tuttavia, è possibile definire proprietà
con un numero arbitrario (ma fissato alla definizione) di argomenti ad esempio:
\begin{lstlisting}
property Stream s t =
    val new : t -> s t
    val yield : t -> s t -> s t
    val consume : s t -> (s t, t)
;;
\end{lstlisting}
Si noti che \texttt{Stream} ha 2 argomenti.

\hypertarget{Constraints sulle proprieta}{\subsubsection{Constraints sulle proprietà}}
In Fex, è possibile definire una proprietà della forma:
\begin{lstlisting}
property C2 a = C1 a Char => 
    <metodi>
;;
\end{lstlisting}
Questo tipo di definizione consiste nell'aggiungere il predicato \texttt{C1 a Char} alla proprietà \texttt{C2 a}.
Semanticamente, questo impone la seguente condizione sul programma:
    \[ \forall type . \; \exists inst(C2, type) \Longrightarrow \exists inst(C1, type, Char) \]
dove $ inst(C, \overline{t}) $ è, banalmente, l'instanza con nome di proprietà $ C $ e tipi $ \overline{t} $. \`E
possibile aggiungere più di un predicato alla stessa proprietà.

\hypertarget{Polimorfismo higher-kinded}{\subsubsection{Polimorfismo higher-kinded}}
Riprendiamo il precedente esempio con \texttt{Stream}:
\begin{lstlisting}
property Stream s t =
    val new : t -> s t
    val yield : t -> s t -> s t
    val consume : s t -> (s t, t)
;;
\end{lstlisting}
Gli argomenti di una proprietà sono sempre variabili di tipo. Come è già stato menzionato in precedenza, Fex supporta
tipi di kind arbitrari; gli argomenti delle proprietà non fanno eccezione da questo punto di vista, infatti, nell'esempio
si può notare come la variabile \textit{s} abbia kind:
\[ * \mapsto * \]
mentre la variabile \textit{t} ha kind:
\[ * \]

\hypertarget{Argomenti arbitrari delle istanze}{\subsubsection{Argomenti arbitrari delle istanze}}
Gli argomenti di un'istanza possono essere tipi di forma arbitraria, ad esempio:
\begin{lstlisting}
property C x y =
    <metodi>
;;

instance C a (Tuple3 a (T b) (m b)) =
    <implementazioni>
;;
\end{lstlisting}
Confrontando il precedente codice Fex con il seguente codice Haskell che può considerarsi quanto meno concettualmente
equivalente:
\begin{lstlisting}
{-# LANGUAGE MultiParamTypeClasses #-}

class C x y where
    <metodi>

instance C a (a, T b, m b) where
    <implementazioni>
\end{lstlisting}
si ha che, compilando questo programma Haskell soltanto con l'estensione all'inizio del codice, si incorrerà nel
seguente messaggio d'errore da parte del compilatore:
\begin{lstlisting}
Illegal instance declaration for 'C a (a, T b, m b)'
(All instance types must be of the form (T a1 ... an)
where a1 ... an are *distinct type variables*,
and each type variable appears at most once in the instance head.
Use FlexibleInstances if you want to disable this.)
\end{lstlisting}
Come suggerisce l'ultima riga del messaggio d'errore, per definire un'istanza del genere in un programma Haskell è
necessario utilizzare l'estensione \texttt{FlexibleInstances} \hyperlink{bibl7}{[7]}. In Fex, a differenza di Haskell,
quel tipo di instanza
è accettato di default. Il seguente vincolo sulla forma dei tipi riportato nel messaggio d'errore viene dunque rilassato:
    \[ TyCon(\overline{\alpha}) \]
dove $ \overline{\alpha} $ sono variabili di tipo distinte e $ TyCon $ è un costruttore di tipo. In seguito, verrà
mostrato quali vincoli (che riguardano i constraints e le istanze) possono essere rilassati e con quali conseguenze.

\hypertarget{Constraints sulle istanze}{\subsubsection{Constraints sulle istanze}}
L'utente ha anche la facoltà di definire istanze con uno o più predicati, ad esempio:
\begin{lstlisting}
instance Stream m String = Monad m =>
    <implementazioni>
;;
\end{lstlisting}
questo tipo di definizione genererà metodi di istanza con le seguenti firme:
\begin{lstlisting}
val new : Monad m => String -> m String
val yield : Monad m => String -> m String -> m String
val consume : Monad m => m String -> (m String, String)
\end{lstlisting}
Il constraint sull'istanza viene quindi applicato al tipo dei metodi di istanza.

\hypertarget{Condizioni sui constraints}{\subsection{Condizioni sui constraints}}
I constraints rappresentano un modo per ``restringere" i possibili tipi che possono istanziare una o più variabili di
tipo. I tipi nelle firme delle variabili (nonché nei type-hinting) possono possedere zero o più constraint; lo stesso
vale, come viene mostrato nei paragrafi precedenti, anche per le proprietà e per le istanze. Esistono, però, dei
vincoli sulla forma dei constraint. Come viene mostrato in \hyperlink{bibl8}{[8]} e \hyperlink{bibl9}{[9]}, esistono
condizioni sufficienti
affinché l'algoritmo di risoluzione delle istanze di Haskell termini. Prima di presentarle, diamo alcune notazioni:
\newline Dato un costrutto $ K $ della forma $ D \Rightarrow C $, definiamo
\begin{itemize}
    \item $ C $ come \textit{Contesto} del costrutto $ K $;
    \item $ H $ come \textit{Testa} del costrutto $ K $.
\end{itemize}
Ora presentiamo le \textit{condizioni di Paterson}:
\begin{itemize}
    \item il contesto $ C $ di una dichiarazione di type-class può menzionare solamente variabili di tipo e
    le variabili di tipo sono distinte in ogni singolo constraint in $ C $;
    \item Per ogni dichiarazione di istanza $ TyCl \; t_1 \; ... \; t_n \Rightarrow C $, nessuna variabile di tipo ha
    più occorrenze nel contesto $ C $ rispetto alla testa $ TyCl \; t_1 \; ... \; t_n $.
    \item Per ogni dichiarazione di istanza $ TyCl \; t_1 \; ... \; t_n \Rightarrow C $, ogni constraint nel contesto
    $ C $ ha meno costruttori e variabili di tipo (presi insieme e contando le ripetizioni) rispetto alla testa
    $ TyCl \; t_1 ... t_n $.
    \item Per ogni due dichiarazioni di istanze $ TyCl \; t_1 \; ... \; t_n \Rightarrow C $,
    $ TyCl \; t_1' \; ... \; t_n' \Rightarrow C' $, non deve esistere una sostituzione $ \varphi $ tale che:
    $ \varphi(t_1) = \varphi(t_1') $, ..., $ \varphi(t_n) = \varphi(t_n') $.
\end{itemize}
Informalmente, l'obiettivo è avere sempre dei contesti più ``piccoli" rispetto alle teste. La politica di Fex è
simile e applica i seguenti vincoli:
\begin{enumerate}
    \item Per ogni constraint $ c $, $ c $ deve contenere almeno una variabile di tipo;
    \item Per ogni istanza $ h \Rightarrow ctx $, il numero di occorrenze di ogni singola variabile $ v $ nel contesto
    $ ctx $ deve essere minore o uguale rispetto al numero di occorrenze di $ v $ nella testa $ h $;
    \item Per ogni istanza $ h \Rightarrow ctx $, le variabili di tipo di $ ctx $ devono essere innestate in meno
    costruttori di tipo rispetto alle variabili di tipo di $ h $ (contando le variabili tutte insieme).
\end{enumerate}
Il codice che implementa questi controlli risiede nel modulo \texttt{Compiler.Constraints.Check}.

\hypertarget{Kind-inference}{\subsection{Kind-inference}}
La kind-inference in Fex è necessaria per poter costruire i token che rappresentano i costruttori di tipi. Prima
di entrare nei dettagli, è bene fare alcune precisazioni sui kind. Innanzitutto, Fex non espone all'utente alcuna
sintassi per manipolare i kind o per scrivere espressioni con i kind. Come conseguenza, la definizione di kind in
Fex (cfr. \hyperlink{Tipi higher-kinded}{Tipi higher-kinded}) non viene estesa con dei quantificatori.
Questo non avviene per i tipi,
infatti, la nozione di mono-tipo viene successivamente estesa con la nozione di schema di tipo, la quale introduce
il polimorfismo parametrico.

L'ipotesi fondamentale (che connoteremo con HK) su cui si basa la kind-inference è sul kind del tipo \textit{funzione}:
\[ (\mapsto) : * \mapsto * \mapsto * \]
In generale, l'inferenza di kind avviene ``osservando" le definizioni dei tipi di dati algebrici, in particolare,
guardando le definizioni dei data-constructors. La definizione di un data-constructor ha la seguente forma:
\[ C \; ty_1 \; ... \; ty_n \]
Supponendo che il data-constructor $ C $ \textit{costruisca} il tipo $ T \; \alpha_1 \; ... \; \alpha_m $, allora
$ C $ avrà tipo (cfr. \hyperlink{Data-constructors}{Data-constructors}):
\[ ty_1 \mapsto ... \mapsto ty_n \mapsto T \; \alpha_1 \; ... \; \alpha_m \]
Come conseguenza dell'ipotesi HK si ha che:
\[ \forall i \in \{ 1, ..., n \}. \; ty_i : * \]
e:
\[ T \; \alpha_1 \; ... \; \alpha_m : * \]
Si ricordi che le variabili di tipo che compaiono nei tipi $ ty_i $ sono legate ai binders
$ \alpha_1 \; ... \; \alpha_m $. Una volta che tutti i data-constructors sono stati visitati, i kind delle variabili di
tipo saranno noti, quindi si può inferire il kind del costruttore di tipo $ T $. Durante la kind-inference, viene
utilizzato un ambiente di tipizzazione (che connoteremo con $ \Delta $) in cui vengono salvate le coppie della forma:
\[ \langle tipo, kind \rangle \]
dove il $ tipo $ è il nome di un qualsiasi tipo che compare senza parametri; sono comprese le variabili di tipo.
In questo contesto, i nomi dei tipi non hanno uno \textit{scope} in quanto è garantito che non ci siano tipi diversi con
identificatori uguali nel programma.
Questa condizione è banalmente garantita per i tipi concreti in quanto non è possibile definire tipi molteplici con nomi
uguali, mentre, per quanto riguarda le variabili di tipo, il renaming
delle variabili di tipo (cfr. \hyperlink{Renaming delle variabili di tipo}{Renaming delle variabili di tipo}) garantisce
la condizione. Inoltre, la sintassi degli
identificatori garantisce che non vi siano variabili di tipo e tipi concreti con nomi uguali.
In generale, vengono visitate tutte le definizioni di data-constructors nelle definizioni di tipo, seguendo queste regole:
\begin{itemize}
    \item se un tipo $ ty_i $ ha la forma di variabile di tipo $ \alpha_j $, per un qualche $ j \in {1, ..., m} $, allora
    per HK, $ \alpha_j : * $;
    \item se un tipo $ ty_i $ ha come testa un tipo concreto $ T' $ che esiste già in $ \Delta $, allora si possono
    inferire i kind degli argomenti di $ ty_i $; per i tipi che esistono già in $ \Delta $ e che non hanno una variabile
    di kind come kind associato, viene effettuato il kind-check, mentre per i tipi non presenti in $ \Delta $ o che
    hanno una variabile di kind come kind associato, vengono inseriti in $ \Delta $ con il nuovo kind inferito. Infine,
    deve valere che $ ty_i : * $;
    \item se un tipo $ ty_i $ ha come testa un tipo concreto $ T' $ che non esiste in $ \Delta $, allora si inferiscono
    i kind degli argomenti di $ ty_i $, senza effettuare la fase di kind-check. Infine, $ T' $ viene inserito in
    $ \Delta $ con kind $ K_1 \mapsto ... \mapsto K_r $, dove $ K_1, ..., K_r $ sono i kind degli argomenti applicati a
    $ T' $;
    \item se un tipo $ ty_i $ ha come testa una variabile di tipo $ m $, deve valere che $ m $ non compare come suo
    argomento diretto. Nel caso non valga la condizione, il programma viene respinto con un errore. Poi l'inferenza
    procede come per i tipi concreti;
    \item se un tipo $ arg $ compare come argomento i-esimo di un costruttore di tipo $ tycon $, se $ tycon \in \Delta $,
    allora sia $ K $ il suo kind, il kind inferito di $ arg $ sarà quello dell'i-esimo argomento di $ K $. Se
    $ tycon \notin \Delta $, allora viene creata una variabile di kind $ \kappa $ e in $ \Delta $ viene inserito il
    binding $ \langle arg, \kappa \rangle $;
    \item le variabili di tipo che compaiono soltanto nei binders avranno kind $ * $;
    \item dopo che tutte le definizioni di tipo sono state visitate, le variabili di tipo in $ \Delta $ che hanno
    variabili di kind come kind associato, acquisiscono il kind $ * $. Questo previene un ulteriore livello di polimorfismo;
    \item ogni qual volta un tipo che ha una variabile di kind $ \kappa $ come kind associato acquisisce un nuovo kind
    $ K $, le occorrenze di $ \kappa $ in $ \Delta $ come variabili libere devono essere $ specializzate $ con $ K $.
\end{itemize}

\hypertarget{Data-constructors}{\subsection{Data-constructors}}
Una delle tabelle di simboli che appare nel modulo \texttt{Compiler.Types.Table} è \texttt{DataConsTable}. In essa
vengono salvati i \textit{data constructor} associati ai loro tipi. I costruttori - che nell'AST vengono rappresentati
dal token \texttt{ADTConstructor} - vengono trasformati nel token \texttt{NotedVal} presente nel modulo
\texttt{Compiler.Ast.Typed} che consiste nella seguente coppia:
\[ \langle dataConRep, type \rangle \]
dove $ dataConRep $ è la rappresentazione sotto forma di stringa del data constructor, mentre $ ty $ è il tipo del
costruttore. Per quanto riguarda il tipo del costruttore, prendiamo in considerazione il seguente esempio di codice
Fex:
\begin{lstlisting}
type Foo x y = Bar Int (M x) y
\end{lstlisting}
In questa definizione di tipo di dato algebrico, vi è un solo costruttore: \texttt{Bar}. Il suo tipo è il seguente:
\[ \forall x, y. \; Int \mapsto M \; x \mapsto y \mapsto Foo \; x \; y \]
Il tipo di ritorno è sempre dato dalla definizione del tipo. Il modulo che si occupa di trasformare i token
\texttt{ADTConstructor} in \texttt{NotedVal} e aggiungerli nella tabella \texttt{DataConsTable} è
\texttt{Compiler.Types.Builder.Cons}.

\hypertarget{Constraint constructors}{\subsection{Constraint constructors}}
Dopo il check sui constraints e la generazione di data-constructors, il compilatore si occupa di generare i cosiddetti
``\textit{constraint constructors}", ovvero token che fungono da modelli per la costruzione di predicati. Il modulo
che li genera è \texttt{Compiler.Types.Tables}; esso prende in input i token dell'AST che rappresentano le proprietà
(\texttt{Interface}) e da essi costruisce i token tipati \texttt{LangNewConstraint} e li inserisce nella tabella
\texttt{ConstraintsTable}. \`E compito di questo modulo controllare che le classi non formino cicli tra loro, ad
esempio, il seguente programma viene rifiutato dal compilatore:
\begin{lstlisting}
property Foo a = Bar (M a) Int =>
    <metodi>
;;

property Bar x y = Foo (K x Char) =>
    <metodi>
;;
\end{lstlisting}
Si noti come i vincoli sui constraints vengano tutti rispettati.

\hypertarget{Costruzione delle istanze}{\subsection{Costruzione delle istanze}}
Nel modulo \texttt{Compiler.Types.Builder.Instances} vengono create le seguenti 3 tabelle:
\begin{itemize}
    \item \texttt{InstsTable}, ovvero la tabella che contiene i bindings (sotto forma di token dell'AST) delle istanze;
    \item \texttt{PropMethodsTable}, ovvero la tabella che contiene i metodi delle proprietà sotto forma di token
    tipati;
    \item \texttt{ImplTable}, ovvero la tabella che contiene i constraints che derivano dalle istanze definite
    dall'utente;
\end{itemize}
Per quanto riguarda l'ultima tabella, data la definizione di un'istanza:
\begin{lstlisting}
instance Eq Char =
    <implementazioni>
;;
\end{lstlisting}
la testa \texttt{Eq Char} ha la forma di un constraint, nonostante non rispetti uno dei vincoli sui constraints (un
constraint deve avere almeno una variabile di tipo), infatti l'istanza viene salvata sotto forma di constraint
(\texttt{LangSpecConstraint}).
Le tre tabelle vengono create in un unico modulo per una questione di efficienza (i token delle istanze vengono visitati
una sola volta). Inoltre, in questo
modulo viene effettuato il controllo del vincolo sull'esistenza dei predicati delle proprietà (cfr.
\hyperlink{Constraints sulle proprieta}{Constraints sulle proprietà}). Prima di inserire i bindings delle istanze nella
tabella \texttt{InstsTable}, viene eseguita una fase di desugaring su di essi. Il problema nasce dal fatto che, data
una proprietà, può esistere un numero arbitrario di istanze e con esse, un numero arbitrario di metodi con lo stesso
nome, perciò è necessario identificare univocamente i metodi di ogni singoli istanza. Il modulo
\texttt{Compiler.Desugar.Names} si occupa di, dato il nome di una variabile e una sequenza di constraints (gli argomenti
di un'istanza), creare un identificatore univoco che non può essere uguale a nessun altro identificatore nel programma.

\hypertarget{Preparazione alla type-inference}{\subsection{Preparazione alla type-inference}}
Prima di effettuare la type-inference è necessario fare alcune considerazioni ed eseguire alcune computazioni.
Innazitutto, il type-system di riferimento è \textit{Hindley-Milner} (o Damas-Hindley-Milner, in seguito lo indicheremo con HM)
con alcune estensioni che riguardano il costrutto del pattern matching, l'inferenza di definizioni ricorsive e la
risoluzione delle istanze. Il modulo che si occupa della preparazione alla type-inference è \texttt{Compiler.Types.Prepare}
Introdurremo il type-system in modo preciso nella sezione \hyperlink{Type-inference}{Type-inference}, tuttavia, ora
presentiamo alcune
caratteristiche del type-system che è bene conoscere come premessa alla preparazione della type-inference.
Innanzitutto, HM viene descritto formalmente da un insieme di \textit{regole} le quali si occupano di ``tipare"
espressioni di un formalismo fissato: una versione del lambda-calcolo estesa con il costrutto ``let..in".
Una delle regole è quella sull'inferenza del tipo di un simbolo (la regola la chiameremo \textbf{Var})
la quale, informalmente, sostiene che: data l'ipotesi di un simbolo $ x $ con schema di tipo $ \sigma $ nel contesto di
un ambiente di tipizzazione $ \Gamma $ e $ \tau $ l'istanziazione del tipo $ \sigma $, si conclude che il simbolo
$ x $ ha tipo $ \tau $. Quindi, si ha che nel momento in cui viene incontrato un simbolo $ x $ in un'espressione, esso deve
essere presente nell'ambiente di tipizzazione $ \Gamma $, al quale, informalmente, si può pensare come una mappa di
simboli e tipi associati ai simboli.
Un'altra regola utile è quella
sull'inferenza delle definizioni mutualmente ricorsive. In particolare, essa sostiene che, l'inferenza di definizioni
mutualmente ricorsive avviene in ``gruppo", ovvero un insieme $ {f_1, ..., f_n} $ di simboli mutualmente ricorsivi
tra loro viene considerato come un'unica definizione (regola \textbf{Rec}) (cfr. \hyperlink{Ricorsione}{Ricorsione}).
Queste regole serviranno come premessa ad alcune fasi nella preparazione della type inference.

\hypertarget{Bindings delle istanze}{\subsubsection{Metodi delle istanze}}
Vale la pena notare come in Core non esista un costrutto particolare per i metodi delle istanze, infatti, il
costrutto di GHC che rappresenta le istanze (\texttt{ClsInst}) non porta con sè informazioni che riguardano i
metodi \hyperlink{bibl10}{[10]}, perciò, è in un qualche modo necessario gestirli. La politica del compilatore è quella di
aggiungerli nell'insieme di bindings del programma e trattarli come qualsiasi altra funzione. La presenza di eventuali
conflitti tra gli identificatori è stata già risolta (cfr. \hyperlink{Costruzione delle istanze}{Costruzione delle istanze}).

\hypertarget{Simboli innestati univoci}{\subsubsection{Simboli innestati univoci}}
Prima della type-inference, vengono visitate tutte le espressioni dei bindings e vengono creati nuovi identificatori
per ogni definizione innestata di variabile (costrutto \texttt{let..in}). I nuovi identificatori sono resi univoci in
tutto il programma. Il modulo che crea e garantisce che i nuovi identificatori siano univoci è
\texttt{Compiler.Desugar.Names}. La proprietà di unicità degli identificatori innestati è molto importante, vedremo in
seguito il motivo (cfr. \hyperlink{Type-inference}{Type-inference}).

\hypertarget{Clusters di definizioni mutualmente ricorsive}{\subsubsection{Clusters di definizioni mutualmente ricorsive}}
Come è stato già menzionato precedentemente, le definizioni mutualmente ricorsive devono essere considerate come insiemi
e non singolarmente. Perciò è necessario distinguere due tipi di bindings, quelli mutualmente ricorsivi e i restanti:
\begin{lstlisting}
data RawBinding =
      RawNonRec (Raw.SDUnion With.ProgState)
    | RawRec [Raw.SDUnion With.ProgState]
\end{lstlisting}
La precedente definizione è nel modulo \texttt{Compiler.Types.Prepare.Lib} e divide i bindings dell'AST in bindings
non ricorsivi e bindings mutualmente ricorsivi. Rimane, quindi, da dividere i bindings del programma. Prima di presentare
l'algoritmo, bisogna effettuare una precisazione molto importante. In precedenza, abbiamo aggiunto i bindings delle
istanze alla lista dei bindings del programma, cambiando, però, gli identificatori, rendendoli univoci all'interno del
programma. Per riconoscere i bindings mutualmente ricorsivi è necessaria una funzione che calcoli le dipendenze
di una definizione. Nelle espressioni, in generale, possono essere menzionati i metodi delle proprietà, ma non possono,
in alcun modo, essere menzionati i metodi delle istanze. Inoltre, il tipo dei metodi delle proprietà è sempre noto a
priori e non è possibile, prima della type-inference, inferire le istanze (cfr. \hyperlink{Dispatch statico}{Dispatch statico})
corrette dei metodi.
Nel calcolo delle dipendenze, i metodi di proprietà possono, quindi, essere esclusi. La funzione \textit{depsOf}, la quale
prende in input un binding \texttt{b} dell'AST e la tabella \texttt{PropMethodsTable}, calcolerà le dipendenze di
\texttt{b}, escludendo le dipendenze che provengono dalla tabella dei metodi di proprietà. L'insieme dei bindings di
un programma può essere visto come un grafo orientato $ G $ in cui:
\begin{itemize}
    \item i nodi sono i bindings;
    \item gli archi sono le dipendenze di un binding.
\end{itemize}
Il problema di trovare i clusters di definizioni mutualmente ricorsive corrisponde a trovare le \textit{componenti
fortemente connesse} del grafo $ G $. Il modulo che implementa l'algoritmo è \texttt{Compiler.Types.Prepare.Recursion}
e utilizza la libreria \texttt{Data.Graph} (\hyperlink{bibl26}{[26]}).

\hypertarget{Sorting dei bindings}{\subsubsection{Sorting dei bindings}}
A questo punto, abbiamo una lista di \texttt{RawBinding}. Possiamo fare la seguente osservazione: i bindings possono
essere visti come un grafo orientato aciclico $ G $, in cui, come prima:
\begin{itemize}
    \item i nodi sono i bindings;
    \item gli archi sono le dipendenze di un binding.
\end{itemize}
L'unica differenza rispetto a prima è l'aciclicità del grafo. Questa proprietà è garantita dal seguente fatto: se
esistesse un ciclo
in $ G $, allora l'algoritmo di raggruppamento dei clusters l'avrebbe trovato e avrebbe creato un cluster di definizioni
mutualmente ricorsive tra loro. La regola di inferenza \textbf{Var} impone che si conosca sempre lo schema di tipo di
un simbolo, perciò, è necessario, prima di effettuare la type-inference, ordinare i bindings in base alle loro dipendenze.
L'ordinamento viene effettuato nel modulo \texttt{Compiler.Types.Prepare.Sort} e l'algoritmo utilizzato è un'estensione
dell'insertion-sort, dove, la nozione di ordinamento è data dalle dipendenze dei bindings, ovvero: \newline
Siano $ b $ e $ b' $ due bindings, si ha che:
\begin{itemize}
    \item $ b > b' $, se $ b' $ è una dipendenza diretta di $ b $ oppure esistono dei bindings $ c_1, ..., c_n $ tali che:
    \newline $ (b > c_1) \wedge (c_1 > c_2) \wedge ... \wedge (c_{n-1} > c_n) \wedge (c_n > b') $;
    \item $ b = b' $, se la componente di $ b $ in $ G $ non è raggiungibile da nessun nodo della componente $ b' $.
\end{itemize}
la proprietà di transitività di questa nozione di ordinamento è garantita dall'assenza di cicli all'interno del grafo.
Di seguito vi è l'algoritmo di sorting di una singola componente del grafo:
\begin{lstlisting}
sortComponent(bindings, remaining, component):
    match bindings, remaining with
        [], [] -> component
        [], (_ : _) -> sortComponent remaining [] component
        (b : t), _ ->
            let (inserted, component') = tryInsert b component in
                if inserted
                then sortComponent t remaining component'
                else sortComponent t (addTail b remaining) component 
\end{lstlisting}
L'algoritmo prende in input i bindings di una componente, i bindings rimanenti e la componente già ordinata. Ora
presentiamo l'algoritmo di inserzione di un binding in una componente già ordinata:
\begin{lstlisting}
tryInsert(binding, component):
    match component with
        [] -> [component]
        (b : t) ->
            if b in depsOf(binding)
            then tryInsert binding t
            else if binding in depsOf(b)
            then (True, binding : component)
            else (False, component)
\end{lstlisting}
Ora abbiamo implementato una versione alternativa dell'insertion sort in cui gli inserimenti vengono effettuati
solamente se si è a conoscenza che il binding da inserire fa parte delle dipendenze dirette di un altro binding già
inserito nella componente. Inoltre, se non vi sono abbastanza informazioni per inserire un binding in una componente,
il suo inserimento viene ``rimandato" (viene inserito nei bindings rimanenti) e verrà effettuato in un'iterazione
successiva. Il codice si trova nel modulo \texttt{Compiler.Types.Prepare.Sort}.

Dopo l'ordinamento dei bindings, è garantito che, ogni qual volta verrà inferito un binding, la premessa
della regola \textbf{Var} sia vera.

\hypertarget{Type-inference}{\subsection{Type-inference}}
Fex permette all'utente di indicare o non indicare il tipo di una variabile. Nel caso non venga indicato, il
compilatore dovrà inferire il tipo della variabile definita, quindi dovrà fare una ``scelta". Per farlo, dovrà prima
inferire il tipo dell'espressione legata alla variabile, ad esempio:
\begin{lstlisting}
type Maybe a = Nothing | Just a
let x = Nothing
\end{lstlisting}
Si può affermare che l'espressione legata alla variabile $ x $ abbia tipo $ Maybe \; \alpha $. Ora, il compilatore
deve scegliere un tipo da sostituire alla variabile di tipo $ \alpha $. Se la scelta fosse arbitraria, ad esempio
$ Int $, l'utilizzo di $ x $ sarebbe limitato solo a un tipo ovvero $ Maybe \; Int $. Un'altra possibile soluzione
potrebbe essere associare a $ x $ il tipo $ Maybe \; a $. Tuttavia, anche questa scelta risulterebbe limitante in quanto,
come è stato già esposto nella sezione sul type-system
(cfr. \hyperlink{Schemi di tipo e polimorfismo parametrico}{Schemi di tipo e polimorfismo parametrico}), la variabile di
tipo $ \alpha $ denoterebbe un tipo fissato non ancora conosciuto. L'introduzione di uno schema di tipo rappresenta una
scelta ``più generale":
\[ x : \forall \alpha. \; Maybe \; \alpha \]
In questo caso, le occorrenze di $ x $ nel programma possono avere molteplici tipi.

\hypertarget{Damas-Hindley-Milner}{\subsubsection{Damas-Hindley-Milner}}
Come è stato già menzionato precedentemente, il sistema di tipi di Fex si basa sul type-system di Damas-Hindley-Milner
(cfr. \hyperlink{bibl27}{[27]} \hyperlink{bibl28}{[28]} \hyperlink{bibl29}{[29]}).
L'algoritmo di inferenza su cui si basa inferisce il tipo più \textit{generale} possibile. Tuttavia, nella letteratura,
vengono presentati due algoritmi principali, ma nella pratica, solo uno di essi è implementato dai compilatori dei
linguaggi di programmazione. Verranno entrambi menzionati, ma solo le regole di inferenza dell'algoritmo effettivamente
implementato verranno esposte:
\begin{itemize}
    \item \textit{Algoritmo W}, è l'algoritmo effettivamente utilizzato dalle implementazioni, in quanto tiene traccia
    delle sostituzioni generate dalle eventuali unificazioni (cfr. \hyperlink{Unificazione}{Unificazione}) e le applica
    all'ambiente di
    tipizzazione. Proprio a causa della gestione delle sostituzioni, la complessità computazionale dell'algoritmo
    nel caso peggiore è esponenziale.
    \item \textit{Algoritmo J}, è un algoritmo più efficiente di \textit{W} (complessità lineare nella lunghezza
    dell'espressione),
    tuttavia, esso non gestisce le sostituzioni, o almeno, esse vengono considerate come side-effects. Viene
    presentato in letteratura come alternativa efficiente a \textit{W}.
\end{itemize}

Come è stato già anticipato, HM viene presentato sotto forma di \textit{regole}, le quali hanno la seguente forma:
\newline
\[ \infer[\textbf{Regola}]{Conclusione}{Premesse} \]
dove le premesse sono un insieme di \textit{giudizi} e \textit{predicati} (non nel senso di constraints) e la conclusione
è un \textit{giudizio}. Un \textit{giudizio} è un'affermazione sul tipo di un simbolo, ad esempio:
\[ x : \sigma \]
afferma che il simbolo $ x $ ha tipo $ \tau $. HM necessita, inoltre, di un modo per ``tener traccia" dei giudizi,
ovvero un modo per accoppiare un simbolo con un tipo. Introduciamo, quindi,
un altro componente: il \textit{contesto} o \textit{ambiente di tipizzazione}. In generale, i giudizi terranno conto
del contesto, infatti avranno la seguente forma:
\[ \Gamma \vdash_W x : \sigma \]
questa scritta afferma che sotto le ipotesi in $ \Gamma $, il token $ x $ ha tipo $ \sigma $.
Ora verranno elencate le regole di inferenza.

\hypertarget{Inferenza di variabile}{\subsubsection{Inferenza di variabile}}
Prima di presentare la regola d'inferenza delle variabili è necessario introdurre due ulteriori regole. La prima è la
regola di specializzazione e introduce una nozione d'ordine parziale tra tipi:
\[ \infer[\textbf{Spec}]{\forall \alpha_1 ... \forall \alpha_n. \; \tau \sqsubseteq \forall \beta_1 ... \beta_m. \; \tau'}{\tau' = \{ \alpha_i \mapsto \tau_i \}\tau & \beta_i \notin free(\forall \alpha_1 ... \forall \alpha_n . \; \tau)} \]
Questa regola sostiene che se esiste una sostituzione $ S = \{ \alpha_i \mapsto \tau_i \} $ tale che $ \tau' = S \tau $,
allora la versione ``quantificata" dei mono-tipi $ \tau $ e $ \tau' $ rispetta la relazione d'ordine parziale
$ \sqsubseteq $. Nella premessa vi è un'ulteriore condizione che afferma che le variabili quantificate di $ \tau' $
non devono apparire come variabili libere in $ \forall \alpha_1 ... \alpha_n. \; \tau $; questo perché le variabili
non legate da un quantificatore (quindi libere) non devono essere sostituite, bensì devono essere trattate come costanti.
All'interno del codice, i token tipati che implementano la type-class \texttt{SpecType} definita in \texttt{Compiler.Ast.Typed}
permettono l'applicazione di una sostituzione al loro tipo.
La prossima regola rappresenta l'algoritmo di instanziazione:
\[ \infer[\textbf{Inst}]{\Gamma \vdash_W e : \tau}{\Gamma \vdash_W e : \sigma & \sigma \sqsubseteq \tau} \]
Questa regola è utile per istanziare uno schema di tipo in mono-tipo. \`E chiaro che è necessario tener conto delle
eventuali variabili libere nello schema di tipo $ \sigma $, ma queste vengono gestite dalla regola di specializzazione.
Ora, possiamo esporre la regola di inferenza di una variabile, la quale è stata già introdotta in precedenza, seppur
in maniera informale.
\[ \infer[\textbf{Var}]{\Gamma \vdash_W x : \tau, \; \emptyset}{x : \sigma \in \Gamma & \tau = inst(\sigma)} \]
\`E doveroso notare che nella conclusione, oltre al giudizio sul tipo della variabile, vi è un altro valore in ``output",
ovvero le eventuali sostituzioni generate dall'inferenza dei costrutti coinvolti. In questo non vi è alcuna sostituzione.

\hypertarget{Inferenza di applicazione}{\subsubsection{Inferenza di applicazione}}
La prossima regola serve per inferire l'applicazione di due espressioni:
\[ \infer[\textbf{App}]{\Gamma \vdash_W e_0 \; e_1 : S_2\tau', \; S_2S_1S_0}{\Gamma \vdash_W e_0 : \tau_0, \; S_0 & S_0\Gamma \vdash_W e_1 : \tau_1, \; S_1 & \tau' = newvar & S_2 = mgu(S_1\tau_0, \tau1 \mapsto \tau')} \]
Vi sono due nuovi operatori:
\begin{itemize}
    \item $ newvar $, il quale ritorna una nuova variabile di tipo $ \alpha $ tale che $ \alpha \notin free(\Gamma) $;
    \item $ mgu $, il quale rappresenta l'algoritmo di unificazione che verrà presentato dettagliatamente nel prossimo
    paragrafo. Tralasciando, quindi, i dettagli implementativi, tale operatore serve per trovare il tipo più generale.
    Come risultato, fornisce una sostituzione $ S_2 $, la quale viene successivamente applicata al mono-tipo $ \tau' $
    per ottenere il tipo di ritorno del costrutto di applicazione.
\end{itemize}
Si noti che, a differenza di \textbf{Var}, in questa regola vi sono tre sostituzioni in output
\begin{itemize}
    \item $ S_0 $ derivante dall'inferenza della prima espressione;
    \item $ S_1 $ derivante dall'inferenza della seconda espressione;
    \item $ S_2 $ derivante dall'esecuzione dell'algoritmo di unificazione tra tipi;
\end{itemize}
Nell'inferenza di $ e_1 $, l'ambiente di tipizzazione di riferimento è $ \Gamma $ a cui viene applicata la sostituzione
$ S_0 $. Una possibile ottimizzazione è la seguente:
\[ \infer[\textbf{App}]{S_0\Gamma \vdash_W e_0 \; e_1 : S_2\tau', \; S_2S_1}{\Gamma \vdash_W e_0 : \tau_0, \; S_0 & S_0\Gamma \vdash_W e_1 : \tau_1, \; S_1 & \tau' = newvar & S_2 = mgu(S_1\tau_0, \tau1 \mapsto \tau')} \]
La sostituzione $ S_0 $ viene, quindi, spostata dall'output direttamente all'ambiente di tipizzazione. Questo per evitare
di applicare la sostituzione all'ambiente di tipizzazione due volte: una prima di inferire $ e_1 $, una dopo il risultato
di output (si presume che le sostituzioni vengano tutte applicate).

\hypertarget{Unificazione}{\subsubsection{Unificazione}}
L'unificazione è quel processo di risoluzione di equazioni tra espressioni simboliche. In generale, i mono-tipi nel
type-system utilizzato in HM (e conseguentemente anche quello utilizzato da Fex) possono essere considerati come
termini di un'equazione. Si parla di mono-tipi, poiché l'unificazione non prevede la quantificazione delle variabili
all'interno di un'equazione. Un problema di unificazione è un insieme finito di equazioni:
\[ \{ \tau_1 = \tau_1' \; ... \; \tau_n = \tau_n' \} \]
dove $ \tau_i, \tau_i' $ sono dei mono-tipi. La soluzione di un problema di unificazione è data da una sostituzione
$ S = \{ \alpha_j \mapsto \mu_j \} $, dove $ \alpha_j $ è una variabile di tipo e $ \mu_j $ è un mono-tipo, tale che:
\[ \forall i \in \{1, ..., n\}. \; S\tau_i = S\tau_i' \]
Il codice sorgente che si occupa dell'unificazione si trova nel modulo \texttt{Compiler.Ast.Typed}, in particolare,
la funzione \texttt{rawUnify} implementa l'algoritmo di unificazione, osserviamo la sua firma:
\begin{lstlisting}
rawUnify
    :: LangHigherType a
    -> LangHigherType a
    -> IsSpecTest
    -> Either (UnificationError a) (Substitution a)
\end{lstlisting}
Notiamo subito che vi sono tre argomenti e che il tipo di ritorno è un result-type.
\begin{lstlisting}
data UnificationError a =
      UnmatchTypes (LangHigherType a) (LangHigherType a)
    | TrySwap (LangHigherType a) (LangHigherType a)
    | OccursCheck (LangHigherType a) (LangHigherType a)
    | UnmatchKinds (LangHigherType a) (LangHigherType a)

type IsSpecTest = Bool
\end{lstlisting}
Il tipo \texttt{UnificationError} possiede 4 casi d'errore e tutti prendono in input due mono-tipi:
\begin{itemize}
    \item \texttt{UnmatchTypes}, ritornato quando non vi è alcuna soluzione all'equazione tra due mono-tipi;
    \item \texttt{TrySwap}, simile a \texttt{UnmatchTypes} nella semantica, ma indica all'algoritmo di non terminare
    subito, analizzeremo approfonditamente questo caso in seguito;
    \item \texttt{OccursCheck}, ritornato quando vi è il tentativo di risolvere un problema del tipo:
        \[ \alpha = f \; ... \; \alpha \; ...  \]
    Questo tipo di equazione porterebbe come risultato un termine infinito visto che $ \alpha $ è sotto-termine di
    se stesso;
    \item \texttt{UnmatchKinds}, ritornato quando i kinds di due mono-tipi non coincidono. In questo caso, non vi
    può essere alcuna soluzione all'equazione, si pensi, ad esempio, a:
        \[ M \; \alpha = M \; Int \; \beta \]
\end{itemize}
Ora passiamo all'algoritmo vero e proprio, il quale sarà presentato mostrando direttamente soltanto una parte di esso
all'interno del codice sorgente in \texttt{rawUnify}:
\begin{lstlisting}
rawUnify monoTy monoTy' isSpecTest =
    case unification monoTy monoTy' empty of
        Left err -> Left err
        Right vvars -> Right $ elems vvars
    where
        unification lhty lhty' vvars
            | not $ sameInfrdKindOf lhty lhty' =
                Left $ UnmatchKinds lhty lhty'
            | occursCheck lhty lhty' =
                Left $ OccursCheck lhty lhty'
            | tyVarAndConcrete lhty lhty' =
                if isSpecTest
                then Left $ UnmatchTypes lhty lhty'
                else unification lhty' lhty vvars
            | bothConcrete lhty lhty' =
                if headEq lhty lhty'
                then argsUnification lhty lhty' vvars
                else Left $ UnmatchTypes lhty lhty'
            | bothTyVar lhty lhty' && argsOf lhty' > argsOf lhty =
                if isSpecTest
                then Left $ UnmatchTypes lhty lhty'
                else unification lhty' lhty vvars
            | otherwise =
                argsUnification lhty lhty' vvars

        argsUnification =
            if isSpecTest
            then unifyOnArgs
            else twoChancesUnifyArgs

        twoChancesUnifyArgs lhty lhty' vvars =
            case unifyOnArgs lhty lhty' vvars of
                Left err @ (TrySwap _ _) ->
                    case unifyOnArgs lhty' lhty vvars of
                        Left _ -> Left err
                        ok @ (Right _) -> ok
                err @ (Left _) -> err
                ok @ (Right _) -> ok

        <resto del codice>
--$ rm this when building
\end{lstlisting}
I commenti nel codice sorgente sono stati rimossi. Come è stato fatto notare in precedenza,
l'algoritmo prende in input tre argomenti, di cui i primi due sono i mono-tipi sui quali si vuole calcolare l'unificazione.
Il terzo argomento è un flag booleano che denota se \texttt{rawUnify} deve essere un test di specializzazione
(\texttt{True}) o semplicemente l'algoritmo di unificazione (\texttt{False}). Un test di specializzazione è l'equivalente
dell'unificazione, ma può essere fatta in un solo ``verso", ovvero se, dati due mono-tipi $ \tau $ e $ \tau' $ e data
la sostituzione di ritorno $ S = \{ \alpha_i \mapsto \tau_i \} $ conseguente alla valutazione dell'unificazione di
$ \tau $ e $ \tau' $, vale:
\[ \forall \alpha_i \in left(S). \; \alpha_i \in \tau' \]
dove con $ \alpha_i \in left(S) $ si intendono solamente le variabili di tipo nella parte sinistra di una sostituzione,
ovvero quelle che devono essere sostituite. Questo rende l'unificazione più stringente
e ciò è utile quando è richiesto che uno dei due mono-tipi non venga modifcato,
ad esempio, quando deve essere effettuato il type-check con un tipo che deriva da una annotazione di tipo (type-hinting).
Infatti, ci si aspetta che, data un'espressione con type-hinting, l'espressione abbia esattamente il tipo indicato
nell'annotazione di tipo.

L'obiettivo dell'algoritmo è riportarsi in una situazione tale che l'equazione tra i due mono-tipi ha la seguente forma:
\[ \alpha = \tau \]
dove $ \alpha $ è una variabile di tipo e $ \tau $ è un mono-tipo. Questo perché tale equazione ha la forma di una
sostituzione: $ \alpha = \tau $.
Tuttavia, nell'algoritmo, viene considerata la forma:
\[ \tau = \alpha \]
Di fatto, però, la semantica dell'unificazione non cambia; i termini dell'equazione possono essere anche invertiti,
cambiando, però, il ``verso" dei controlli e delle operazioni nell'algoritmo.
Per ottenere un'equazione di quella forma vi sono due operazioni fondamentali:
\begin{itemize}
    \item \textit{swap}, in cui, se deve essere effettuata l'unificazione di due mono-tipi $ \tau $ e $ \tau' $,
    l'ordine di quest'ultimi viene scambiato. Questa operazione è utile quando il mono-tipo $ \tau' $ è più
    ``specializzato" rispetto a $ \tau $. In generale, il mono-tipo più generale (o meno specializzato) possibile
    è una variabile singoletto, la quale, come è stato menzionato prima, deve stare nella parte destra dell'equazione.
    \item \textit{decompose}, in cui, se deve essere effettuata l'unificazione di due mono-tipi $ \tau $ e $ \tau' $,
    viene effettuata l'unificazione tra i vari argomenti dei due mono-tipi, raccogliendo successivamente le
    varie sostituzioni ottenute. Questa operazione è utile quando un'operazione di \textit{swap} non è vantaggiosa (ovvero
    il mono-tipo $ \tau $ è già ``più specializzato" di $ \tau' $). Essa permette di considerare mono-tipi più piccoli.
\end{itemize}

Analizzando il codice, la prima chiamata viene fatta a \texttt{unification}, la quale considera varie casistiche:
\begin{enumerate}
    \item
    \begin{lstlisting}
| not $ sameInfrdKindOf lhty lhty' =
    Left $ UnmatchKinds lhty lhty'
    \end{lstlisting}
    Nel primo caso viene effettuato un controllo sui kind dei due mono-tipi. Se sono diversi, viene restituito un errore;
    \item
    \begin{lstlisting}
| occursCheck lhty lhty' =
    Left $ OccursCheck lhty lhty'
--$ rm this when building
    \end{lstlisting}
    Nel secondo caso viene effettuato un altro controllo: il cosiddetto \textit{occurs check}. Se uno dei due mono-tipi
    ha la forma di una variabile di tipo e quest'ultima compare nell'altro mono-tipo, allora viene ritornato un errore.
    \item
    \begin{lstlisting}
| tyVarAndConcrete lhty lhty' =
    if isSpecTest
    then Left $ UnmatchTypes lhty lhty'
    else unification lhty' lhty vvars
--$ rm this when building
    \end{lstlisting}
    Nel terzo caso vi è il primo controllo sulla struttura dei mono-tipi: se \texttt{lhty} ha come testa una variabile
    di tipo e \texttt{lhty'} ha come testa un tipo concreto, allora vi è un'operazione di \textit{swap}, ovvero vi è
    una chiamata ricorsiva di \texttt{unification} con gli argomenti invertiti. Se l'unificazione è \textit{strict},
    viene ritornato un errore, in quanto una variabile di tipo, per definizione, non è più specializzata di un tipo
    concreto. In questo caso, l'operazione di \textit{swap} non può essere effettuata in quanto verrebbe a meno il vincolo
    secondo il quale le variabili del primo mono-tipo (\texttt{lhty}) non devono essere sostituite.
    \item
    \begin{lstlisting}
| bothConcrete lhty lhty' =
    if headEq lhty lhty'
    then argsUnification lhty lhty' vvars
    else Left $ UnmatchTypes lhty lhty'
--$ rm this when building
    \end{lstlisting}
    Nel quarto caso viene testato se entrambi i mono-tipi hanno tipi concreti come testa; se così fosse, le teste
    devono rappresentare lo stesso tipo concreto, altrimenti viene restituito un errore, in quanto non esisterebbe
    alcuna sostituzione che permetterebbe l'uguaglianza tra i due mono-tipi. Se, invece, le teste sono uguali, allora
    viene eseguita un'azione di \textit{decompose}, ovvero vengono effettuate delle chiamate ricorsive sugli argomenti
    dei due mono-tipi. Analizzeremo la funzione \texttt{argsUnification} in seguito.
    \item
    \begin{lstlisting}
| bothTyVar lhty lhty' && argsOf lhty' > argsOf lhty =
    if isSpecTest
    then Left $ UnmatchTypes lhty lhty'
    else unification lhty' lhty vvars
--$ rm this when building
    \end{lstlisting}
    Nel quinto caso viene controllato se entrambi i mono-tipi hanno variabili di tipo come testa e se il numero di
    argomenti del secondo mono-tipo è strettamente maggiore di quello del primo. In questo caso, vi è un'operazione
    di \textit{swap} e la conseguente chiamata ricorsiva. Come prima, se l'unificazione è \textit{strict},
    l'operazione di \textit{swap} non può essere effettuata e viene restituito un errore.
    \item
    \begin{lstlisting}
| otherwise =
    argsUnification lhty lhty' vvars
    \end{lstlisting}
    Nell'ultimo caso, vengono valutati tutti i casi restanti: quello che viene eseguito è un'azione di \textit{decompose}
    sugli argomenti dei due mono-tipi.
\end{enumerate}

La funzione \texttt{argsUnification} implementa l'azione di \textit{decompose}. Per farlo, è necessario ``accoppiare" gli
argomenti dei mono-tipi e questo avviene in \texttt{unifyOnArgs} che si occupa di effettuare le eventuali chiamate
ricorsive. Senza entrare nei dettagli implementativi di \texttt{unifyOnArgs}, una volta che le chiamate ricorsive sono
state effettuate, se non vi è stato alcun errore, allora l'output sarà una sequenza di sostituzioni. Tuttavia, le
sostituzioni potrebbero non essere consistenti tra loro; si osservi il seguente esempio:
\[ \tau = M \; k \; k, \; \tau' = M \; a \; b \]
dove $ M $ è un tipo concreto, mentre gli altri termini sono tutti variabili di tipo. Effettuando un'azione di
\textit{decompose}, si ottengono le seguenti sostituzioni:
\[ S_1 = \{k \mapsto a\}, \; S_2 = \{k \mapsto b\} \]
\`E chiaro che $ S_1 $ e $ S_2 $ non sono consistenti. In questi casi, \texttt{unifyOnArgs}, invece di ritornare
l'errore \texttt{UnmatchTypes}, ritorna \texttt{TrySwap}. La semantica di quest'ultimo indica che è necessario effettuare
un'azione di \textit{swap} e riprovare con \textit{decompose}. Continuando con l'esempio precedente, otterremo:
\[ S_1 = \{a \mapsto k\}, \; S_2 = \{b \mapsto k\} \]
In questo caso, $ S_1 $ e $ S_2 $ possono essere unite in un'unica sostituzione:
\[ S = \{ a \mapsto k, \; b \mapsto k \} \]
In generale, se anche il ``secondo tentativo" non va a buon fine, l'algoritmo termina con un errore. La funzione che
implementa questa parte è \texttt{twoChancesUnifyArgs}. Chiaramente, come si può notare dall'implementazione di
\texttt{argsUnification}, se l'unificazione è \textit{strict}, allora il ``secondo tentativo" non può essere effettuato,
in quanto vi è prima un'azione di \textit{swap} che dovrebbe essere eseguita.

\hypertarget{Inferenza di lambda astrazione}{\subsubsection{Inferenza di lambda astrazione}}
La seguente regola inferisce il tipo del costrutto di lambda-astrazione:
\[ \infer[\textbf{Abs}]{\Gamma \vdash_W \lambda x. e : S\tau \mapsto \tau', S}{\tau = newvar & \Gamma, \; x : \tau \vdash_W e : \tau', \; S} \]
Nell'inferenza di questo costrutto, l'argomento $ x $ ha come tipo una nuova variabile di tipo, alla quale viene
successivamente applicata la sostituzione derivante dall'inferenza dell'espressione $ e $. Dalla regola, si rende evidente
la necessità di un tipo che consista nella ``mappatura" di un tipo verso un altro tipo.

%TODO: cfh
\hypertarget{Inferenza del costrutto ``let..in"}{\subsubsection{Inferenza del costrutto ``let..in"}}
Prima di presentare la regola di inferenza, è necessario introdurre la regola di \textit{generalizzazione}:
\[ \infer[\textbf{Gen}]{\Gamma \vdash_W e : \forall \alpha. \; \tau}{\Gamma \vdash_W e : \tau & \alpha \notin free(\Gamma)} \]
Questa regola quantifica le variabili di tipo che non compaiono già come variabili libere nell'ambiente di tipizzazione.
Per il costrutto \textit{let..in}, per indicare la generalizzazione, utilizzeremo la seguente notazione:
\[ \overline{\Gamma}(\tau) = \forall \overline{\alpha}. \; \tau \qquad \overline{\alpha} = free(\tau) - free(\Gamma) \]
Ora possiamo presentare la regola di inferenza del costrutto \textit{let..in}:
\[ \infer[\textbf{Let}]{\Gamma \vdash_W let \; x = e_0 \; in \; e_1 : \tau', \; S_1S_0}{\Gamma \vdash_W e_0 : \tau, \; S_0 & S_0\Gamma, x : \overline{S_0\Gamma}(\tau) \vdash_W e_1 : \tau', \; S_1} \]
\`E doveroso spiegare il motivo dell'esistenza del costrutto \textit{let..in}. Osserviamo il seguente esempio di
espressione scritta nel formalismo System-F:
\[ (\lambda id. \; (id \; True, id \; 42)) (\lambda x. \; x) \]
un tale programma non può essere tipato, in quanto il tipo della variabile \textit{id} non è uno schema di tipo. Si ricordi,
infatti, che nel costrutto della lambda-astrazione, il tipo dell'argomento è un mono-tipo. Per questo motivo, è
necessario il costrutto \textit{let..in}:
\[ let \; id = \lambda x. \; x \; in \; (id \; True, id \; 42) \]
infatti, il programma appena mostrato è \textit{ben tipato}, in quanto il tipo di \textit{id} è uno schema di tipo.

\hypertarget{Implementazione delle regole di inferenza}{\subsubsection{Implementazione delle regole di inferenza}}
Il codice sorgente che implementa le regole di inferenza (e quindi la type-inference) è situato nel modulo
\texttt{Compiler.Types.Builder.Type}. Alcuni costrutti di Fex si ispirano a quelli del formalismo \textit{System-F},
tuttavia, molti vengono estesi con sintassi differenti.

\hypertarget{Lambda-astrazione in Fex}{\paragraph{Lambda-astrazione in Fex}}
Prendendo come riferimento la definizione del token
dell'AST che rappresenta il costrutto di lambda-astrazione:
\begin{lstlisting}
newtype Lambda a = Lambda ([SymbolName a], Expression a, a)
\end{lstlisting}
si può notare come non vi sia un unico argomento, bensì, una lista. Stessa argomentazione vale per l'applicazione
di espressioni:
\begin{lstlisting}
newtype AppExpression a = AppExpr (Expression a, [Expression a], a)
\end{lstlisting}
Con le dovute accortezze, l'algoritmo di inferenza considera questi costrutti come l'equivalente di costrutti
concatenati in \textit{System-F}. Ad esempio:
\[ \texttt{lam x y z -> x + y + z} \equiv \lambda x. \; \lambda y. \; \lambda z. \; x + y + z \]

\hypertarget{Simboli globali in Fex}{\paragraph{Simboli globali in Fex}}
Un altra nota è necessaria sulle definizioni di simboli in Fex. Infatti, Fex permette la definizione di simboli
globali e questo ``rompe" la regola di inferenza del costrutto \textit{let..in}. Questo problema viene risolto
considerando soltanto parte delle premesse del costrutto \textit{let..in}. Si consideri, ad esempio:
\begin{lstlisting}
let x = e
\end{lstlisting}
come prima azione, viene inferito il tipo di \texttt{e}, sia esso il mono-tipo $ \tau $, poi viene effettuata la
generalizzazione, quindi $ x $ avrà tipo $ \forall \overline{\alpha}. \; \tau $. A questo punto, non è necessario
compiere altre azioni e il binding $ x : \forall \overline{\alpha}. \; \tau $ può essere aggiunto nell'ambiente di
tipizzazione.

\hypertarget{Costrutto ``let..in" in Fex}{\paragraph{Costrutto ``let..in" in Fex}}
Il costrutto \textit{let..in} in Fex ha nella sintassi un'altra estensione rispetto al suo omonimo in \textit{System-F},
ovvero i bindings accettano anche argomenti:
\begin{lstlisting}
let f x y = (x, y)
\end{lstlisting}
In questo caso, la politica dell'algoritmo di inferenza è trattare gli argomenti \texttt{x} e \texttt{y} come argomenti
di un costrutto di lambda-astrazione (cfr. Lambda-astrazione), quindi, si può pensare a questa equivalenza in termini
di type-inference con \textit{System-F}:
\[ \texttt{let f x y = (x, y)} \equiv let \; f = \lambda x. \; \lambda y. \; (x, y) \]

\hypertarget{Estensioni dell'algoritmo di inferenza}{\subsection{Estensioni dell'algoritmo di inferenza}}
Fex possiede numerosi costrutti che arricchiscono la sintassi di base di \textit{System-F}, inoltre, introduce
concetti nuovi come i predicati e i tipi qualificati. Di conseguenza, l'algoritmo di inferenza di tipo di \textit{System-F}
deve essere esteso.

\hypertarget{Pattern matching}{\subsubsection{Pattern matching}}
Fex possiede due costrutti diversi per il pattern matching
(cfr. \hyperlink{Caratteristiche del linguaggio}{Caratteristiche del linguaggio}). Il primo costrutto
permette di ``ispezionare" il valore di una singola espressione, per esempio:
\begin{lstlisting}
type Maybe a = Nothing | Just a

val f : List a -> String
let f l =
    match first l with
          Nothing -> "Error_404"
        | Just _ -> "Ok_200"
\end{lstlisting}
Il secondo costrutto, invece, non agisce su un'espressione, ma sugli argomenti di altri costrutti, per esempio:
\begin{lstlisting}
val f : Maybe a -> b -> String
let f
    | Nothing _ -> "Error_404"
    | (Just _) _ -> "Ok_200"
\end{lstlisting}
In qualsiasi caso, vi sono sempre uno o più valori da ispezionare e uno o più casi, i quali sono costituiti da una o più
\textit{case expressions} a sinistra della freccia (nel codice, spesso vengono nominate come \textit{matching expressions})
e un'espressione di ritorno a destra della freccia. Quindi abbiamo un costrutto della forma:
\[ match \; val_1 \; ... \; val_n \; with \; case_1 \; ... \; case_m \]
dove $ val_i $ è un token che definiamo come \textit{token top-level} o \textit{token di scrutinio} e
$ case_j $, che chiameremo semplicemente \textit{caso}, ha la forma:
\[ me_{1j}, \; ... \; me_{nj} \rightarrow e_j \]
L'algoritmo di inferenza di tale costrutto è il seguente:
\begin{itemize}
    \item inferisci $ val_i $, applica le sostituzioni all'ambiente di tipizzazione e poi fai lo stesso con $ val_{i+1} $,
    finché non termina la lista di token di scrutinio;
    \item nell'inferenza dei casi, dovrà valere:
        \[ \forall j. \; typeOf(val_i) \sqsubseteq typeOf(me_{ij}) \]
    ovvero le case expressions non potranno avere un tipo più generale dei corrispondenti token di scrutinio;
    \item nell'inferenza dei casi, è necessario tener traccia dei tipi dei casi già inferiti in precedenza, in
    quanto non sempre dai casi precedenti si inferisce il tipo più specializzato tra tutti i casi $ case_j $, quindi,
    i casi già inferiti verrano mantenuti;
    \item inferisci $ case_j $, applica le sostituzioni all'ambiente di tipizzazione e ai casi precedenti, poi passa
    a $ case_{j+1} $;
    \begin{itemize}
        \item nell'inferenza del caso j-esimo, il quale avrà la forma $ me_{1j} \; ... \; me_{nj} \rightarrow e_j $,
        verranno inferite, una alla volta, le case expressions $ me_{ij} $;
        \item inferisci come se fosse un argomento di una lambda astrazione $ me_{ij} $ la quale avrà mono-tipo $ \tau $,
        trova il tipo dell'i-esima case expression nei casi precedenti, sia esso $ \tau' $, poi effettua
        l'unificazione tra $ \tau $ e $ \tau' $ e applica le sostituzioni
        all'ambiente di tipizzazione e ai casi precedenti, poi continua con $ me_{i+1j} $ finché non termina la lista di
        case expressions;
        \item inferisci $ e_j $, la quale avrà mono-tipo $ \tau $, trova il tipo delle espressioni dei casi, sia esso
        $ \tau' $, effettua l'unificazione tra $ \tau $ e $ \tau' $ e applica le sostituzioni all'ambiente di tipizzazione
        e ai casi precedenti.
    \end{itemize}
\end{itemize}

\hypertarget{Eliminazione del pattern matching ``in larghezza"}{\paragraph{Eliminazione del pattern matching ``in larghezza"}}
Come è stato già menzionato precedentemente, Fex espone due sintassi diverse per effettuare il pattern matching,
una delle quali può possedere un numero arbitrario di case-expressions. Tuttavia, il token che rappresenta un
\textit{caso} del costrutto di pattern matching può avere solamente una \textit{matching expression} (equivalente
a una case expression), la sua definizione si trova in \texttt{Compiler.Ast.Typed}:
\begin{lstlisting}
data NotedCase a = NotedCase (NotedMatchExpr a) (NotedExpr a) a
\end{lstlisting}
Questo rende necessaria una fase di desugaring che avviene subito dopo l'inferenza di un costrutto di pattern matching,
quindi durante la fase di type-inference. Si consideri il seguente codice Fex:
\begin{lstlisting}
type Maybe a = Nothing | Just a
type Either a b = Left a | Right b

let f
    | Nothing _ -> 42
    | (Just x) (Left y) -> x + y
    | _ (Right y) -> y
\end{lstlisting}
Dopo aver inferito il tipo dell'espressione legata ad \texttt{f}, vengono creati gli argomenti di \texttt{f}, i quali
sono legati alle case expressions nell'espressione di pattern matching, siano essi \texttt{v1} e \texttt{v2}. Dopodiché,
vengono utilizzati i costruttori delle tuple per creare le case expressions nel costrutto di pattern matching:
\begin{lstlisting}
let f v1 v2 =
    match (v1, v2) with
          (Nothing, _) -> 42
        | (Just x, Left y) -> x + y
        | (_, Right y) -> y
\end{lstlisting}
Questo metodo ha una limitazione, ovvero non può superare il numero di argomenti massimo di un costruttore di una tupla
di Haskell, quindi è dipendente da esso. Di solito il costruttore di tuple ``più grande" in Haskell ha 62 argomenti, quindi
questa limitazione viene accettata, in quanto difficilmente l'utente definisce funzioni con più di 62 argomenti, tuttavia,
non è escluso che in futuro questa implementazione venga migliorata. \`E bene menzionare che vi è questa dipendenza da
parte di Fex, in quanto le tuple in Fex sono implementate semplicemente come tuple in Haskell. Il codice sorgente che
implementa questa funzionalità si trova in \texttt{Compiler.Desugar.BreadthPM}.

\hypertarget{Constraints}{\subsubsection{Constraints}}
La sintassi Fex permette di esprimere espressioni il cui tipo possiede dei predicati; ciò richiede che l'algoritmo
classico di type-inference venga esteso. L'algoritmo esteso di Fex si ispira a \hyperlink{bibl2}{[2]}, in cui Martin Sulzmann,
Martin Odersky, Martin Wehr presentano un'estensione di HM che supporta anche tipi che possiedono predicati. I giudizi
di tipo vengono estesi con un'ipotesi aggiuntiva sui constraints e avranno la forma:
\[ CS, \Gamma \vdash_W e : \tau \]
Perciò l'ambiente di tipizzazione viene esteso con un altro ``attore", ovvero un contesto di soddisfabilità, il quale sarà
utile a capire quali constraints possono essere accettati e quali vengono rifiutati. Nel caso
di Fex, il contesto di soddisfabilità è dato dall'insieme delle istanze.
L'articolo \hyperlink{bibl2}{[2]} presenta inizialmente il problema di come ridefinire le regole istanziazione
(\textbf{Inst}) e generalizzazione (\textbf{Gen}).
Sull'istanziazione di uno schema di tipo si arriva subito alla seguente conclusione:
\[ \infer[\textbf{Inst}]{CS, \Gamma \vdash_W e : \{ \overline{\tau} / \overline{\alpha} \}\tau'}{CS, \Gamma \vdash_W e : \forall \overline{\alpha}. D \Rightarrow \tau' & CS \vdash^e \{ \overline{\tau} / \overline{\alpha} \} D} \]
Questa regola sostiene che, quando si istanzia uno schema di tipo $ \forall \overline{\alpha}. D \Rightarrow \tau $, gli
unici mono-tipi validi sono quelli della forma $ \{ \overline{\tau} / \overline{\alpha} \} \tau' $ e tali che il constraint
$ \{ \overline{\tau} / \overline{\alpha} \}D $ \textit{è soddisfatto} da uno o più constraints nel contesto di
soddisfabilità $ CS $.
Di seguito è riportato in pseudo-codice l'algoritmo di soddisfabilità dei constraints, definito in
\texttt{Compiler.Ast.Typed}. Esso si basa sull'unificazione \textit{strict}:
\begin{lstlisting}
isSatisfied(C => args, C' => args'):
    C == C' and
    areEq(args, args')
\end{lstlisting}
dove \texttt{areEq} è l'algoritmo che effettua il test di specializzazione tra i vari tipi di \texttt{args}
e \texttt{args'}, eseguendo, successivamente, il test di uguaglianza tra mono-tipi:
\begin{lstlisting}
areEq(args, args'):
    match moreSpecOneByOne(args, args', []) with
        failure -> false
        args'' -> args == args1''

moreSpecOneByOne(args, args', resArgs):
    match args, args' with
        [] [] -> resArgs
        [] _ -> fail
        _ [] -> fail
        (ty : t) (ty' : t') ->
            let subst = isMoreSpec(ty, ty') in
            let t' = apply(subst, t') in
            let resArgs = resArgs ++ [apply(subst, ty')] in
                moreSpecOneByOne(t, t', resArgs)
\end{lstlisting}
la funzione \texttt{apply} prende in input una sostituzione e uno o più token che vengono ritornati dopo che è stata
applicata loro la sostituzione. La funzione \texttt{isMoreSpec} è invece il test di specializzazione (unificazione
\textit{strict}) tra due mono-tipi; infatti, come da definizione, gli argomenti dei constraints sono mono-tipi.
Si ricordi che il test di unificazione può fallire. In generale, diremo che un constraint $ C $ \textit{è soddisfatto} da
un constraint $ C' $ se vale $ isSatisfied(C, C') $; viceversa, diremo che un constraint $ C $ \textit{soddisfa} un
constraint $ C' $ se vale $ isSatisfied(C', C) $. Inoltre, dato il contesto di soddisfabilità $ CS $ e un constraint $ C $,
diremo che $ C $ \textit{è soddisfatto} se esiste $ C' \in CS $ tale che $ C $ è soddisfatto da $ C' $.
Si noti come, a causa dell'utilizzo dell'unificazione \textit{strict}, il test soddisfabilità tra constraints non sia
commutativo. Si osservi il seguente esempio in Fex:
\begin{lstlisting}
property Foo a b c =
    <metodi>
;;
\end{lstlisting}
Supponiamo che esistano le seguenti istanze di \texttt{Foo}:
\begin{lstlisting}
Foo Char a b
Foo Int Int Int
Foo a String Int
\end{lstlisting}
Il constraint \texttt{Foo String String Int} è soddisfatto in quanto esiste l'istanza \texttt{Foo a String Int} che
lo soddisfa. La non-commutatività si nota subito, infatti, \texttt{Foo a String Int} non sarebbe soddisfatto da
texttt{Foo String String Int}, in quanto la variabile di tipo \texttt{a} non è più specializzata (o meno generale) rispetto
al tipo concreto \texttt{String}. Il constraint \texttt{Foo x Char y}, invece, non è soddisfatto in quanto nessuna
delle tre istanze lo soddisfa.

Ora che è stata presentata formalmente la nozione di soddisfabilità di un constraint, possiamo procedere
con la ridefinizione della
regola di generalizzazione. Nell'articolo \hyperlink{bibl2}{[2]} viene proposta la seguente regola:
\[ \infer[\textbf{Gen}]{CS \wedge \exists \overline{\alpha}. D, \Gamma \vdash_W e : \forall \overline{\alpha}. D \Rightarrow \tau}{CS \wedge D, \Gamma \vdash_W e : \tau & \overline{\alpha} \notin free(\Gamma) \cup free(CS)} \]
Si noti come vi è un nuovo quantificatore: $ \exists \overline{\alpha}. D $. Questa notazione è utile per quantificare
le variabili di tipo all'interno di un constraint.

Per quanto riguarda l'algoritmo utilizzato in Fex, vi è un ulteriore contesto/ambiente di cui è necessario tener conto,
ovvero quello dei problemi di constraints correnti. Questa ulteriore estensione dell'ambiente di tipizzazione è utile
per mantenere i constraints che sono presenti nell'attuale contesto di inferenza. In seguito, vedremo in che modo
viene utilizzato. Lo indicheremo con $ \Omega $.

Indipendentemente dalle regole \textbf{Inst} e \textbf{Gen} ridefinite, è necessario fissare un comportamento che
riguarda le altre regole di inferenza. In particolare, è necessario decidere il momento in cui viene eseguito l'algoritmo
di soddisfabilità dei constraints presenti in $ \Omega $. Questa azione è detta anche \textit{normalizzazione}.
In \hyperlink{bibl2}{[2]}, vi è la proposta di nuove regole di inferenza in cui la normalizzazione viene effettuata alla
fine delle premesse di ogni regola di inferenza, eccetto quella della lambda-astrazione, in cui la normalizzazione non viene
effettuata affatto. Tuttavia, sempre nell'articolo,
vi è un'ulteriore proposta nella quale la normalizzazione viene effettuata in maniera \textit{lazy}, ovvero appena dopo
la generalizzazione ed è questa la politica utilizzata da Fex. Rimangono alcune questione aperte:
\begin{enumerate}
    \item quand'è che un predicato viene aggiunto a $ \Omega $?
    \item quali sono i constraints (se esistono) da normalizzare dopo che è stata effettuata la generalizzazione?
\end{enumerate}
Per rispondere a entrambe le domande, mostriamo come avviene l'inferenza di tipo in Fex in presenza di constraints.
Osserviamo il seguente esempio di codice Fex:
\begin{lstlisting}
property Foo a b c =
    val foo : a -> b -> c -> Char
;;

let f x y z = foo x y z
\end{lstlisting}
Andando a inferire l'espressione legata ad \texttt{f}, un token che viene incontrato è \texttt{foo}. Il suo schema di
tipo associato è $ \forall a, b, c. \; Foo \; a \; b \; c \Rightarrow a \mapsto b \mapsto c \mapsto Char $. A questo
punto, si può applicare la regola \textbf{Var}, che, a sua volta, applica la regola
\textbf{Inst}. Si ricordi che la politica di Fex sulla normalizzazione è \textit{lazy}, perciò
il constraint $ Foo \; a \; b \; c $ non viene normalizzato immediatamente, bensì viene aggiunto all'ambiente dei
problemi di constraints. $ \Omega $ si rende quindi necessario proprio a questo fine, ovvero ritardare la normalizzazione.
\`E bene notare che esso non compare nelle regole di inferenza; questo è possibile in quanto la politica \textit{lazy}
può essere considerata come \textit{side-effect} che non cambia la semantica della valutazione dei constraints.
Quindi la conclusione della regola \textbf{Var} in questo caso sarà semplicemente:
\[ CS, \Gamma \vdash_W foo : a1 \mapsto a2 \mapsto a3 \mapsto Char \]
dove $ a1 $, $ a2 $ e $ a3 $ sono variabili nuove. Continuando ad applicare le regole di inferenza \textbf{App} e
\textbf{Var} si arriverà ad avere, prima di applicare \textbf{Let} per generalizzare \texttt{f}, il seguente
ambiente di tipizzazione:
\[ \Gamma = \{ x : a1, y : a2, z : a3 \} \]
\[ \Omega = \{ Foo \; a1 \; a2 \; a3 \} \]
\[ CS = \emptyset \]
Applicando la regola \textbf{Let}, si applica anche la regola \textbf{Gen}; in questo caso, l'unico constraint
nel contesto è $ Foo \; a1 \; a2 \; a3 $ e il mono-tipo da generalizzare associato a \texttt{f} è
$ \tau = a1 \mapsto a2 \mapsto a3 \mapsto Char $. Rimane da decidere quando un constraint deve essere aggiunto allo
schema e quando, eventualmente, deve essere normalizzato. Innanzitutto, è necessario osservare le variabili libere
del mono-tipo $ \tau $, siano esse $ \overline{\alpha} = free(\tau) $. Un constraint $ C $ viene aggiunto allo schema
di tipo se:
\[ free(C) \subseteq \overline{\alpha} \wedge free(C) \neq \emptyset \]
La prima condizione richiede che tutte le variabili libere nel constraint siano presenti anche nel mono-tipo, mentre
la seconda condizione previene l'occorrenza di constraints senza variabili di tipo. Nel caso in cui la condizione non
venga rispettata, il constraint $ C $ deve essere normalizzato. Tornando all'esempio precedente, si ha che
il constraint $ Foo \; a1 \; a2 \; a3 $ può essere aggiunto allo schema di tipo che verrà generato.
\texttt{f} avrà, quindi, il seguente tipo:
\[ \forall a1, a2, a3. \; Foo \; a1 \; a2 \; a3 \Rightarrow a1 \mapsto a2 \mapsto a3 \mapsto Char \]
Si osservi ora quest'altro esempio:
\begin{lstlisting}
property Bar a b =
    val bar : a -> b -> Char
;;

let z = z

let g x y = (foo x y z, bar x y)
\end{lstlisting}
Nell'applicazione delle regole di inferenza, si ottiene che l'espressione legata a \texttt{g} ha tipo
\texttt{(Char, Char)} e prima di applicare la regola \textbf{Let}, è naturale pensare all'ambiente dei problemi di
constraints siffatto:
\[ \Omega = \{ Foo \; a1 \; a2 \; b, Bar \; a1 \; a2 \} \]
e al mono-tipo da generalizzare:
\[ \tau = a1 \mapsto a2 \mapsto Char \]
Al momento della generalizzazione, si ha che:
\[ free(Foo \; a1 \; a2 \; b) \not\subseteq free(\tau) \wedge free(Bar \; a1 \; a2) \subseteq free(\tau) \]
Di conseguenza, il constraint $ Bar \; a1 \; a2 $ può essere aggiunto allo schema di tipo, mentre
$ Foo \; a1 \; a2 \; b $ deve essere normalizzato in quanto la variabile di tipo $ b $ non compare nel mono-tipo $ \tau $.

\hypertarget{Normalizzazione}{\paragraph{Normalizzazione}}
Per quanto riguarda la normalizzazione, essa corrisponde alla seguente operazione:
\begin{lstlisting}
normalize(C, instances):
    any (fun C' -> isSatisfied(C, C')) in instances
\end{lstlisting}
Tuttavia, è necessario perfezionare l'implementazione, in quanto, al momento, non è possibile selezionare l'istanza
che soddisfa il constraint (cfr. \hyperlink{Dispatch statico}{Dispatch statico}). L'istanza che soddisfa il constraint,
infatti, deve essere unica,
in quanto è necessario non avere ambiguità sull'implementazione da scegliere. Chiaramente, se non vi sono istanze
che soddisfano un constraint, il programma viene respinto con un errore. Bisogna, quindi, fissare una semantica
di inferenza di istanza nel caso in cui vi siano due o più istanze che soddisfano un constraint. La scelta di Fex è
quella di selezionare l'istanza ``più specializzata". Definiamo ora una relazione d'ordine parziale tra constraints,
siano $ C $ e $ C' $ due constraints:
\begin{itemize}
    \item se $ isSatisfied(C, C') \wedge isSatisfied(C', C) $, allora $ C = C' $;
    \item se $ isSatisfied(C, C') \wedge \neg(isSatisfied(C', C)) $, allora $ C > C' $;
    \item se $ \neg(isSatisfied(C, C')) \wedge isSatisfied(C', C) $, allora $ C < C' $;
    \item se $ \neg(isSatisfied(C, C')) \wedge \neg(isSatisfied(C', C)) $, allora la relazione d'ordine è indefinita.
\end{itemize}
dove l'operatore $ \neg $ è la naturale negazione booleana. Di seguito, vi è definito l'algoritmo di selezione dell'istanza
ispirato da [13]:
\begin{enumerate}
    \item l'input è una sequenza $ cs $ di constraints;
    \item scarta da $ cs $ tutti quei constraints $ c $ tali che esiste $ c' \in cs $ che non sia $ c $ e tale che
    $ c' > c $;
    \item se alla fine vi rimane più di un constraint, allora non è possibile selezionare un'istanza; se, invece, vi
    è una sola istanza rimasta, allora essa viene selezionata.
\end{enumerate}
Il fatto che la relazione d'ordine sia parziale non rappresenta un problema, infatti, le istanze vengono scartate solo
se la relazione d'ordine è definita.

\hypertarget{Confronto con Haskell}{\paragraph{Confronto con Haskell}}
Fex ha un sistema di tipi molto simile a quello di Haskell, il quale, a sua volta, possiede un constraint-system
``costruito" sopra HM. Anche Haskell ha una politica \textit{lazy} di risoluzione delle istanze, ma, a differenza di Fex,
in caso di molteplici istanze che soddisfano un constraint, l'azione di default è quella di rifiutare il programma,
a meno che l'utente non utilizzi alcune estensioni del linguaggio (cfr. \hyperlink{bibl13}{[13]}).
La politica di Fex è più elastica e permette all'utente
di avere più istanze valide per certi constraints, tuttavia, ciò rappresenta una responsabilità in più da parte
dell'utente, il quale deve essere consapevole di come avviene l'inferenza di istanza. Vi è quindi un tradeoff tra
elasticità e consapevolezza dell'utente.

\hypertarget{Type-hinting}{\subsubsection{Type-hinting}}
Fex permette all'utente di indicare il tipo di un'espressione (o di un simbolo). Questo comporta una modifica
all'algoritmo di inferenza di tipo. Data un'espressione con type-hinting, lo schema generale è inferire prima
l'espressione e successivamente ``applicare" il type-hinting. Per quanto riguarda i casi ricorsivi delle espressioni
(tutti i casi fuorché i casi di base: variabili, letterali e data-constructors), il tipo proveniente dal type-hinting
viene ``scomposto" in più sotto-tipi che vengono utilizzati come type-hinting nell'inferenza dei casi ricorsivi. Tale
scomposizione è definita sui tipi funzione ed è chiamata, all'interno del codice sorgente, operazione di
\textit{unfolding}:
\begin{lstlisting}
unfoldType(ty):
    match ty with
        (ty1 -> ty2) -> ty1 : unfoldType ty2
        _ -> [ty]
\end{lstlisting}
Questa operazione è molto utile, ad esempio, nell'inferenza della lambda astrazione. Si osservi il seguente sorgente
Fex:
\begin{lstlisting}
(lam x -> f x) : Char -> String
\end{lstlisting}
Il tipo \texttt{Char -> String} viene scomposto in \texttt{Char} e \texttt{String}. Nell'inferenza dell'argomento
\texttt{x}, il type-hinting sarà esattamente \texttt{Char}, mentre per l'inferenza dell'espressione \texttt{f x}
il type-hinting sarà \texttt{String}.

Per quanto riguarda, invece, i casi di base, lo schema è:
\begin{itemize}
    \item cerca nell'ambiente di tipizzazione il tipo del token, sia esso lo schema di tipo $ \sigma $;
    \item applica la regola di istanziazione su $ \sigma $, sia $ \tau $ il mono-tipo risultante;
    \item sia $ \pi $ il mono-tipo istanziato dallo schema di tipo proveniente dal type-hinting;
    \item effettua l'unificazione \textit{strict} tra $ \pi $ e $ \tau $; si ricordi che tale operazione, a differenza
    dell'unificazione originale, non gode della proprietà commutativa, in quanto l'azione di \textit{swap} non è
    ammessa. Il mono-tipo $ \pi $ non può essere modificato dopo la sostituzione;
    \item applica la sostituzione all'ambiente di tipizzazione e al token tipato.
\end{itemize}
La regola è molto simile a quella senza type-hinting, tuttavia, viene effettuata un'unificazione in più.
Dato che l'insieme dei letterali e dei data-constructors all'interno del linguaggio non può avere variabili libere,
un'ottimizzazione che viene applicata ai loro casi è non applicare la sostituzione
all'ambiente di tipizzazione. Un'operazione che, in generale, è piuttosto dispendiosa.

\hypertarget{Ricorsione}{\subsubsection{Ricorsione}}
Il formalismo System-F non ammette la ricorsione, inoltre, non è nemmeno possibile
trovare il tipo per il cosiddetto \textit{combinatore Y}, un modo per definire la ricorsione in linguaggi che
generalmente non la supportano. Questo rende \textit{System-F} non Turing-completo, in quanto qualsiasi programma
termina. Fex (come anche Haskell e molti altri linguaggi funzionali) ammette, invece, una sintassi per la ricorsione.
Questo fatto porta con sé alcune conseguenze molto importanti. Innanzitutto, la premessa della regola \textbf{Var}
non è vera per le funzioni ricorsive, in quanto con la presenza della ricorsione i simboli possono apparire in un
programma prima ancora che il loro tipo venga generalizzato. Inoltre, sarebbe impossibile ordinare i bindings in base
alle loro dipendenze. Per questo motivo, i bindings ricorsivi vengono raggruppati
(cfr. \hyperlink{Clusters di definizioni mutualmente ricorsive}{Clusters di definizioni mutualmente ricorsive}) e
considerati come un unico binding della forma:
\[ rec \; v_1 = e_1 \; and \; v_2 = e_2 \; and \; ... \; v_n = e_n \]
Tuttavia, il problema della regola \textbf{Var} non è ancora risolto, quindi è necessario definire una nuova regola
che riguarda i bindings ricorsivi:
\[ \infer[\textbf{Rec}]{\Gamma \vdash_W rec \; v_1 = e_1 \; and \; v_n = e_n \; in \; e : \tau }{\Gamma, \Gamma' \vdash_W e_1 : \tau_1 & ... & \Gamma, \Gamma' \vdash_W e_n : \tau_n & \Gamma, \Gamma'' \vdash_W e : \tau } \]
dove:
\[ \Gamma' = v_1 : \tau_1, ..., v_n : \tau_n \]
\[ \Gamma'' = v_1 : \overline{\Gamma}(\tau_1), ..., v_n : \overline{\Gamma}(\tau_n) \]
La regola inferisce prima tutte le espressioni $ e_i $ e poi applica la regola di generalizzazione \textbf{Gen} soltanto
alla fine. Nell'algoritmo di inferenza di Fex, prima di inferire un cluster di bindings ricorsivi, vengono creati
nuovi mono-tipi della forma di singole variabili di tipo, i quali vengono assegnati ai bindings del cluster.
Durante l'inferenza,
questi mono-tipi possono essere specializzati. In questo modo, si può applicare la regola
\textbf{Var} ad ogni occorrenza di simbolo ricorsivo. L'istanziazione di un mono-tipo non rappresenta un problema
in quanto le variabili libere non vengono istanziate. In questo caso, l'algoritmo di istanziazione corrisponde
alla funzione identità.

\hypertarget{Generazione del codice}{\section{Generazione del codice}}
L'output dell'algoritmo di type inference è un \texttt{TypedProgram}, una tabella che mantiene i bindings tipati.
La generazione del codice consiste in tre sotto-fasi:
\begin{enumerate}
    \item alcuni task che riguardano il desugaring, necessari per la generazione del codice;
    \item traduzione dai token tipati di \texttt{Compiler.Ast.Typed} ai token della sintassi del linguaggio Core;
    \item esecuzione del back-end di Haskell, il quale genera il codice a basso livello di target differenti.
\end{enumerate}
Di seguito, lo schema delle fasi finali del compilatore:
\newline

\begin{tikzpicture}[node distance=2cm]
\node (typedProg) [object] {Bindings tipati};
\node (staticDispatch) [object, below of=typedProg] {Bindings tipati};
\draw [arrow] (typedProg) -- node[anchor=east] {Dispatch statico} (staticDispatch);
\node (lambdas) [object, below of=staticDispatch] {Bindings tipati};
\draw [arrow] (staticDispatch) -- node[anchor=east] {Lambda-astrazione} (lambdas);
\node (deepPm) [object, below of=lambdas] {Bindings tipati};
\draw [arrow] (lambdas) -- node[anchor=east] {Rimozione del pattern matching ``profondo"} (deepPm);
\node (scrutinee) [object, below of=deepPm] {Bindings tipati};
\draw [arrow] (deepPm) -- node[anchor=east] {Unificazione degli scrutini} (scrutinee);
\node (coreGen) [object, below of=scrutinee] {Sintassi Core};
\draw [arrow] (scrutinee) -- node[anchor=east] {Generazione della sintassi Core} (coreGen);
\node (backend) [object, below of=coreGen] {Codice finale};
\draw [arrow] (coreGen) -- node[anchor=east] {Back-end Haskell} (backend);
\node (desugar) [object, right of=lambdas, xshift=4cm] {Desugaring};
\draw [arrow] (desugar) -- (staticDispatch);
\draw [arrow] (desugar) -- (lambdas);
\draw [arrow] (desugar) -- (deepPm);
\draw [arrow] (desugar) -- (scrutinee);
\end{tikzpicture}

\hypertarget{Fasi di desugaring}{\subsection{Fasi di desugaring}}
Prima della vera e propria generazione del codice, sono necessarie alcune fasi di desugaring. Esse vengono eseguite
dopo la fase di type-inference poiché, se il programma non può essere tipato, i messaggi d'errore risultano più comprensibili
per l'utente.

\hypertarget{Dispatch statico}{\subsubsection{Dispatch statico}}
Il polimorfismo ad-hoc viene implementato attraverso il meccanismo delle type-classes.
La sintassi Core non possiede, però, costrutti per i metodi delle type-classes e per i bindings delle istanze.
La politica del compilatore Fex è quella di aggiungere i bindings delle istanze ai bindings del programma,
cambiando i nomi degli stessi metodi di istanze diverse
(cfr. \hyperlink{Costruzione delle istanze}{Costruzione delle istanze}). Questa politica ha
importanti implicazioni: innanzitutto, nelle espressioni del programma, i metodi delle istanze possono essere chiamati,
ma essi possono essere nominati solamente tramite gli identificatori dei metodi delle proprietà. Ad esempio:
\begin{lstlisting}
property Foo a =
    val foo : a -> a
;;

instance Foo String =
    let foo s = s
;;

let f x = (x, foo "42")
\end{lstlisting}
Come è stato già esposto in precedenza, il metodo \texttt{foo} dell'istanza \texttt{Foo Int} avrà un identificatore
diverso da \texttt{foo}. Tuttavia, \texttt{foo} compare nell'espressione legata a \texttt{f}. \`E necessario,
quindi, un modo per trasformare, eventualmente, le occorrenze dei metodi di proprietà con i giusti metodi di istanza.
Si pensi a quest'altro pezzo di codice Fex:
\begin{lstlisting}
type Box a = Box a

val g : Foo (Box a) => a -> Box a
let g x = foo (Box x)

let h x = g x
\end{lstlisting}
\`E chiaro come la chiamata di \texttt{foo} all'interno dell'espressione legata a \texttt{g} dipenda da quale istanza
della forma \texttt{Foo (Box a)} venga utilizzata. Vale lo stesso per \texttt{g} nei confronti dell'espressione legata
ad \texttt{h}. In linea generale, il suddetto problema è legato al polimorfismo ad-hoc e viene risolto attraverso un
metodo di dispatch statico. Il polimorfismo ad-hoc permette di creare funzioni che hanno comportamenti diversi a
seconda del loro tipo. Il dispatch statico si occupa di ``scegliere" correttamente le funzioni a disposizione nel
programma (i metodi d'istanza) che implementano i relativi metodi di proprietà nel programma. A differenza di molti
linguaggi object-oriented, in cui la selezione dei metodi avviene a run-time attraverso meccanismi di dispatch dinamici,
in Fex la selezione viene risolta completamente a compile-time, evitando così l'overhead che può essere causato
da un algortimo di dispatch dinamico. In particolare, il dispatch statico in Fex viene implementato attraverso
la \textit{monomorfizzazione} \hyperlink{bibl11}{[11]}, già introdotta, in realtà, nella costruzione delle istanze.
Ora renderemo la
monomorfizzazione generale a tutte le funzioni del programma; l'implementazione che seguirà si ispira a un articolo
sul blog di Jeremy Mikkola \hyperlink{bibl12}{[12]}. Si ripensi alla precedente funzione \texttt{g}. Il suo
schema di tipo ha con sè dei predicati. In questo caso, durante la fase di type-inference, alla funzione \texttt{g}
verrà aggiunto un parametro che corrisponderà al predicato \texttt{Foo (Box a)}, ovvero stiamo trattando il tipo di
\texttt{g} come se fosse:
\begin{lstlisting}
val g : Foo (Box a) -> a -> Box a
\end{lstlisting}
\`E necessario estendere la definizione di tipo qualificato:
\[ QT := MT \; | \; C \; | \; C \Rightarrow QT \]
Quindi, un tipo qualificato in Fex può possedere soltanto constraints. Questa eventualità viene tuttavia limitata
a pochi casi possibili e, comunque, Fex non espone una sintassi per esprimere tipi qualificati che possiedono
soltanto constraints.

Sempre durante la fase di type-inference, ogni volta che la funzione \texttt{g} viene utilizzata nel programma, oltre
alla normale applicazione della regola di inferenza \textbf{Var} e alla generazione del token tipato che rappresenta
una variabile, verrà generata e applicata a \texttt{g} un'ulteriore variabile. Quest'ultima viene associata al
relativo problema di constraint e viene salvata nell'ambiente di tipizzazione $ \Omega $. Quindi, gli elementi in
$ \Omega $ hanno, in realtà, la seguente forma:
\[ \rangle C, v \langle \]
dove $ C $ è un constraint e $ v $ è l'identificatore di una variabile, detta anche \textit{variabile di dispatch}.
Al momento della generalizzazione,
dai problemi di constraint non normalizzati vengono estratte le variabili associate, le quali diventano parte degli
argomenti del binding. Ad esempio, l'implementazione della funzione \texttt{h} a livello di token tipati sarà:
\begin{lstlisting}
let h v1 x = g v1 x
\end{lstlisting}
dove \texttt{v1} è una variabile di dispatch. Le espressioni
che contengono le variabili di dispatch sono definite in \texttt{Compiler.Ast.Typed}:
\begin{lstlisting}
data DispatchTok a =
      DispatchVar (NotedVar a) a
    | DispatchVal (OnlyConstraintScheme a) a
\end{lstlisting}
Il primo caso (\texttt{DispatchVar}) rappresenta una variabile di dispatch. Il secondo caso, invece, rappresenta un
\textit{valore di dispatch}, il quale è costituito da uno schema di tipo contenente solamente constraints ed è utile
per ``selezionare" l'implementazione giusta di una funzione.

La fase di type-inference, oltre a un programma tipato, ritorna anche una sequenza di nomi di simboli da ridefinire.
Questi ultimi sono tutti quei simboli che compaiono nel programma nella cui espressione vi sono valori di constraints.
Se in un'espressione un simbolo ha solo valori di dispatch applicati (e quindi non variabili di dispatch), allora è necessario
generare un nuovo simbolo il cui identificatore univoco viene creato a partire dai valori di dispatch, come è avvenuto
per i metodi di istanza (cfr. \hyperlink{Costruzione delle istanze}{Costruzione delle istanze}).
Nel modulo \texttt{Compiler.Desugar.AdHoc} viene implementato un algoritmo di monomorfizzazione che, ricorsivamente,
cerca nelle espressioni tutte le occorrenze di simboli a cui sono applicati valori di constraints e
genera le conseguenti implementazioni.
Ovviamente, la generazione di nuove implementazioni può dar vita a nuove espressioni con valori di constraints,
quindi l'algoritmo agisce ricorsivamente finché non esistono più implementazioni da generare:
\begin{enumerate}
    \item l'input è una serie di bindings $ bs $ in cui cercare i valori di dispatch;
    \item $ \forall b \in bs $, vengono cercati, nell'espressione legata a $ b $, tutte le occorrenze di simboli che
    hanno applicati solo valori di dispatch;
    \item ogni occorrenza viene sostituita con un nuovo identificatore univoco in tutto il programma costituito da:
        \[\langle symid \rangle \langle dispatchval_1 \rangle ... \langle dispatchval_n \rangle \]
    dove $ \langle symid \rangle $ è l'identificatore originale e $ \langle dispatchval_i \rangle $ è l'i-esimo valore di
    dispatch applicato;
    \item $ \forall sym $ che viene sostituito, $ sym $ viene aggiunto alla lista $ rs $ di simboli da creare, con le
    informazioni sui valori di dispatch che servono per creare il nuovo identificatore;
    \item $ \forall r \in rs $, viene creato un nuovo binding in cui:
    \begin{itemize}
        \item l'identificatore è costruito utilizzando il metodo menzionato precedentemente;
        \item gli argomenti di dispatch spariscono;
        \item nell'espressione associata, vengono cercate tutte le occorrenze delle variabili di dispatch, le quali
        vengono sostituite dai valori di dispatch associati a $ r $;
        \item applica i passi (3) e (4) per ogni occorrenza di simbolo che ha soltanto valori di dispatch applicati;
    \end{itemize}
    \item ripeti (5) finché non terminano i simboli in $ rs $.
\end{enumerate}

\hypertarget{Lambda-astrazione}{\subsubsection{Lambda-astrazione}}
Riportiamo la definizione di binding:
\begin{lstlisting}
type BindingSingleton a = (NotedVar a, [NotedVar a], NotedExpr a)
\end{lstlisting}
Come si può notare dalla definizione, un binding è una tripla costituita da un identificatore, degli argomenti e
un'espressione. In questa fase, vengono modificate tutte le espressioni in modo da eliminare gli argomenti dalla
tripla. Di preciso, gli argomenti del binding diventano argomenti di un costrutto di lambda-astrazione che ha come
espressione l'espressione legata del binding. Si ricordi che, durante la type inference, gli argomenti di un binding
vengono trattati come argomenti di un costrutto di lambda-astrazione
(cfr. \hyperlink{Lambda-astrazione in Fex}{Lambda-astrazione in Fex}).

\hypertarget{Rimozione del pattern matching profondo}{\subsubsection{Rimozione del pattern matching profondo}}
Si consideri il seguente sorgente Fex:
\begin{lstlisting}
type Success = Ok200
type Error = Undefined | Err401 | Err404 | Err500
type ErrType = ServerError | ClientError

type Maybe a = Nothing | Just a
type Either a b = Left a | Right b

let f x =
    match x with
          Nothing -> Just ServerError
        | Just (Right Ok200) -> Nothing
        | Just (Left Err401) -> Just ClientError
        | Just (Left Err404) -> Just ClientError
        | Just _ -> Just ServerError
\end{lstlisting}
Il codice relativo alla funzione \texttt{f} è semanticamente equivalente a:
\begin{lstlisting}
let f x =
    match x with
          Nothing -> Just ServerError
        | Just x1 ->
            (match x1 with
                  Right x2 ->
                    (match x2 with
                        Ok200 -> Nothing
                    )
                | Left x2 ->
                    (match x2 with
                          Err401 -> Just ClientError
                        | Err404 -> Just ClientError
                        | _ -> Just ServerError
                    )
            )
\end{lstlisting}
La differenza tra le due versioni è che nella seconda i data-constructors appaiono solo come teste (e non come argomenti)
nelle case expressions. In seguito, vedremo perché è importante che valga questa proprietà
(cfr. \hyperlink{Generazione del codice Core}{Generazione del codice Core}). Questa fase di desugaring trasforma le
espressioni di pattern matching, modificando le
case expressions, in modo che ad ogni loro argomento che contiene un data-contructor, viene creata una nuova
expressione di pattern matching innestata. Il codice relativo a questa fase si trova in \texttt{Compiler.Desugar.DeepPM}.

\hypertarget{Unificazione degli scrutini}{\subsubsection{Unificazione degli scrutini}}
Questa è un'altra fase di desugaring che riguarda le espressioni di pattern matching. Le motivazioni di questa
fase verranno chiarite in seguito (cfr. \hyperlink{Generazione del codice Core}{Generazione del codice Core}). Prima di
entrare nei dettagli di questa fase,
è necessario definire cosa sono le \textit{variabili di scrutinio} o più semplicemente \textit{scrutini}. Sia
$ match \; e \; with \; m_1 \rightarrow e_1 \; ... \; m_n \rightarrow e_n $ un'espressione di pattern matching,
chiameremo \textit{scrutinio} la variabile legata direttamente al valore dell'espressione $ e $. Ad esempio:
\begin{lstlisting}
let f x y =
    match g x y with
          Just v -> Just (v + y)
        | a -> Just y
        | b -> b
\end{lstlisting}
Nell'espressione legata ad \texttt{f}, le variabili di scrutinio sono \texttt{a} e \texttt{b}, mentre la variabile
\texttt{v} non lo è. Ciò che avviene nella fase di desugaring è la trasformazione delle case-expressions in modo che
una sola variabile di scrutinio venga utilizzata, quindi il codice Fex precedentemente mostrato diventerà:
\begin{lstlisting}
let f x y =
    match g x y with
          Just v -> Just (v + y)
        | v1 -> Just y
        | v1 -> v1
\end{lstlisting}
Si noti che le variabili delle sotto-espressioni vengono opportunamente aggiornate.

\hypertarget{Generazione del codice Core}{\subsection{Generazione del codice Core}}
Core è un linguaggio minimale utilizzato da GHC come rappresentazione intermedia di Haskell. La generazione
del codice Core è la fase finale del front-end del compilatore Fex. Core è una variante
esplicitamente tipata (a differenza di Haskell e Fex) del formalismo System-F \hyperlink{bibl3}{[3]}. L'obiettivo del
compilatore
Fex è costruire dei token tipati e manipolarli in modo che la traduzione dal programma tipato Fex all'equivalente
Core sia più lineare possibile. GHC espone al cliente delle API che permettono di utilizzare le funzionalità del
compilatore \hyperlink{bibl14}{[14]}, comprese quelle per creare e manipolare un programma Core. Di seguito vi è il tipo
di dato algebrico utilizzato nelle API di GHC per rappresentare un'espressione Core:
\begin{lstlisting}
data Expr b
  = Var   Id
  | Lit   Literal
  | App   (Expr b) (Arg b)
  | Lam   b (Expr b)
  | Let   (Bind b) (Expr b)
  | Case  (Expr b) b Type [Alt b]
  | Cast  (Expr b) Coercion
  | Tick  (Tickish Id) (Expr b)
  | Type  Type
  | Coercion Coercion

type Arg b = Expr b

type Alt b = (AltCon, [b], Expr b)

data AltCon
  = DataAlt DataCon
  | LitAlt  Literal
  | DEFAULT

data Bind b = NonRec b (Expr b)
            | Rec [(b, (Expr b))]
\end{lstlisting}
I commenti sono stati rimossi, cfr. \hyperlink{bibl15}{[15]} per il codice sorgente originale. Il tipo \texttt{Expr}
rappresenta
un'espressione Core, mentre l'alias di tipo \texttt{Alt} rappresenta un \textit{caso} del costrutto di pattern matching.
Il tipo \texttt{Bind}, invece, rappresenta i bindings di Core.
L'obiettivo finale è costruire un token \texttt{CoreProgram}:
\begin{lstlisting}
type CoreProgram = [CoreBind]

type CoreBind = Bind CoreBndr

type CoreBndr = Var
\end{lstlisting}
Il tipo \texttt{Var} è rappresenta le variabili in Core con un informazione sul loro tipo. Nel contesto di Core, con il
\textit{variabile} si intende un concetto di ampio spettro, infatti, \texttt{Var} comprende le variabili di un programma,
le variabili di tipo, le variabili di kind, etc.. Come si può notare dalle strutture dati utilizzate nelle API di GHC e
dalle fasi precedenti del compilatore Core, l'obiettivo del compilatore Fex è creare token tipati che possano essere
successivamente trasformati in token Core nel modo più lineare possibile. Alcuni esempi sono:
\begin{itemize}
    \item la definizione di binding di Core è molto simile a quella utilizzata dal compilatore Fex (cfr.
    \hyperlink{Approccio a tabelle}{Approccio a tabelle});
    \item i costrutti di pattern matching, dopo le varie fasi di desugaring, non possiedono costruttori innestati nelle
    case expressions, come si può notare dal costruttore di un \textit{caso} in Core:
    \begin{lstlisting}
type Alt b = (AltCon, [b], Expr b)
    \end{lstlisting}
    dove la variabile di tipo \texttt{b} rappresenterà una variabile Core (\texttt{CoreBndr});
    \item il costrutto di pattern matching di Core richiede come secondo parametro una variabile di scrutinio:
    \begin{lstlisting}
| Case  (Expr b) b Type [Alt b]
    \end{lstlisting}
    Il fatto che gli scrutini siano stati tutti unificati
    (cfr. \hyperlink{Unificazione degli scrutini}{Unificazione degli scrutini}) rappresenta un vantaggio in termini di
    implementazione;
    \item i bindings non hanno la nozione di argomenti, infatti, gli argomenti dei bindings di Fex sono stati
    trasformati in argomenti dei costrutti di lambda-astrazione (cfr. \hyperlink{Lambda-astrazione}{Lambda-astrazione});
    \item Le definizioni dei token tipati in Fex portano con sè l'informazione sul proprio tipo in modo che i token
    Core possano essere costruiti. Si ricordi che Core è un linguaggio esplicitamente tipato, quindi ogni costrutto che
    ha la nozione di tipo, ha il tipo esplicitato nella sintassi.
\end{itemize}
I dettagli implementativi dell'effettiva traduzione si trovano nel modulo \texttt{Compiler.Codegen.ToCore}.

\hypertarget{Traduzione dei tipi e dei costruttori}{\paragraph{Traduzione dei tipi e dei costruttori}}
Il modulo \texttt{Compiler.Codegen.Type} si occupa di trasformare i token di Fex che rappresentano i tipi nei
corrispettivi token di Core \hyperlink{bibl16}{[16]}. Il tipo di dato algebrico nelle API di GHC per rappresentare un tipo
in Core è \texttt{Type}. A differenza di Fex in cui per ogni nozione diversa di tipo è stato creato un token diverso,
\texttt{Type} racchiude molte nozioni di tipo, tra cui: mono-tipo, schema di tipo, variabili di tipo, etc.. Dal
punto di vista delle API di GHC, \texttt{Type} è un tipo di dato astratto, infatti, sempre il modulo \texttt{Type}
espone una serie di funzioni per costruire i tipi in Core. Altri tipi di dati fondamentali in Core sono \texttt{TyCon} e
\texttt{DataCon}, definiti rispettivamente in \hyperlink{bibl17}{[17]} e \hyperlink{bibl18}{[18]}, i quali rappresentano
rispettivamente i type-constructors e
i data-constructors. In precedenza, quando è stato presentato il tipo di approccio (a tabelle) che utilizza il
compilatore Fex
per costruire e salvare i token tipati, è stato fatto un confronto con GHC
(cfr. \hyperlink{Confronto con GHC}{Confronto con GHC}) che utilizza un
approccio molto diverso. Infatti, in GHC i token vengono salvati in una struttura a grafo. Questo metodo si
ripercuote soprattutto a livello programmatico, infatti, prendendo in considerazione \texttt{TyCon} e \texttt{DataCon},
vi è una mutua dipendenza tra i due token. Questo implica che per costruire un costruttore di tipo è
necessario avere a disposizione i data-constructors associati, mentre per costruire un data-constructor è necessario avere a
disposizione il costruttore di tipo associato. La conseguenza è la scrittura di codice mutualmente ricorsivo, tuttavia,
si ricordi che Haskell è un linguaggio con un modello di valutazione delle espressioni \textit{lazy} e questo aiuta a
prevenire la ricorsione infinita in codice mutualmente ricorsivo.

\hypertarget{Back-end}{\subsection{Back-end}}
La fase finale di compilazione di un programma Fex consiste nel compilare un programma Core in target differenti.
La pagina di documentazione ufficiale di GHC espone nei dettagli il ciclo di compilazione di un programma Haskell (e
di conseguenza di un programma Core) \hyperlink{bibl19}{[19]}. Lo schema semplificato è il seguente:
\newline

\begin{tikzpicture}[node distance=2cm]
\node (Core) [process] {Programma Core};
\node (Stg) [process, below of=Core] {Programma STG};
\draw [arrow] (Core) -- node[anchor=west] {pipeline di Core} (Stg);
\node (Cmm) [process, below of=Stg] {Programma C - -};
\draw [arrow] (Stg) -- node[anchor=west] {pipeline di STG} (Cmm);
\node (Asm) [process, below of=Cmm] {Codice assembly};
\node (C) [process, left of=Asm, xshift=-3cm] {Codice C};
\node (Llvm) [process, right of=Asm, xshift=3cm] {Codice Llvm};
\draw [arrow] (Cmm) -- (C);
\draw [arrow] (Cmm) -- (Asm);
\draw [arrow] (Cmm) -- (Llvm);
\end{tikzpicture}
\newline

STG è un'ulteriore rappresentazione intermedia di Haskell, ma, a differenza di Core, non è un linguaggio tipato.
Uno degli obiettivi più importanti di STG è rendere efficiente la valutazione delle espressioni sull'hardware standard.
C - - è un linguaggio C-like, sviluppato con l'obiettivo di essere generato dai compilatori di linguaggi ad alto livello.
Nella traduzione da un formalismo a un altro, vi sono vari passaggi (pipeline) in cui vengono effettuate, oltre alla
traduzione effettiva, alcune ottimizzazioni.

I tre target principali sono:
\begin{itemize}
    \item Codice C, che viene restituito in output come \textit{pretty-printing} dal codice C - -;
    \item Codice Assembly specifico di una macchina;
    \item Codice Llvm, portabile e ottimizzato per molte architetture.
\end{itemize}

\hypertarget{Sviluppi futuri}{\section{Sviluppi futuri}}
Fex è un linguaggio sperimentale in evoluzione. Vi sono alcune \textit{features} che arricchiscono il linguaggio che
potrebbero essere implementate in futuro. Di seguito, sono elencate alcune caratteristiche ritenute interessanti da
aggiungere al linguaggio. Nella presentazione delle funzionalità aggiuntive, verranno analizzate prima le
basi teoretiche, discutendo i vantaggi (e, eventualmente, gli svantaggi) che vengono portati al linguaggio e i problemi
che vengono risolti, dopodiché, in ogni sezione, verranno discusse una o più possibili implementazioni.

\hypertarget{Tipi lineari}{\subsection{Tipi lineari}}
Un sistema di tipi \textit{lineari} è una caratteristica dei linguaggi di programmazione (specialmente funzionali) molto
interessante, la quale porta numerosi vantaggi all'utente in termini pratici, tuttavia, non è una \textit{feature}
largamente adottata nei linguaggi mainstream. Vi è stata una proposta di implementazione in GHC \hyperlink{bibl20}{[20]} che
ha successivamente dato vita a un'estensione di Haskell. Questa sezione si ispirerà alla proposta che è stata fatta per GHC:
verranno date alcune definizioni al fine di introdurre la nozione di \textit{tipo lineare} e verrà, a sua volta, proposta una
bozza di implementazione nel compilatore Fex. Due vantaggi pratici che offre un sistema di tipi lineari sono: l'utilizzo
e il controllo della mutabilità con interfacce \textit{pure}; maggiore controllo in computazioni che riguardano l'IO.
In generale, i tipi lineari offrono maggiori garanzie all'utente programmatore.

Innanzitutto, prima di dare qualsiasi definizione, è bene precisare
come nell'articolo \hyperlink{bibl20}{[20]} l'approccio utilizzato non sia dividere i tipi in due macro-insiemi (lineari
e non-lineari), bensì la \textit{linearità} viene
associata al tipo funzione. Informalmente, una funzione è \textit{lineare} se consuma il suo argomento una e una sola
volta. Indicheremo con $ \triangleright $ il tipo funzione lineare. Il motivo per il quale la linearità viene definita
sul tipo funzione e non sulla totalità dei tipi è che questo permette di avere retrocompatibilità su un type-system già
esistente. Per quanto riguarda la nozione di \textit{valutazione} (o \textit{consumazione}) di un valore, è bene
introdurre la politica che viene utilizzata da Haskell: la \textit{valutazione lazy}. Questo tipo di valutazione delle
espressioni permette di valutare le espressioni non nel momento in cui vengono legate alle variabili,
bensì quando c'è l'effettivo bisogno del loro risultato per effettuare altre computazioni. Questa politica permette di
ritardare il più possibile la valutazione delle espressioni.
Precisiamo cosa si intende con \textit{consumare un valore esattamente una volta}, diamo questa
definizione proposta anche nell'articolo di riferimento:
\begin{itemize}
    \item per consumare un valore di tipo atomico (ovvero di un tipo con costruttore di tipo senza argomenti, ad esempio
    il tipo \texttt{Char}) esattamente una volta, lo si valuti;
    \item per consumare un valore di tipo funzione esattamente una volta, si applichi a esso un argomento e consuma
    il risultato esattamente una volta;
    \item per consumare un valore di un tipo di dato algebrico esattamente una volta, si esegua il pattern matching su
    di esso e si consumi tutte le sue componenti lineari esattamente una volta.
\end{itemize}
Con questa definizione possiamo rispondere alla seguente domanda sul tipo dei data-constructors: quale tipo
dovremmo assegnare a un data-constructor? Prendiamo come esempio il costruttore delle coppie \texttt{(,)}:
\begin{itemize}
    \item $ (,) : \forall \alpha, \beta. \; \alpha \; \triangleright \; \beta \; \triangleright \; (\alpha, \beta) $
    \item $ (,) : \forall \alpha, \beta. \; \alpha \mapsto \beta \mapsto (\alpha, \beta) $
\end{itemize}
La prima scelta è quella corretta, infatti, se il risultato dell'espressione \texttt{(,) x y} è consumato esattamente
una volta, allora, per definizione, \texttt{x} e \texttt{y} vengono consumati esattamente una volta.

Questo sistema di tipi può portare, tuttavia, ad alcuni problemi. Si pensi, come viene mostrato in un esempio in
\hyperlink{bibl20}{[20]}, alla seguente implementazione della funzione \texttt{map} di Haskell:
\begin{lstlisting}
map _ [] = []
map f (x : t) = f x : map f t
\end{lstlisting}
La funzione \texttt{map} può avere i seguenti due tipi:
\begin{itemize}
    \item $ map : \forall \alpha, \beta. \; (\alpha \; \triangleright \; \beta) \mapsto [\alpha] \; \triangleright \; [\beta] $
    \item $ map : \forall \alpha, \beta. \; (\alpha \mapsto \beta) \mapsto [\alpha] \mapsto [\beta] $
\end{itemize}
I due tipi sono incompatibili tra loro. Introduciamo, quindi, un livello di polimorfismo sulla
\textit{molteplicità} dei tipi funzione:
\begin{itemize}
    \item $ map : \forall \alpha, \beta, p. \; (\alpha \mapsto_p \beta) \mapsto [\alpha] \mapsto_p [\beta] $
\end{itemize}
Con la notazione appena mostrata, stiamo asserendo che \texttt{map} accetta come argomenti una funzione con molteplicità
illimitata (ovvero non-lineare) e una lista con molteplicità $ p $, inoltre, la funzione in input accetta un argomento
di molteplicità $ p $. Definiamo formalmente la molteplicità:
\[ \pi := 1 \; | \; \omega \; | \; p \; | \; \pi + \pi \; | \; \pi * \pi \]
dove $ \omega $ è la molteplicità illimitata e $ p $ è una variabile di molteplicità. Inoltre, valgono le seguenti
relazioni affermazioni:
\begin{itemize}
    \item $ + $ e $ * $ sono associativi e commutativi;
    \item $ * $ ha precedenza maggiore di $ + $;
    \item $ \forall p. \; p * 1 = 1 * p = p $;
    \item $ \omega * \omega = \omega $;
    \item $ 1 + 1 = 1 + \omega = \omega + \omega = \omega $.
\end{itemize}
Consideriamo $ \mapsto \; \equiv \; \mapsto_{\omega} $ e $ \triangleright \; \equiv \; \mapsto_1 $. Si noti come valga
la seguente affermazione:
\[ \forall \pi, \mu. \; \pi + \mu = \omega \]
Tuttavia, se aggiungiamo l'elemento neutro di $ + $, sia esso $ 0 $, allora la precedente affermazione non vale più.
Dato questo sistema di tipi lineari, è ora necessario aggiornare l'algoritmo di type inference in modo che inferisca
anche le molteplicità dei vari tipi funzione. Per questo, in \hyperlink{bibl20}{[20]} viene definito un lambda-calcolo
le cui regole
d'inferenza inferiscono la molteplicità corretta dei tipi funzione. Non specificheremo di seguito le regole d'inferenza,
tuttavia, è bene fare alcune precisazioni. Innanzitutto, nelle regole di inferenza, si può pensare ai tipi nell'ambiente
di tipizzazione come l'input di un giudizio e alle molteplicità come l'output.
Esistono due regole diverse per quanto riguarda le variabili e i data-constructors e questo è dato dalla diversa
definizione di linearità per le funzioni e per i data-constructors. Per quanto riguarda, invece, il costrutto di
lambda-astrazione, visto che deve essere creato un tipo funzione e a priori non si conosce la molteplicità di questo
tipo, a esso viene associata una variabile di molteplicità, la quale non deve esistere già come variabile libera. Per la
regola di applicazione, è necessario guardare la
molteplicità della freccia (nel tipo dell'espressione che applica) e a seconda della sua molteplcità, le variabili libere
nell'argomento dell'applicazione dovranno essere utilizzate con la stessa molteplicità.

\subsubsection{Bozza di implementazione in Fex}
Ora che è stato presentato un sistema di tipi lineari, si può delineare una bozza di implementazione in Fex.
Il sistema di tipi lineari da aggiungere a quello già esistente di Fex si ispira a quello appena presentato,
riutilizzando le varie nozioni incontrate. Ma prima di aggiornare il sistema di tipi di Fex, è doveroso definire
la politica di valutazione delle espressioni in Fex (al momento indefinita e dipendente dall'implementazione).
Una scelta che può essere coerente con il sistema di tipi lineari da aggiungere è quella di una politica \textit{lazy}
come quella di Haskell. Questo comporta che la traduzione dei token tipati in codice Core sia tale che le espressioni
di Fex legate alle variabili non vengano valutate al momento della definizione delle variabili, bensì, la loro
valutazione venga ritardata fin quando ci sarà una computazione all'interno del programma che necessiterà il loro
valore. Fatta questa importante premessa, ora è possibile discutere l'estensione del sistema di tipi di Fex.
Un primo aggiornamento che può essere effettuato è alla nozione di mono-tipo:
\[ MT \; := \; \alpha \; | \; T \; | \; MT \; MT \; | \; MT \mapsto_{\pi} MT \]
Dove $ \pi $ è la nozione di molteplicità fornita in precedenza (anch'essa da aggiungere al sistema di tipi).
Dopodiché, sarà necessario estendere l'algoritmo di type inference, in particolare:
\begin{itemize}
    \item è necessario aggiungere un ``attore" all'ambiente di tipizzazione, sia esso $ \Pi $, il quale associa ai
    token una molteplicità. Anche in questo ambiente sarà necessario tener conto delle variabili libere, in quanto
    la nozione di molteplicità stessa può possedere variabili di molteplicità;
    \item estendere le regole di inferenza in modo che, oltre alla normale type inference, eseguano anche l'inferenza
    di molteplicità. Si presume che, come per i tipi, in Fex le molteplicità possano o non possano essere esplicitate;
    \item per quanto riguarda le regole di inferenza, al momento della generalizzazione, sarà necessario generalizzare
    anche le eventuali variabili di molteplicità che non compaiono libere in $ \Pi $.
\end{itemize}

\hypertarget{Effetti}{\subsection{Effetti}}
Gli \textit{effetti algebrici} rappresentano un modo per ottenere più controllo su comportamenti impuri da parte dei
programmi. Essi si basano sul fatto che tali comportamenti impuri derivano da un insieme di \textit{operazioni} come,
per esempio, \texttt{read} e \texttt{write} per la lettura e la scrittura su file, \texttt{raise} per le eccezioni o
\texttt{fork} e \texttt{wait} per la concorrenza.
L'idea alla base è avere un sistema di tipi che comprende una nuovo insieme di tipi, detti \textit{tipi effetti}, in
grado di inglobare l'insieme di \textit{operazioni} che portano \textit{side-effects} (di qualsiasi genere). Linguaggi
come Koka \hyperlink{bibl21}{[21]} implementano un sistema di tipi che supporta anche gli \textit{effetti}. In questa
sezione, verranno
discusse le principali caratteristiche di un sistema di \textit{effetti}, seguendo l'articolo \hyperlink{bibl22}{[22]},
poi verrà proposta un'implementazione nel compilatore Fex, analizzando i cambiamenti da portare e gli eventuali ostacoli.

Nell'articolo \hyperlink{bibl22}{[22]} viene, innazitutto, fissato un linguaggio. L'approccio utilizzato qui è lo stesso,
tuttavia, il linguaggio che verrà definito sarà leggermente diverso da quello nell'articolo:
\[ v := e \; | \; fun \; x \mapsto c \; | \; h \]
\[ h := handler \{ return \; x \mapsto c, \; op_1(x;k) \mapsto c_1, \; ..., \; op_n(x;k) \mapsto c_n \} \]
\[ c := return \; v \; | \; op(v;y.c) \; | \; do \; x \leftarrow c_1 \; in \; c_2 \; | \; v1 \; v2 \; | \; with \; v \; handle \; c \]
dove $ e $ sono i costrutti di System-F, estesi anche con valori costanti. $ v $ rappresenta i \textit{valori},
$ h $ rappresenta gli handler e $ c $ le computazioni.
\`E necessaria una discussione di alcuni costrutti appena definiti:
\begin{itemize}
    \item il costrutto $ return \; v $, dove $ v $ è un valore, rappresenta la computazione pura ed è utile ogni qual
    volta viene utilizzato un valore in un contesto in cui ci si aspetta una computazione;
    \item la chiamata di un'\textit{operazione} $ op(v;y.c) $ passa un parametro $ v $ all'operazione $ op $, il
    risultato viene legato a $ y $ e, infine, viene valutata la computazione $ c $, chiamata anche $ continuazione $.
    Un'operazione rappresenta un'esecuzione con side-effect;
    \item il costrutto $ do \; x \leftarrow c_1 \; in \; c_2 $ è chiamato \textit{sequenziazione}. Viene valutata
    prima la computazione $ c_1 $, poi il risultato viene legato alla variabile $ x $ e, infine, si valuta $ c_2 $.
    Questo costrutto non è da confondersi con la chiamata di un'operazione, in quanto, sebbene simili, la sequenziazione
    è utile per mettere in sequenza qualsiasi computazione;
    \item un \textit{handler} definisce delle azioni da eseguire al momento della chiamata di certe operazioni
    (specificate dall'handler stesso). L'handling della computazione pura è opzionale;
    \item il costrutto $ with \; v \; handle \; c $ valuta la computazione $ c $ utilizzando l'handler nel valore
    $ v $ per gestire le computazioni in $ c $;
\end{itemize}
Nel formalismo appena presentato, vi è una divisione tra computazioni pure e impure. A questo punto, si può definire
un sistema di tipi. Per farlo, estenderemo il sistema di tipi di System-F:
\[ VT := PT \; | \; PT \rightsquigarrow CT \; | \; CT \Rrightarrow CT \]
\[ CT := VT ! \{ op_1, \; ..., \; op_n \} \]
\`E doveroso fare alcune precisazioni riguardo a queste definizioni. Innanzitutto, $ VT $ è il tipo dei valori, mentre
$ CT $ è il tipo delle computazioni. Con $ PT $ si intende uno schema di tipo, è indifferente se il suo caso base sia
un tipo qualificato (cfr. \hyperlink{Schemi di tipo e polimorfismo parametrico}{Schemi di tipo e polimorfismo parametrico})
oppure un mono-tipo. Il caso $ CT \Rrightarrow CT $ rappresenta il tipo degli
handler, invece, il caso $ PT \rightsquigarrow CT $ rappresenta il tipo funzione che ha come tipo di ``ritorno" un tipo
di computazione. Le regole di inferenza per il formalismo in questione sono abbastanza intuitive e non verranno riportate
(esse sono comunque definite in \hyperlink{bibl22}{[22]}). I tipi delle computazioni racchiudono le possibili operazioni
che fanno parte dell'effetto. Questo fatto permette di catturare in maniera precisa i side-effects prodotti dalle operazioni.

\subsubsection{Bozza di implementazione in Fex}
Un sistema di effetti offre un meccanismo generale per gestire le computazioni impure. Fex è un linguaggio
funzionale puro e, al momento, non ha un meccanismo per gestire tutto ciò che non riguarda il ``mondo puro". Un sistema
di effetti potrebbe essere un modo per aggiungere computazioni impure, come input/output, eccezioni, multi-threading,
mantenendo separato il ``mondo puro" da quello ``impuro".

Per aggiungere un sistema di effetti a Fex, è necessario, innanzitutto, aggiungere una nuova sintassi per tutti quei
costrutti necessari all'implementazione degli effetti, in particolare bisogna aggiungere:
\begin{itemize}
    \item una sintassi per definire gli effetti, ovvero i tipi delle computazioni. Questo diventa opzionale se alcuni
    effetti vengono considerati come built-in, tuttavia, ciò limiterebbe molto le possibilità del programmatore che non
    sarebbe in grado di estendere l'insieme degli effetti;
    \item una sintassi per definire le operazioni, ma questo dovrebbe essere una conseguenza della possibilità di
    definire effetti;
    \item una sintassi per definire gli handler, infatti, senza di essi, non sarebbe possibile gestire i side-effects;
    \item una sintassi che permette di utilizzare gli handler. Nel formalismo presentato, un costrutto del genere è
    rappresentato da $ with \; v \; handle \; c $;
\end{itemize}
Dopodiché, è necessario definire un sistema di effetti. Un possibile approccio è quello di Koka \hyperlink{bibl23}{[23]},
il quale, potendo
inferire tutti i side-effects in una computazione, connota il tipo ritorno con gli effetti opportuni. In particolare,
l'assenza di side-effects è connotata con $ total $, l'effetto per le funzioni matematiche. In ogni caso, si rende
necessaria un'estensione del sistema di tipi di Fex, infatti, dati i nuovi costrutti del linguaggio, bisogna quanto
meno avere un tipo per le computazioni e un tipo per gli handler. L'aggiornamento delle regole di inferenza avviene di
conseguenza, tuttavia, per quanto riguarda queste ultime, vale la pena fare alcune precisazioni. Per tipizzare
una computazione nel modo giusto, è necessario eseguire un'analisi statica della computazione e individuare tutte le
operazioni definite in un effetto. Inoltre, è possibile eseguire l'analisi statica degli handler e decidere se un handler
gestisce tutte le operazioni utilizzate in una computazione. Tuttavia, se, come in Koka, vi sono effetti che comprendono,
ad esempio, anche la divergenza di un programma, non esiste un'operazione specifica associata all'effetto di divergenza.
In questo caso, è necessario un'analisi statica specifica, infatti, l'unico ``costrutto" con cui si ottiene la divergenza
in Fex è la ricorsione,
quindi, ogni qual volta vi sono uno o più bindings ricorsivi, il loro tipo di ritorno può essere l'effetto divergenza.
Il fatto che un qualche simbolo $ f $ abbia come tipo l'effetto divergenza non implica che la valutazione di $ f $
provochi l'effettiva divergenza del programma. Questo perché il problema della fermata non è decidibile.

Una volta che sono stati generati gli eventuali token tipati, si rende necessario estendere la generazione del codice.
Anche Haskell possiede un meccanismo per dividere il mondo puro da quello impuro: viene utilizzata la monade \texttt{IO},
la quale ingloba un token che rappresenta lo stato del mondo impuro. A differenza di un sistema ad effetti come quello
presentato precedentemente, questo meccanismo utilizzato da Haskell è meno fine, in quanto tratta la (quasi) totalità
delle computazioni impure con la monade \texttt{IO}. Questo implica un potenziale ostacolo: la ``compilazione" dei
side-effects potrebbe dipendere da come viene ``compilata" la monade \texttt{IO}. In ogni caso, è necessario un
\textit{mapping}
dagli effetti definiti nel linguaggio ai ``token" di basso livello del compilatore. A questo proposito, esiste un
modulo delle API di GHC (\texttt{PrimOp}) che espone tutte quelle operazioni che non possono essere definite in un
sorgente Haskell.

\hypertarget{Varianti polimorfe}{\subsection{Varianti polimorfe}}
Le \textit{varianti polimorfe} rappresentano un meccanismo di estendibilità dei tipi, in particolare esse risolvono
il task di, dato un tipo, estenderlo con nuove funzioni e nuovi costruttori. Questo è un problema ricorrente nella
programmazione e viene risolto anche dai linguaggi orientati agli oggetti tramite meccanismi di sotto-tipaggio e
ereditarietà, tuttavia, le varianti polimorfe costituiscono un modo più conciso ed elegante di estendere i tipi.
Esse sono state implementate in OCaml, ma non sono diffuse in molti altri linguaggi di programmazione. Esse rompono
il vincolo informale secondo il quale un data-constructor appartiene a uno e un solo type-constructor; si ricordi che
tale vincolo è valido in Fex e Haskell. L'implementazione OCaml offre anche una nuova sintassi per contraddistinguere
le varianti polimorfe (il loro identificatore è preceduto dal carattere backtick \`).
In questa sezione, verranno brevemente presentati i principali vantaggi delle varianti polimorfe, dopodiché, non verrà
presentata un'implementazione in Fex, bensì verranno discusse alcune proposte di implementazioni provenienti da
altre fonti.

Come mostra Garrigue in \hyperlink{bibl24}{[24]}, uno dei vantaggi principali delle varianti polimorfe è quello già
citato precedentemente: dopo aver fissato un tipo di dato, estendere le API dei costruttori del tipo di dato. Si osservi il
seguente pseudo-codice:
\begin{lstlisting}
type HttpCode = Ok200 | Err404 | Err500
\end{lstlisting}
dove \texttt{HttpCode} è un tipo e \texttt{Ok200}, \texttt{Err404} e \texttt{Err500} sono data-constructors associati
solamente al tipo \texttt{HttpCode}. In seguito diremo che, dato un data-constructor $ dc $ e un type-constructor $ tc $,
$ dc \in tc $, se $ dc $ è associato a $ tc $ e a nessun'altro construttore di tipo. Poniamo che un utente stia
sviluppando il codice per un server http e, inizialmente, rappresenti i codici di risposta http con il tipo
\texttt{HttpCode}. Un metodo per evitare di riscrivere il codice relativo al server ogni volta che è necessario un nuovo
codice di risposta è definire a priori \texttt{HttpCode} con tutti i possibili codici di risposta http. \`E chiaro che è una
soluzione parecchio onerosa per l'utente e molti dei codici potrebbero rimanere inutilizzati. Inoltre,
alcune funzioni implementate nel codice del server potrebbero gestire insiemi molto ristretti di codici http, ad esempio:
\begin{lstlisting}
let handleReq json =
    let res = process(json);
    lookupAndSend res

val lookupAndSend : (HttpCode, String) -> HttpAction ()
let lookupAndSend res =
    match res with
          (Ok200, msg) -> send(msg)
        | (Err404, path) -> send("path " ++ path ++ " not found")
        | (Err500, _) -> send("Excuse me, my fault")

let server =
    listen(fun req -> handleReq(req))
\end{lstlisting}
Sarebbe veramente tedioso per l'utente fare pattern matching nella funzione \texttt{lookupAndSend} su tutti i codici
http esistenti. Utlizzando le varianti polimorfe, possiamo isolare i casi da gestire per \texttt{lookupAndSend}:
\begin{lstlisting}
val lookupAndSend : ({< `Ok200, `Err404, `Err500}, String) -> HttpAction ()
let lookupAndSend res =
    match res with
          (`Ok200, msg) -> send(msg)
        | (`Err404, path) -> send("path " ++ path ++ " not found")
        | (`Err500, _) -> send("Excuse me, my fault")
\end{lstlisting}
Utilizziamo la sintassi \texttt{\{< `Ok200, `Err404, `Err500\}} per indicare che il tipo è polimorfo (anche in assenza
di variabili di tipo) e che può essere ristretto in seguito. Supponiamo di avere poi il seguente pezzo di codice:
\begin{lstlisting}
val possibleRes : List {> `Ok200, `Err404, `Err500}
let possibleRes = [`Ok200, `Err404, `Err500]
\end{lstlisting}
Anche in questo caso, il tipo \texttt{\{> `Ok200, `Err404, `Err500\}} è polimorfo, ma, a differenza di prima, questo
tipo può essere esteso con altri varianti polimorfe. Successivamente, l'utente potrebbe aggiungere il seguente codice:
\begin{lstlisting}
val defaultRes : List {> `Err400, `Tmp307, `Ok200, `Err404, `Err500}
let defaultRes = [`Err400, `Tmp307] ++ possibleRes
\end{lstlisting}
Al di là della sottile differenza tra \texttt{<} e \texttt{>} che riguarda la tipizzazione delle varianti polimorfe,
questo è un esempio di come possono essere utilizzate le varianti polimorfe per estendere un tipo. Un altro effetto
benefico portato dall'uso delle varianti polimorfe è la condivisione di tipi di dato tra librerie inizialmente separate.
Infatti, senza le varianti polimorfe, sarebbe necessario costruire l'astrazione per ``far comunicare" le
librerie nel modo giusto, utilizzando un attore di terze parti che funga da interfaccia comune tra le librerie coinvolte.
Invece, con le varianti polimorfe, è possibile definire lo stesso tipo nelle librerie coinvolte, permettendo
alle librerie di essere indipendenti tra loro.

\subsubsection{Implementazioni}
In letteratura, vi sono numerose proposte di implementazione delle varianti polimorfe. Di seguito, ne elencheremo alcune,
analizzando eventuali vantaggi e svantaggi:
\begin{itemize}
    \item come propone inizialmente Garrigue \hyperlink{bibl24}{[24]}, una possibile compilazione delle varianti polimorfe
    può essere effettuata
    facendo un mapping da varianti a valori numerici interi. Questa soluzione è semplice da implementare e piuttosto
    efficiente, tuttavia, presenta un grosso svantaggio: è necessario conoscere tutte le varianti polimorfe di un programma.
    Il problema nasce se il linguaggio permette la compilazione separata di sorgenti, infatti, a quel punto non è possibile
    conoscere contemporaneamente tutte le varianti polimorfe in un programma;
    \item sempre Garrigue \hyperlink{bibl24}{[24]} propone un'implementazione altrettanto efficiente e che risolve il
    problema della
    compilazione separata: un mapping dai nomi delle varianti ai loro valori hash. Sebbene questo metodo sia semplice,
    efficiente e permetta la compilazione separata, presenta anch'esso uno svantaggio: cosa succede se due varianti
    polimorfe diverse hanno lo stesso valore hash? Questo evento è piuttosto raro (se si utilizza una ``buona" funzione hash)
    e proprio per questo risulta un ottimo compromesso tra i vantaggi che porta e i casi ``sfortunati", tuttavia, è
    necessario comunque gestire questa eventualità. Garrigue \hyperlink{bibl24}{[24]} propone di emettere un errore a tempo
    di compilazione,
    il quale è sempre una scelta migliore di un errore a run-time. \`E chiaro che l'utente dovrà cambiare manualmente
    l'implementazione dei sorgenti. Questo diventa particolarmente sconveniente quando i sorgenti in questione non sono
    stati scritti dall'utente stesso, ma sono esterni;
    \item Kagawa \hyperlink{bibl25}{[25]} tratta l'implementazione delle varianti polimorfe in Haskell e propone una
    soluzione di più
    ``alto livello": un mapping dalle varianti polimorfe alle type-classes di Haskell. Questa soluzione, come vedremo, è
    piuttosto ``elegante" in quanto non è necessario estendere il linguaggio, tuttavia, presenta qualche ostacolo.
    Presentiamo, prima di tutto, la sintassi scelta da Kagawa \hyperlink{bibl25}{[25]}:
\begin{lstlisting}
variant cs => a in VariantName ts where
    Constr_1 :: ty_11 -> ... -> ty_1n -> a
    ...
    Constr_m :: ty_m1 -> ... -> ty_mn -> a
\end{lstlisting}
    Le varianti polimorfe sono rappresentate dai simboli con suffisso \texttt{Constr}, i quali hanno degli identificatori
    che iniziano con una lettera maiuscola. Si può notare come la dichiarazione di una variante sia molto simile a quella
    di un type-class. Tuttavia, vi è una restrizione, ovvero che la variabile di tipo \texttt{a} deve apparire come tipo
    di ritorno di ogni variante polimorfa. Il contesto \texttt{cs} specifica le superclassi, che, a loro volta, dovranno
    essere delle varianti. \texttt{ts} è una sequenza di argomenti (variabili di tipo) come per le type-classes.
    Il costrutto per definire una variante subisce la seguente trasformazione:
    \newline
    prima:
\begin{lstlisting}
variant cs => a in VariantName ts
\end{lstlisting}
    dopo:
\begin{lstlisting}
class cs => VariantName a ts | a -> ts
\end{lstlisting}
    Come si nota dal codice, sono state utilizzate due estensioni (le quali hanno buon supporto da parte dei compilatori
    Haskell): \texttt{MultiParamTypeClasses} e \texttt{FunctionalDependencies}.
    Kagawa \hyperlink{bibl25}{[25]} estende, successivamente, la sintassi Haskell con altri due tipi di
    dichiarazioni (i record e le istanze di varianti) e spiega come sia necessario un aggiornamento dell'algoritmo di
    inferenza di tipo. In ogni caso, questo metodo di implementazione ha il vantaggio di essere direttamente traducibile
    (escludendo la modifica all'algoritmo di inferenza di tipo) in codice Haskell, quindi, risulta essere solamente
    zucchero sintattico.
\end{itemize}

\section{Conclusione}
Il paradigma funzionale permette, in generale, la scrittura di codice componibile e dichiarativo e ciò comporta una
maggiore manutenibilità ed espandibilità dei software (talvolta anche a discapito delle prestazioni). Un sistema di tipi
come HM offre parecchie garanzie e questo previene comportamenti non voluti da parte dei programmi, compresi errori
a run-time. Linguaggi di programmazione \textit{debolmente} tipati quali Javascript o C forniscono poche garanzie in termini
di sistema di tipi. Questo porta ad avere un minor controllo sui programmi e da ciò conseguono una serie di problematiche che
spesso riguardano anche la sicurezza. Altri linguaggi, come Python o Java, che vengono considerati \textit{fortemente} tipati,
non prevengono comunque una vasta gamma di errori che riguardano i tipi e che vengono identificati soltanto a run-time,
avendo come effetto negativo che tali errori, se non gestiti, portano a un crash dei programmi. Fex si propone come linguaggio
funzionale puro, inteso come in grado di isolare le computazioni deterministiche. Fex si propone altresì di gestire il
``mondo impuro" tramite il sistema di tipi con una granularità più fine rispetto a quella di Haskell
(si ricordi che, per gestire i comportamenti ``impuri", Haskell utilizza, in generale, la monade \texttt{IO}), arricchendo il
sistema di tipi stesso.
Quest'ultimo obiettivo non è stato raggiunto, in quanto Fex risulta momentaneamente un linguaggio che calcola solamente funzioni
matematiche.

\section*{Bibliografia}
\begin{enumerate}[label={[\arabic*]}]
    %[1]
    \item \hypertarget{bibl1} Amr Sabry - What is a purely functional language? - in: Journal of Functional Programming, Volume 8, Issue 1,
    January 1998, pp. 1 - 22
    %[2]
    \item \hypertarget{bibl2} Martin Sulzmann, Martin Odersky, Martin Wehr - Type Inference with Constrained Types - in: Theory and Practice
    of Object Systems · January 1999
    %[3]
    \item \hypertarget{bibl3} Stephen Diehl - Dive into GHC: Targeting Core - url: \url{https://www.stephendiehl.com/posts/ghc_03.html} -
    ultima visita: 21/02/2023
    %[4]
    \item \hypertarget{bibl4} libreria Parsec - url: \url{https://hackage.haskell.org/package/parsec} - ultima visita: 21/02/2023
    %[5]
    \item \hypertarget{bibl5} Data types for Haskell entities - url:
    \url{https://gitlab.haskell.org/ghc/ghc/-/wikis/commentary/compiler/entity-types} - ultima visita: 21/02/2023
    %[6]
    \item \hypertarget{bibl6} Edward Y. Zang - Backpack without symbol tables - in: May 11, 2016 - url:
    \url{http://web.mit.edu/~ezyang/Public/backpack-symbol-tables.pdf} - ultima visita: 21/02/2023
    %[7]
    \item \hypertarget{bibl7} Haskell Flexible Instances extension - url:
    \url{https://ghc.gitlab.haskell.org/ghc/doc/users_guide/exts/instances.html#extension-FlexibleInstances} -
    ultima visita: 21/02/2023
    %[8]
    \item \hypertarget{bibl8} Haskell instance resolution termination conditions - url:
    \url{https://ghc.gitlab.haskell.org/ghc/doc/users_guide/exts/instances.html#instance-termination-rules} -
    ultima visita: 21/02/2023
    %[9]
    \item \hypertarget{bibl9} Martin Sulzmann, Gregory J. Duck, Simon Peyton-Jones, Peter J. Stuckey - Understanding Functional Dependencies
    via Constraint Handling Rules - in: Journal of Functional Programming, Volume 17, Issue 1, January 2007, pp. 83 - 129
    %[10]
    \item \hypertarget{bibl10} Definizione di ClsInst in GHC, version 8.10.7 - url:
    \url{https://hackage.haskell.org/package/ghc-8.10.7/docs/InstEnv.html#t:ClsInst} - ultima visita: 21/02/2023
    %[11]
    \item \hypertarget{bibl11} Monomorphization - url:
    \url{https://doc.rust-lang.org/book/ch10-01-syntax.html#performance-of-code-using-generics} - ultima visita: 21/02/2023
    %[12]
    \item \hypertarget{bibl12} Type inference for Haskell, part 15 - url:
    \url{https://jeremymikkola.com/posts/2019_01_15_type_inference_for_haskell_part_15.html} - ultima visita: 21/02/2023
    %[13]
    \item \hypertarget{bibl13} Overlapping instances in Haskell - url:
    \url{https://ghc.gitlab.haskell.org/ghc/doc/users_guide/exts/instances.html#overlapping-instances} -
    ultima visita: 21/02/2023
    %[14]
    \item \hypertarget{bibl14} The GHC API, version 8.10.7 - url: \url{https://hackage.haskell.org/package/ghc-8.10.7} -
    ultima visita: 21/02/2023
    %[15]
    \item \hypertarget{bibl15} CoreSyn, version 8.10.7 - url: \url{https://hackage.haskell.org/package/ghc-8.10.7/docs/CoreSyn.html} -
    ultima visita: 21/02/2023
    %[16]
    \item \hypertarget{bibl16} Type, version 8.10.7 - url: \url{https://hackage.haskell.org/package/ghc-8.10.7/docs/Type.html} -
    ultima visita: 21/02/2023
    %[17]
    \item \hypertarget{bibl17} TyCon, version 8.10.7 - url: \url{https://hackage.haskell.org/package/ghc-8.10.7/docs/TyCon.html} -
    ultima visita: 21/02/2023
    %[18]
    \item \hypertarget{bibl18} DataCon, version 8.10.7 - url: \url{https://hackage.haskell.org/package/ghc-8.10.7/docs/DataCon.html} -
    ultima visita: 21/02/2023
    %[19]
    \item \hypertarget{bibl19} GHC Commentary: The Compiler - url: \url{https://gitlab.haskell.org/ghc/ghc/-/wikis/commentary/compiler/} -
    ultima visita: 21/02/2023
    %[20]
    \item \hypertarget{bibl20} Jean-Philippe Bernardy, Mathieu Boespflug, Ryan R. Newton, Simon Peyton Jones, Arnaud Spiwack - Linear Haskell -
    in: arXiv:1710.09756
    %[21]
    \item \hypertarget{bibl21} Effect Typing - url: \url{https://koka-lang.github.io/koka/doc/book.html#why-effects} - ultima visita: 21/02/2023
    %[22]
    \item \hypertarget{bibl22} Matija Pretnar - An Introduction to Algebraic Effects and Handlers. Invited tutorial paper - in: Electronic
    Notes in Theoretical Computer Science, Volume 319, 21 December 2015, Pages 19-35
    %[23]
    \item \hypertarget{bibl23} Effect types - url: \url{https://koka-lang.github.io/koka/doc/book.html#sec-effect-types} -
    ultima visita: 21/02/2023
    %[24]
    \item \hypertarget{bibl24} Jacques Garrigue - Programming with Polymorphic Variants - url:
    \url{https://caml.inria.fr/pub/papers/garrigue-polymorphic_variants-ml98.pdf} - ultima visita: 21/02/2023
    %[25]
    \item \hypertarget{bibl25} Koji Kagawa - Polymorphic Variants in Haskell - in: Haskell '06: Proceedings of the 2006 ACM SIGPLAN workshop
    on Haskell, 17 Semptember 2006, Pages 37-47
    %[26]
    \item \hypertarget{bibl26} Data.Graph - url:
    \url{https://downloads.haskell.org/ghc/latest/docs/libraries/containers-0.6.6/Data-Graph.html} -
    ultima visita: 21/02/23
    %[27]
    \item \hypertarget{bibl27}
    Hindley, R. (1969). The Principal Type-Scheme of an Object in Combinatory Logic. Transactions of the American
    Mathematical Society, 146, 29–60. - url: \url{https://doi.org/10.2307/1995158} - ultima visita: 21/02/23
    %[28]
    \item \hypertarget{bibl28}
    Robin Milner - A theory of type polymorphism in programming - in: Journal of Computer and System Sciences, Volume 17,
    Issue 3, December 1978, Pages 348-375
    %[29]
    \item \hypertarget{bibl29}
    Luis Damas, Robin Milner - Principal type-schemes for functional programs - in: POPL'82: Proceedings of the
    9th ACM SIGPLAN-SIGACT symposium on Principles of programming languages, January 1982, Pages 207-212
    %[30]
    \item \hypertarget{bibl30} Inside Erlang, The Rare Programming Language Behind Whatsapp's Success -
    url: \url{https://www.fastcompany.com/3026758/inside-erlang-the-rare-programming-language-behind-whatsapps-success} -
    ultima visita: 25/02/23
    %[31]
    \item \hypertarget{bibl31} Cardano docs - url: \url{https://docs.cardano.org/introduction} - ultima visita: 25/02/23
    %[32]
    \item \hypertarget{bibl32} Hack - url: \url{https://hacklang.org/} - ultima visita: 25/02/23
    %[33]
    \item \hypertarget{bibl33} Flow - url: \url{https://flow.org/} - ultima visita: 25/02/23
    %[34]
    \item \hypertarget{bibl34} Template Meta-programming for Haskell - in: Haskell '02: Proceedings of the 2002 ACM SIGPLAN
    workshop on Haskell, 3 October 2002, Pages 1-16
    %[35]
    \item \hypertarget{bibl35} The LLVM Compiler Infrastructure - url: \url{https://llvm.org/} - ultima visita: 25/02/23
    %[36]
    \item \hypertarget{bibl36} Fex-lang GitHub repository - url: \url{https://github.com/bogo8liuk/Fex-lang} -
    ultima visita: 27/02/23
\end{enumerate}

\end{document}
